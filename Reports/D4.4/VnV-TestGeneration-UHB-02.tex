
\subsubsection{Model-based Test Generation for the ETCS Ceiling Speed Monitor}
\label{sec:csmunibremen}

\paragraph{Contributing project partners}
% Usually, one main partner, perhaps with contributions from others
Main contribution by University of Bremen, additional contributors: DLR and Siemens 

\paragraph{Process step}
% Classification of the activity according to D2.3a
% Name what is verified and to which report this would contribute.
% Use the numbers, e.g. System Design Verification Report (1-12).
This activity is part of the SW Design (Phase 4). It
contributes to the SW Component Test
Specification (4-22). 

\paragraph{Object of verification}
% Which openETCS artifact (github link) or other documents/programs
% etc. (provide references)
The object of verification are implementations of the ETCS
Ceiling Speed Monitor (CSM).  


\paragraph{Available specification}
% The specification against which the object is to be checked. Usually
% coming form some openETCS artifacts (GitHub reference, process
% artifact number) or background material (reference, artifact number).

The specification of speed and distance monitoring in \cite[Sec.~3.13]{subset-026:3.3.0}.

\paragraph{Objective}
%In an ordinary development, the main objective would be to verify or validate
%something. Here, in openETCS, it would be quite common to demonstrate
%the applicability of some method/tool, or evaluate its capabilities.

The main objective is to evaluate and demonstrate the new input
equivalence class partition test generation method developed by the
team of the University of Bremen. The method guarantees 100 per cent
error detection inside a fault domain, and is expected to provide high
coverage outside the domain. Its results on the CSM are compared with the
relevant system test cases as defined in then ETCS standard conformity
test specification, Subset~076.


\paragraph{Method/Approach}
% Short description of how the verification/validation is performed

A test model specifying the expected behaviour of the CSM has been
developed in SysML, using state machines and block diagrams.  The
model elements have been linked to the associated ETCS system
requirements.  Since this SysML language subset can be associated with
a formal semantics, it is possible to execute algorithms that
automatically generate sets of executable test cases from the
model. These sets of test cases permit to check implementations for
compliance with the model. The tracing information enable to derive
detailed coverage and fault identification information.

The existing SUBSET-076 test cases were formalised using linear
temporal logic (LTL), so that the same test data generation concept
could be applied as for the test cases that were automatically
identified: SUBSET-076 test cases do not provide concrete test data
for every test step, but specify the general constraints from which
concrete data can be elaborated.  This approach also allows to trace
the model coverage achieved by the SUBSET-076 test cases.

All tests were executed against software mutants derived from a
reference implementation, using 3 different mutation generators in
order to avoid a mutation bias. For each testing strategy applied it
was checked
\begin{itemize}
\item which parts of the test model were covered by the test execution, and
\item which fault coverage (percentage of ``killed'' mutants) was achieved.
\end{itemize}





\paragraph{Means/Tools}
% Could be integrated with the previous paragraph. Assign an
% approrpiate tool class (T1, T2, T3) according to EN 50128. Try to assign a
% maturity level 
%https://github.com/openETCS/validation/blob/master/VerificationAndValidationPlan/V02/VnVUsrStrTmplt-140709-02.pdf

The whole approach is fully supported by RT-Tester and its model-based
testing component RTT-MBT. Test cases are described by LTL
formulas. An integrated SMT-solver generate solutions for the LTL
formulas which add  concrete data and makes the test cases
executable. From the SysML test
model, the tool automatically derives LTL
formulas which describe the test cases. For the SUBSET-076 test cases,
the LTL formulas have been provided manually and completed by the
solver. 

\paragraph{Results}
% Results related to the objective.
% Refer to appropriate document (preferably GitHub) for more complete description.
The results can be summarised as follows.
\begin{enumerate}
\item The new equivalence class testing method shows significantly
  higher test strength than all other methods used in the
  comparison. It achieved nearly 100\% fault coverage for mutants
  outside the fault domain (mutants inside the fault domain are always
  killed, due to the guaranteed fault detection properties).

\item The new method is very well suited for software testing and
  HW/SW integration testing, where the high number of test cases
  (approx.~5000 cases) can easily be executed, in particular, because
  the test suite is fully automated. The new method, however, yields
  too many test cases to be applied on system testing level with real
  trains on real tracks.


\item The SUBSET-076 test cases are missing 2 cases for the CSM in order to achieve
requirements coverage. These can be easily identified and added. As a result,
these test comprise 11 cases.

\item With the missing test cases added, the SUBSET-076 achieve only a fault coverage of
62\% -- this would certainly not suffice to obtain certification credit. It is
possible, however, to add an acceptable number of test cases to the SUBSET-076
suite for the CSM which would significantly increase its test strength.


\end{enumerate}


All   results have been published in 
\begin{itemize}
\item Jan Peleska and Wen-ling Huang: Complete model-based equivalence
  class testing. Int J Softw Tools Technol Transfer. Published online:
  21 November 2014. DOI 10.1007/s10009-014-0356-8.

\item Felix H\"ubner, Wen-ling Huang, and Jan Peleska: Experimental
  Evaluation of a Novel Equivalence Class Partition Testing
  Strategy. In Jasmin Christian Blanchette and Nikolai Kosmatov
  (eds.): Tests and Proofs - 9th International Conference, TAP 2015,
  Held as Part of STAF 2015, L'Aquila, Italy, July 22-24,
  2015. Proceedings. Lecture Notes in Computer Science 9154, Springer,
  2015, pp. 155-172, doi 10.1007/978-3-319-21215-9\_10.

\item C{\'e}cile Braunstein, Anne E. Haxthausen, Wen-ling Huang, Felix
  H\"ubner, Jan Peleska, Uwe Schulze, and Linh Vu Hong: Complete
  Model-Based Equivalence Class Testing for the ETCS Ceiling Speed
  Monitor. In S. Merz and J. Pang (eds.): Proceedings of the ICFEM
  2014. Springer, LNCS 8829, pp. 380-395, 2014. DOI
  10.1007/978-3-319-11737-9\_25.


\item Technical Report http://www.informatik.uni-bremen.de/agbs/testingbenchmarks/
\newline
openETCS/ceiling-speed-monitoring/testing\_the\_etcs\_csm.pdf


\item C{\'e}cile Braunstein, Wen-ling Huang, Felix H\"ubner, Jan
  Peleska, and Uwe Schulze: Evaluation of Model-Based Testing
  Strategies for the ETCS Ceiling Speed Monitor.  Submitted to
  Software Testing, Verification and Reliability journal.

Also available as technical report
\end{itemize}

\paragraph{Observations/Comments}
% An optional section where anything can be included which has been
% observed without direct connection

It is interesting to note that typical model-coverage driven test
cases (e.g. transition coverage, MC/DC coverage), while achieving
higher model coverage than the SUBSET-076 tests, do not achieve much
higher fault coverage (approx.~68\%).  The reason is that these test
cases are not invariant under syntactic model transformations: with
another -- through semantically equivalent -- model, higher or lower
test strength would be achieved with the coverage-driven test cases
derived from that model.

In contrast to that, the new equivalence class testing strategy is
elaborated from the {\it semantic} representation of the model and is
therefore invariant (i.e.~always maximal) under all syntactic model
transformations that leave the behavioural semantics unchanged.

Verified Systems International GmbH who maintain the commercial
version of RT-Tester have won the runner-up trophy of the EU
Innovation Radar Innovation Prize\footnote{see {\tt
    https://www.verified.de/publications/papers-2015/\newline
    eu-innovation-radar-price-runner-up-trophy-for-verified-systems-international/}}
for implementing the equivalence class testing strategy described
above in the commercial version of RT-Tester.

\paragraph{Conclusion}
%What has been achieved, what is missing, what has been learned

The new test strategy has shown to provide superior test strength when
compared to SUBSET-076 test cases and conventional model-coverage
driven test cases that are typically provided by other model-based
testing tools. As of today, RT-Tester is the only testing tool where
the new test strategy is implemented.


\subsubsection{Model-based Testing of the ETCS Target Speed Monitor}
\label{sec:targetspeedmonitorbremen}

\paragraph{Contributing project partners}
% Usually, one main partner, perhaps with contributions from others
Main contribution by University of Bremen, additional contributors: DLR and Siemens 

\paragraph{Process step}
% Classification of the activity according to D2.3a
% Name what is verified and to which report this would contribute.
% Use the numbers, e.g. System Design Verification Report (1-12).
This activity is part of the SW Design (Phase 4). It
contributes to the SW Component Test
Specification (4-22). 

\paragraph{Object of verification}
% Which openETCS artifact (github link) or other documents/programs
% etc. (provide references)

The object of verification is an implementation of the target speed monitoring function of the EVC,
see technical report
\begin{itemize}
\item[[~1]] Felix H{\"u}bner, Christoph Hilken, and Jan Peleska.
Combination of Behavioral and Parametric Diagrams for Model-based
Testing -- Application to ETCS Target Speed Monitoring. Submitted to DAC 2016, also available as 
technical openECTS report 2014-11-25.

Available under \url{https://github.com/openETCS/validation/tree/master/VnVUserStories/VnVUserStoryUniBremen/04-Results}
\end{itemize}


\paragraph{Available specification}
% The specification against which the object is to be checked. Usually
% coming form some openETCS artifacts (GitHub reference, process
% artifact number) or background material (reference, artifact number).
ETCS system specification, SUBSET-026-3;
model parts are also available in [1]. The whole target speed monitoring model will be made available on {\tt http://www.mbt-benchmarks.org}.

\paragraph{Objective}
%In an ordinary development, the main objective would be to verify or validate
%something. Here, in openETCS, it would be quite common to demonstrate
%the applicability of some method/tool, or evaluate its capabilities.
For creating a SysML test model of the target speed monitoring function, both time-discrete
(e.g.~trigger of the emergency brakes) and time-continuous (e.g.~time-dependent 
train location, speed, and acceleration) variables 
need to be considered. SysML state machines
are suitable for modelling concurrent real-time behaviour of time-discrete control
functions. For time-continuous aspects, the report [1] describes how to use
parametric constraints and associated diagrams for modelling. It is also explained
how the parametric specifications are made available to the SMT solver creating 
concrete test data from models. As a result, the solver generates data that 
complies with the time-continuous physical constraints of the model. 


\paragraph{Method/Approach}
% Short description of how the verification/validation is performed

Parametric constraints represent a language aspect of the SysML which has not yet
been fully investigated in the research communities. Using so-called constraint
blocks, these constraints can be specified. Typically, parametric constraints 
represent system invariants or -- this is the relevant aspect for the target speed monitor --
physical laws, such as acceleration-dependent speed and speed-dependent location. 
For our application, these laws also comprise the ETCS braking curves modelling the
speed changes of the braking train.
Parametric constraints can be specified using general physical variables; these are 
bound to concrete model variables using parametric diagrams. 

It is shown in [1] how parametric constraints can be used to calculate physically
meaningful train behaviours, that is, meaningful changes of speed and location over
time, taking into account the braking actions. The method follows a 2-step approach: 
first, a model abstraction is created, and  the equivalence class testing 
strategy described in Section~\ref{sec:csmunibremen} is used to identify
test cases with guaranteed fault detection properties. Next, the calculated tests
are refined with respect to time-dependent behaviour, so that still the same 
equivalence classes  are used, but the representatives for location and speed
are selected in a way that complies with the physical laws.

 


\paragraph{Means/Tools}
% Could be integrated with the previous paragraph. Assign an
% approrpiate tool class (T1, T2, T3) according to EN 50128. Try to assign a
% maturity level 
%https://github.com/openETCS/validation/blob/master/VerificationAndValidationPlan/V02/VnVUsrStrTmplt-140709-02.pdf

The method has been implemented in the RT-Tester tool as part of the WP7-related
activities of the University of Bremen team.

\paragraph{Results}
% Results related to the objective.
% Refer to appropriate document (preferably GitHub) for more complete description.

The results show that the method can be automatically performed with acceptable 
computation time. 

\paragraph{Observations/Comments}
% An optional section where anything can be included which has been
% observed without direct connection

To our best knowledge, this is the first SysML-based method
for calculating test data with guaranteed fault detection properties in presence of
both time-discrete and time-continuous observables.


\paragraph{Conclusion}
%What has been achieved, what is missing, what has been learned

The method developed here is highly relevant for testing cyber-physical systems
in general. Verified Systems International GmbH who maintain the commercial version
of RT-Tester has already decided to make this method available in 2016.