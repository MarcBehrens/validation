

\subsubsection{Basic Verification of the Implementation of ``Speed and
  Distance Monitoring'' }
\label{sec:}

\paragraph{Contributing project partners}
% Usually, one main partner, perhaps with contributions from others
This work has been performed by the University of Rostock.

\paragraph{Process step}
% Classification of the activity according to D2.3a
% Name what is verified and to which report this would contribute.
% Use the numbers, e.g. System Design Verification Report (1-12).

This activity is part of Phase~5, SW Component Implementation and Test. It
addresses the requirement 5.1 (basic tests performed by the
Implementer). 


\paragraph{Object of verification}
% Which openETCS artifact (github link) or other documents/programs
% etc. (provide references)

The object of verification is the SCADE package
\texttt{SpeedSupervision\_Integration} incorporating the sub packages
\texttt{CalcBrakingCurves}, \texttt{SDM\_Commands},
\texttt{SDM\_GradientAcceleration}, \texttt{SDM\_Models},
\texttt{SDM\_TargetLimits} and \texttt{TargetManagement}, together
with the helper and type package \texttt{SDM\_Types}.

\paragraph{Available specification}
% The specification against which the object is to be checked. Usually
% coming form some openETCS artifacts (GitHub reference, process
% artifact number) or background material (reference, artifact number).

The specification of speed and distance monitoring is given in
\cite[Sec.~3.13]{subset-026:3.3.0}. 
%Subset-026 (v3.3) Figure 28. 

\paragraph{Objective}
%In an ordinary development, the main objective would be to verify or validate
%something. Here, in openETCS, it would be quite common to demonstrate
%the applicability of some method/tool, or evaluate its capabilities.

The objective is to establish that the code performs the expected
functionality in standard (expected) usage scenarios. It is not
meant to be exhaustive. Also, it shall check for performance and
memory usage abnormalities. 

\paragraph{Method/Approach}
% Short description of how the verification/validation is performed

The test has been performed by integrating the package
\texttt{SpeedSupervision\_UnitTest\_Pkg} into the main module which
provides the implementation code with inputs of a usage scenario. The
test simulates a linear movement via generated odometry inputs,
providing constant default national values, reasonable train data and
a generic track that consists of one constant speed profile, a single,
non-extending movement authority and a null-gradient. The correctness
of the implementation reaction is checked mainly manually. 

\paragraph{Means/Tools}
% Could be integrated with the previous paragraph. Assign an
% appropriate tool class (T1, T2, T3) according to EN 50128. Try to assign a
% maturity level 
%https://github.com/openETCS/validation/blob/master/VerificationAndValidationPlan/V02/VnVUsrStrTmplt-140709-02.pdf

The test has been performed with the simulation functions (SCADE Suite
Simulator) of the SCADE Suite Advanced Modeler, Version 6.1. Since the
basic verification is to complemented by a full verification, the tool
qualification level is (only) T1.  

\paragraph{Results}
% Results related to the objective.
% Refer to appropriate document (preferably GitHub) for more complete description.

The test has been applied to each version of the
implementation. Only implementation versions which passed the test had
been given clearance for integration.


% \paragraph{Observations/Comments}
% % An optional section where anything can be included which has been
% % observed without direct connection



% The test case is relatively static, simulating a linear movement via
% generated odometry inputs, providing constant default national values,
% reasonable train data and a generic track that consists of one
% constant speed profile, a single, non-extending movement authority and
% a null-gradient. The test bench does not compare against expected
% values and heavily relies on human inspection. Without a multitude of
% speed profiles (MRSP) this test case cannot cover multiple EBD
% curves. This is up to larger test setups that also provide better data
% visualizations.

\paragraph{Conclusion}
%What has been achieved, what is missing, what has been learned

The test established readiness for integration testing. It does not
cover the full
verification of the code according to the Component Test Specification
(4-22). 