

\subsubsection{Verification of Modes and Levels Management function (Systerel) }
\label{sec:}

\paragraph{Contributing project partners}
% Usually, one main partner, perhaps with contributions from others
The work has been performed by Systerel

\paragraph{Process step}
% Classification of the activity according to D2.3a
% Name what is verified and to which report this would contribute.
% Use the numbers, e.g. System Design Verification Report (1-12).


This activity contributes to:
\begin{itemize}
\item the Software Requirement Specification  (3.16)
\item the Software Specification Verification Report  (3.18)  
\item the Software Architecture and Design Specification (4.19)
\item the Software Design Verification Report  (4.23)  
\item the Software Components (5.24)
\item the Software Component Verification Report  (5.26)  
\end{itemize}



\paragraph{Object of verification}
% Which openETCS artifact (github link) or other documents/programs
% etc. (provide references)

The object of verification is the Scade model for the modes and levels management function at {\url{https://github.com/openETCS/modeling/tree/master/model/Scade/System/ObuFunctions/ManageLevelsAndModes}}. 


\paragraph{Available specification}
% The specification against which the object is to be checked. Usually
% coming form some openETCS artifacts (GitHub reference, process
% artifact number) or background material (reference, artifact number).

The model implements the requirements of modes and levels management function, as described in {\itshape System Requirements Specification, Chapter 4 and 5}.
All the chapter are not covered, only the part related to  the definition of current mode and level.


\paragraph{Objective}
%In an ordinary development, the main objective would be to verify or validate
%something. Here, in openETCS, it would be quite common to demonstrate
%the applicability of some method/tool, or evaluate its capabilities.


The goal is to produce a Scade model, to generate automatically executable C code, and to  verify properties on Scade model by model-checking.



\paragraph{Method/Approach}
% Short description of how the verification/validation is performed

At first, a formal model is defined with the Scade suite tool.
Then functional properties are formally verified on the model by  model-checking.

Finally, C code is automatically translated from the Scade model.

\paragraph{Means/Tools}
% Could be integrated with the previous paragraph. Assign an
% approrpiate tool class (T1, T2, T3) according to EN 50128. Try to assign a
% maturity level 
%https://github.com/openETCS/validation/blob/master/VerificationAndValidationPlan/V02/VnVUsrStrTmplt-140709-02.pdf


The means used are:
\begin{itemize}
\item Scade suite to  design, check and simulate the model 
\item Systerel Smart Solver (S3) to perform model-checking (can be certified as T2 tool)
\item KCG translator to produce C code (Code translator shall be T3 level to obtain certified code).
\end{itemize}


\paragraph{Results}
% Results related to the objective.
% Refer to appropriate document (preferably GitHub) for more complete description.


\begin{itemize}
\item The result is a Scade model of the mode and level management function integrated in the whole EVC scade model.
\item The C code has been automatically  generated.
\item Some properties have been checked by model checking.
\end{itemize}

See {\url{https://github.com/openETCS/validation/blob/master/Reports/D4.3/D4.3.1-Final-VV-report-on-model/D4.3.1.pdf}} and {\url{https://github.com/openETCS/validation/blob/master/VnVUserStories/VnVUserStorySysterel/04-Results/e-Scade_S3/Scade_S3_VnV.pdf}}.



\paragraph{Observations/Comments}
% An optional section where anything can be included which has been
% observed without direct connection


\paragraph{Conclusion}
%What has been achieved, what is missing, what has been learned



Benefits of formal  methods in an a posteriori verification process of critical systems has been recognized by our industrial customers:

\begin{itemize}
\item Contrarily to a human generated test-based verification solution, a formal safety verification is
intrinsically complete. It is equivalent to search for every possible falsification.
\item It clearly identifies the complete list of assumptions upon which the safety relies.
\item A certified solution allows for a reduction of the testing and review efforts (only the generic safety
specification has to be reviewed).
\item The use of formal verification in the qualification of critical software sends a strong
and positive message to the market, and is sometime even a requirement for some customers.
\end{itemize}

