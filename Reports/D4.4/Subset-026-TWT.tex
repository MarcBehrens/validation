\subsubsection{Verification of Chapter~5 of Subset~026 (TWT)}
\label{sec:verif-Subset-026}

\paragraph{Contributing project partners}
% Usually, one main partner, perhaps with contributions from others
The work has been performed by TWT.

\paragraph{Process step}
% Classification of the activity according to D2.3a

This activity is part of the verification of the Elaborated System
Requirements which are based on Subset~026 \cite{subset-026:3.3.0}. It
contributes to the System Design Verification Report (1-12). In
formalizing and analyzing the procedures it findings contribute also
to the definition of the Elaborated System Requirements themselves
(1-07).

\paragraph{Object of verification}
% Which openETCS artifact (github link) or other documents/programs
% etc. (provide references)

The object of verification are the procedures defined in Chapter~5 of
Subset~026 \cite[5]{subset-026:3.3.0}. \textit{NN} of the \qq{25}
procedures have been analyzed. 

\paragraph{Available specification}
% The specification against which the object is to be checked. Usually
% coming form some openETCS artifacts (GitHub reference) or background
% material (reference).
The procedures are checked for consistency. They are not checked
against an external specification.
%This example is specific!

\paragraph{Objective}
%In an ordinary development, the main objective would be to verify or validate
%something. Here, in openETCS, it would be quite common to demonstrate
%the applicability of some method/tool, or evaluate its capabilities.
The main objective w.r.t.\ verification is check the procedure definitions for
consistency and some sanity conditions. A by-product are
formalizations which can enter the Elaborated System Requirements (1-07). 


\paragraph{Method/Approach}
% Short description of how the verification/validation is performed
The control flow of the procedures is modeled with colored Petri nets
(CPNs) in the tool \cite{Westergaard2013apn}. Each model is checked
independently by a second person. The necessity of formalization
coming with the modeling uncovers inconsistencies in textual
specifications. With the help of the simulation and checking
facilities of the CPN tools, sanity conditions on the models are
checked.

\paragraph{Means/Tools}
% Could be integrated with the previous paragraph. Assign an
% approrpiate tool class (T1, T2, T3) according to EN 50128. Try to assign a
% maturity level 
%https://github.com/openETCS/validation/blob/master/VerificationAndValidationPlan/V02/VnVUsrStrTmplt-140709-02.pdf
The CPN tools are ...

\paragraph{Results}
% Results related to the objective.
% Refer to appropriate document (preferably GitHub) for more complete description.
The modeling and analysis uncovered 36 inconsistencies, ambiguities
and gaps in the Subset~026 which were reported
in~\cite{specfindings}. \cmmnt{to be revised/completed}

\paragraph{Observations/Comments}
% An optional section where anything can be included which has been
% observed without direct connection


\paragraph{Conclusion}
%What has been achieved, what is missing, what has been learned

\cmmnt{Missing procedures?}
The numerous specification findings illustrate the need for validating
the specification. CPNs are well-suited to model the behavioral
aspects described in Subset-026 chapter 5. The size of the model
clearly indicates the complexity of the procedures, even at the
current level of abstraction. The main benefit comes from the
activity of formalization itself, and of incomplete, but valuable,
simulations. 
