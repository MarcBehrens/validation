\documentclass{template/openetcs_article}
% Use the option "nocc" if the document is not licensed under Creative Commons
%\documentclass[nocc]{template/openetcs_article}
\usepackage{lipsum,url}
\usepackage{supertabular}
\usepackage{multirow}
\usepackage{color, colortbl}
\definecolor{gray}{rgb}{0.8,0.8,0.8}
\usepackage[modulo]{lineno}
\graphicspath{{./template/}{.}{./images/}}
\begin{document}
\frontmatter
\project{openETCS}

%Please do not change anything above this line
%============================



% The document metadata is defined below

%user specified macros


\newcommand{\VV}{Verification \& Validation\xspace}
\newcommand{\vv}{verification \& validation\xspace}

\def\CC{{C\nolinebreak[4]\hspace{-.05em}\raisebox{.4ex}{\tiny\bf ++}}}

\newcommand{\adhesion}{\mbox{\inl{AdhesionFactor}}\xspace}

\newcommand{\bitwalker}{\mbox{\texttt{Bitwalker}}\xspace}

\newcommand{\poke}{\mbox{\texttt{Bitwalker\_Poke}}\xspace}
\newcommand{\peek}{\mbox{\texttt{Bitwalker\_Peek}}\xspace}
\newcommand{\acsl}{\mbox{\textsf{ACSL}}\xspace}
\newcommand{\isoc}{\mbox{\textsf{C}}\xspace}
\newcommand{\framac}{\mbox{\textsf{Frama-C}}\xspace}
\newcommand{\framacwp}{\mbox{\textsf{Frama-C\slash WP}}\xspace}
\newcommand{\why}{\mbox{\textsf{Why}}\xspace}
\newcommand{\wpframac}{\mbox{\textsf{WP}}\xspace}
\newcommand{\altergo}{\mbox{\textsf{Alt-Ergo}}\xspace}
\newcommand{\qed}{\mbox{\textsf{Qed}}\xspace}
\newcommand{\cvc}{\mbox{\textsf{CVC4}}\xspace}
\newcommand{\z}{\mbox{\textsf{Z3}}\xspace}
\newcommand{\coq}{\mbox{\textsf{Coq}}\xspace}
\newcommand{\cealist}{\mbox{\textsf{CEA LIST}}\xspace}
\newcommand{\fokus}{\mbox{\textsf{Fraunhofer FOKUS}}\xspace}

\newcommand{\inl}[1]{\lstinline[style=inline]{#1}}


%Background color of boxes in process graphic
\definecolor{light-gray}{gray}{0.95}

%assign a report number here
\reportnum{OETCS/WP4/D4.4V0.1}


%define your workpackage here
\wp{Work-Package 4: ``Verification \& Validation Strategy''}

%set a title here
\title{openETCS Final Report on Verification and Validation}

%set a subtitle here
\subtitle{}

%set the date of the report here
\date{December 2015}

%document approval
%define the name and affiliation of the people involved in the
%documents approbation here 
\creatorname{Marc Behrens}                                                                                                                                
\creatoraffil{ Deutsches Zentrum für Luft und Raumfahrt e.V.}                    
\techassessorname{[assessor name]}
\techassessoraffil{[affiliation]}

\qualityassessorname{Jan Welte}
\qualityassessoraffil{TU Braunschweig}

\approvalname{Klaus-R\"udiger Hase}
\approvalaffil{DB Netz}

%define a list of authors and their affiliation here


\author{Marc Behrens \and Hardi Hungar}


\affiliation{DLR\\
  Lilienthalplatz 7\\
  38108 Brunswick, Germany
%   \\eMail:hardi.hungar@dlr.de 
}



% define the coverart
\coverart[width=350pt]{openETCS_EUPL}

%define the type of report
\reporttype{Final Report}

\begin{abstract}
%define an abstract here
  This document summarizes the approach, scope and result of the
  verification and validation activities in the project openETCS.
\end{abstract}

%=============================
%Do not change the next three lines
\maketitle

%Modification history
%if you do not need a modification history table for your document
%simply comment out the eight lines below 
%=============================
\section*{Modification History}
\tablefirsthead{
\hline 
\rowcolor{gray} 
Version & Section & Modification / Description & Author \\\hline}
\begin{supertabular}{| m{1.2cm} | m{1.2cm} | m{6.6cm} | m{4cm} |}
 0.0 & all & initial & Marc Behrens \\\hline
 0.1 & all & revision and addition & Hardi Hungar \\\hline
\end{supertabular}

\tableofcontents
\listoffiguresandtables
\newpage
%=============================
%Uncomment the next line if you need line numbers for tracebility when the document is in review
%\linenumbers

%=============================
% The actual document starts below this line
%=============================

%Start here

\section{Introduction}

According to \cite[3.1.48]{EN50128:2011}, verification is an activity
to check whether the output of a development phase meets the
requirements. This concerns formalities, traceability, and, w.r.t.\
the main content, completeness, correctness and consistency. Within
openETCS, examples of each kind of verification have been
performed. Thereby, also new methods and tools have been evaluated and
adapted. 

Validation concerns the compliance of the end result of the
development with the user requirements. This has been done employing
the demonstrator of the EVC software. 

This document summarizes the activities described in more detail in
separate reports. It explains how these separate activities fit into
the development process of openETCS as defined in the deliverable
D2.3a.    

Most verification activities are actually reviews of documents (or
even programs). For general review activities, a process has been
defined in \cite{openETCS:D1.3.1}. 

\newpage







%Examples are below
\section{Verification and Validation in the Development Lifecycle}
\label{sec:Lifecycle}

\begin{figure}[hbt]
  \centering
  \def\svgwidth{.9\textwidth}
  {\tiny
  \input{Prcss2_3a-03.pdf_tex}}
  \caption{openETCS Development Lifecycle}
  \label{fig:lifecycle2}
\end{figure}

Fig.~\ref{fig:lifecycle2} is an overview of the openETCS development
lifecycle, taken from D2.3a. It depicts the process for a complete
development of the EVC software, of which a part has been performed
within the project. Verification, resp., validation, has to be done in
each of the phases of the development.

\section{Overview of Verification and Validation Activities}
\label{sec:overview}

\bgcmmnt{Some sample notes are included in the subsections. To be
  checked for correct assignment to the phases, extended to become
  self-contained summaries of the activities with results and
  contributions. Note: Also evaluating a new verification method is a
  contribution to be mentioned, if this is a side or main effect of
  the activity. Do not forget to add yourself as an author if you contribute.}

\subsection{Verification and Validation in the Planning Phase}
\label{sec:vnv-0}

There have been reviews of the planning documents \cmmnt{compile a
  list}. 

\cmmnt{\textbf{Template Start}}


\subsubsection{Template Verification [Validation] of [what] }
\label{sec:}

\paragraph{Contributing project partners}
% Usually, one main partner, perhaps with contributions from others

\paragraph{Process step}
% Classification of the activity according to D2.3a
% Name what is verified and to which report this would contribute.
% Use the numbers, e.g. System Design Verification Report (1-12).

\paragraph{Object of verification}
% Which openETCS artifact (github link) or other documents/programs
% etc. (provide references)


\paragraph{Available specification}
% The specification against which the object is to be checked. Usually
% coming form some openETCS artifacts (GitHub reference, process
% artifact number) or background material (reference, artifact number).

\paragraph{Objective}
%In an ordinary development, the main objective would be to verify or validate
%something. Here, in openETCS, it would be quite common to demonstrate
%the applicability of some method/tool, or evaluate its capabilities.


\paragraph{Method/Approach}
% Short description of how the verification/validation is performed

\paragraph{Means/Tools}
% Could be integrated with the previous paragraph. Assign an
% approrpiate tool class (T1, T2, T3) according to EN 50128. Try to assign a
% maturity level 
%https://github.com/openETCS/validation/blob/master/VerificationAndValidationPlan/V02/VnVUsrStrTmplt-140709-02.pdf

\paragraph{Results}
% Results related to the objective.
% Refer to appropriate document (preferably GitHub) for more complete description.

\paragraph{Observations/Comments}
% An optional section where anything can be included which has been
% observed without direct connection


\paragraph{Conclusion}
%What has been achieved, what is missing, what has been learned

\cmmnt{\textbf{Template End}}

\subsection{Verification and Validation in the System Design Phase}
\label{sec:vnv-1}

% TWT analyzed sub-system requirements from \cite[Chapter~5]{subset-026:3.3.0}. The
% requirements have been modeled as colored Petri nets and subjected to
% formal analyses. This activity is part of the System Design
% Verification. 

\subsubsection{Verification of Chapter~5 of Subset~026 (TWT)}
\label{sec:verif-Subset-026}

\paragraph{Contributing project partners}
% Usually, one main partner, perhaps with contributions from others
The work has been performed by TWT.

\paragraph{Process step}
% Classification of the activity according to D2.3a

This activity is part of the verification of the Elaborated System
Requirements which are based on Subset~026 \cite{subset-026:3.3.0}. It
contributes to the System Design Verification Report (1-12). In
formalizing and analyzing the procedures it findings contribute also
to the definition of the Elaborated System Requirements themselves
(1-07).

\paragraph{Object of verification}
% Which openETCS artifact (github link) or other documents/programs
% etc. (provide references)

The object of verification are the procedures defined in Chapter~5 of
Subset~026 \cite[5]{subset-026:3.3.0}. \textit{NN} of the \qq{25}
procedures have been analyzed. 

\paragraph{Available specification}
% The specification against which the object is to be checked. Usually
% coming form some openETCS artifacts (GitHub reference) or background
% material (reference).
The procedures are checked for consistency. They are not checked
against an external specification.
%This example is specific!

\paragraph{Objective}
%In an ordinary development, the main objective would be to verify or validate
%something. Here, in openETCS, it would be quite common to demonstrate
%the applicability of some method/tool, or evaluate its capabilities.
The main objective w.r.t.\ verification is check the procedure definitions for
consistency and some sanity conditions. A by-product are
formalizations which can enter the Elaborated System Requirements (1-07). 


\paragraph{Method/Approach}
% Short description of how the verification/validation is performed
The control flow of the procedures is modeled with colored Petri nets
(CPNs) in the tool \cite{Westergaard2013apn}. Each model is checked
independently by a second person. The necessity of formalization
coming with the modeling uncovers inconsistencies in textual
specifications. With the help of the simulation and checking
facilities of the CPN tools, sanity conditions on the models are
checked.

\paragraph{Means/Tools}
% Could be integrated with the previous paragraph. Assign an
% approrpiate tool class (T1, T2, T3) according to EN 50128. Try to assign a
% maturity level 
%https://github.com/openETCS/validation/blob/master/VerificationAndValidationPlan/V02/VnVUsrStrTmplt-140709-02.pdf
The CPN tools are ...

\paragraph{Results}
% Results related to the objective.
% Refer to appropriate document (preferably GitHub) for more complete description.
The modeling and analysis uncovered 36 inconsistencies, ambiguities
and gaps in the Subset~026 which were reported
in~\cite{specfindings}. \cmmnt{to be revised/completed}

\paragraph{Observations/Comments}
% An optional section where anything can be included which has been
% observed without direct connection


\paragraph{Conclusion}
%What has been achieved, what is missing, what has been learned

\cmmnt{Missing procedures?}
The numerous specification findings illustrate the need for validating
the specification. CPNs are well-suited to model the behavioral
aspects described in Subset-026 chapter 5. The size of the model
clearly indicates the complexity of the procedures, even at the
current level of abstraction. The main benefit comes from the
activity of formalization itself, and of incomplete, but valuable,
simulations. 


\subsection{Verification and Validation in the Sub-System Architecture Design Phase}
\label{sec:vnv-2}

The DLR verified the Sub-System Architecture Design
\cmmnt{citations}. \cmmnt{\qq{correct phase}}

\subsection{Verification and Validation in the SW Specification Phase}
\label{sec:vnv-3}

\subsection{Verification and Validation in the SW Design Phase}
\label{sec:vnv-4}

Model-based testing applied to design models  \cmmnt{\qq{U Bremen}}

\subsection{Verification and Validation in the SW Component Phase}
\label{sec:vnv-5}

\begin{itemize}
\item Dedicated tests on single components \cmmnt{\qq{DB}}
\item Formal code verification (FRAMA C on the bitwalker)
\end{itemize}


\subsection{Verification and Validation in the SW Integration Phase}
\label{sec:vnv-6}

Automatized integration tests on the SW components. 

\subsection{Verification and Validation in the SW Validation Phase}
\label{sec:vnv-7}

There have been validations on
\begin{itemize}
\item the integrated software within the \qq{SCADE simulation
    environment}, subjecting the SW with a simulated environment to
  operational use cases.
\item an integration of the SW on a reference hardware, applying
  operational use cases.  
\end{itemize}

\section{Conclusion}
\label{sec:conclusion}


%\nocite{*}

\bibliographystyle{unsrt}

\bibliography{bibliography}



%===================================================
%Do NOT change anything below this line

\end{document}
