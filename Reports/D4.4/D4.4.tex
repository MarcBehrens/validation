\documentclass{template/openetcs_article}
% Use the option "nocc" if the document is not licensed under Creative Commons
%\documentclass[nocc]{template/openetcs_article}
\usepackage{lipsum,url}
\usepackage{supertabular}
\usepackage{multirow}
\usepackage{color, colortbl}
\definecolor{gray}{rgb}{0.8,0.8,0.8}
\usepackage[modulo]{lineno}
\graphicspath{{./template/}{.}{./images/}}
\begin{document}
\frontmatter
\project{openETCS}

%Please do not change anything above this line
%============================



% The document metadata is defined below

%user specified macros


\newcommand{\VV}{Verification \& Validation\xspace}
\newcommand{\vv}{verification \& validation\xspace}

\def\CC{{C\nolinebreak[4]\hspace{-.05em}\raisebox{.4ex}{\tiny\bf ++}}}

\newcommand{\adhesion}{\mbox{\inl{AdhesionFactor}}\xspace}

\newcommand{\bitwalker}{\mbox{\texttt{Bitwalker}}\xspace}

\newcommand{\poke}{\mbox{\texttt{Bitwalker\_Poke}}\xspace}
\newcommand{\peek}{\mbox{\texttt{Bitwalker\_Peek}}\xspace}
\newcommand{\acsl}{\mbox{\textsf{ACSL}}\xspace}
\newcommand{\isoc}{\mbox{\textsf{C}}\xspace}
\newcommand{\framac}{\mbox{\textsf{Frama-C}}\xspace}
\newcommand{\framacwp}{\mbox{\textsf{Frama-C\slash WP}}\xspace}
\newcommand{\why}{\mbox{\textsf{Why}}\xspace}
\newcommand{\wpframac}{\mbox{\textsf{WP}}\xspace}
\newcommand{\altergo}{\mbox{\textsf{Alt-Ergo}}\xspace}
\newcommand{\qed}{\mbox{\textsf{Qed}}\xspace}
\newcommand{\cvc}{\mbox{\textsf{CVC4}}\xspace}
\newcommand{\z}{\mbox{\textsf{Z3}}\xspace}
\newcommand{\coq}{\mbox{\textsf{Coq}}\xspace}
\newcommand{\cealist}{\mbox{\textsf{CEA LIST}}\xspace}
\newcommand{\fokus}{\mbox{\textsf{Fraunhofer FOKUS}}\xspace}

\newcommand{\inl}[1]{\lstinline[style=inline]{#1}}


%Background color of boxes in process graphic
\definecolor{light-gray}{gray}{0.95}

%assign a report number here
\reportnum{OETCS/WP4/D4.4V0.2}


%define your workpackage here
\wp{Work-Package 4: ``Verification \& Validation Strategy''}

%set a title here
\title{openETCS Final Report on Verification and Validation}

%set a subtitle here
\subtitle{}

%set the date of the report here
\date{December 2015}

%document approval
%define the name and affiliation of the people involved in the
%documents approbation here 
\creatorname{Marc Behrens}                                                                                                                                
\creatoraffil{ Deutsches Zentrum f\"ur Luft und Raumfahrt e.V.}                    
\techassessorname{[assessor name]}
\techassessoraffil{[affiliation]}

\qualityassessorname{Jan Welte}
\qualityassessoraffil{TU Braunschweig}

\approvalname{Klaus-R\"udiger Hase}
\approvalaffil{DB Netz}

%define a list of authors and their affiliation here


\author{Hardi Hungar}
\affiliation{DLR, main editing}
\author{Marc Behrens, Mirko Caspar, Michael M\"onsters}

\affiliation{DLR%\\
%  Lilienthalplatz 7\\
%  38108 Brunswick, Germany
%   \\eMail:hardi.hungar@dlr.de 
}

\author{Jan Peleska, Uwe Schulze}
\affiliation{University Bremen}

\author{Thorsten Schulz}
\affiliation{University Rostock}

\author{Stefan Rieger}
\affiliation{TWT}






% define the coverart
\coverart[width=350pt]{openETCS_EUPL}


%define the type of report
\reporttype{Final Report}

\begin{abstract}
%define an abstract here
  This document summarizes the approach, scope and result of the
  verification and validation activities in the project openETCS.
\end{abstract}

%=============================
%Do not change the next three lines
\maketitle

%Modification history
%if you do not need a modification history table for your document
%simply comment out the eight lines below 
%=============================
\section*{Modification History}
\tablefirsthead{
\hline 
\rowcolor{gray} 
Version & Section & Modification / Description & Author \\\hline}
\begin{supertabular}{| m{1.2cm} | m{1.2cm} | m{6.6cm} | m{4cm} |}
 0.0 & all & initial & Marc Behrens \\\hline
 0.1 & all & revision and addition & Hardi Hungar \\\hline
 0.2 & all & revision and addition & Hardi Hungar, contributions by partners\\\hline
\end{supertabular}

\tableofcontents
\listoffiguresandtables
\newpage
%=============================
%Uncomment the next line if you need line numbers for tracebility when the document is in review
%\linenumbers

%=============================
% The actual document starts below this line
%=============================

%Start here

\section{Introduction}

According to \cite[3.1.48]{EN50128:2011}, verification is an activity
to check whether the output of a development phase meets the
requirements. This concerns formalities, traceability, and, w.r.t.\
the main content, completeness, correctness and consistency. Within
openETCS, examples of each kind of verification have been
performed. Thereby, also new methods and tools have been evaluated and
adapted. 

Validation concerns the compliance of the end result of the
development with the user requirements. This has been done employing
the demonstrator of the EVC software. 

This document summarizes the activities described in more detail in
separate reports. It explains how these separate activities fit into
the development process of openETCS as defined in the deliverable
D2.3a.    

Most verification activities are actually reviews of documents (or
even programs). For general review activities, a process has been
defined in \cite{openETCS:D1.3.1}. 

\newpage







%Examples are below
\section{Verification and Validation in the Development Lifecycle}
\label{sec:Lifecycle}

\begin{figure}[hbt]
  \centering
  \def\svgwidth{.9\textwidth}
  {\tiny
  \input{Prcss2_3a-03.pdf_tex}}
  \caption{openETCS Development Lifecycle}
  \label{fig:lifecycle2}
\end{figure}

Fig.~\ref{fig:lifecycle2} is an overview of the openETCS development
lifecycle, taken from D2.3a. It depicts the process for a complete
development of the EVC software, of which a part has been performed
within the project. Verification, resp., validation, has to be done in
each of the phases of the development.

\section{Overview of Verification and Validation Activities}
\label{sec:overview}


The verification and validation activities of openETCS fall in two
categories. 
\begin{itemize}
\item They may serve the purpose of supporting the development
of the EVC SW. These are activities as defined in the process
definition D2.3a, actual verification or validation of design
artifacts.
\item They may serve to demonstrate or evaluate methods or tools for
  V\&V. Such methods or tools are applied either to available design
  artifacts, or some such artifacts are created specifically for the
  purpose of the demonstration/evaluation.
\end{itemize}
Both kinds of activities are reported about in this document. It is
structured according to the phases of the development. 

\addtocounter{subsection}{-1}
\subsection{Verification and Validation in the Planning Phase}
\label{sec:vnv-0}

There have been reviews of the planning documents. These activities
are not reported in detail, here.

% \cmmnt{\textbf{Template Start}}
% 

\subsubsection{Template Verification [Validation] of [what] }
\label{sec:}

\paragraph{Contributing project partners}
% Usually, one main partner, perhaps with contributions from others

\paragraph{Process step}
% Classification of the activity according to D2.3a
% Name what is verified and to which report this would contribute.
% Use the numbers, e.g. System Design Verification Report (1-12).

\paragraph{Object of verification}
% Which openETCS artifact (github link) or other documents/programs
% etc. (provide references)


\paragraph{Available specification}
% The specification against which the object is to be checked. Usually
% coming form some openETCS artifacts (GitHub reference, process
% artifact number) or background material (reference, artifact number).

\paragraph{Objective}
%In an ordinary development, the main objective would be to verify or validate
%something. Here, in openETCS, it would be quite common to demonstrate
%the applicability of some method/tool, or evaluate its capabilities.


\paragraph{Method/Approach}
% Short description of how the verification/validation is performed

\paragraph{Means/Tools}
% Could be integrated with the previous paragraph. Assign an
% approrpiate tool class (T1, T2, T3) according to EN 50128. Try to assign a
% maturity level 
%https://github.com/openETCS/validation/blob/master/VerificationAndValidationPlan/V02/VnVUsrStrTmplt-140709-02.pdf

\paragraph{Results}
% Results related to the objective.
% Refer to appropriate document (preferably GitHub) for more complete description.

\paragraph{Observations/Comments}
% An optional section where anything can be included which has been
% observed without direct connection


\paragraph{Conclusion}
%What has been achieved, what is missing, what has been learned

% \cmmnt{\textbf{Template End}}

\subsection{Verification and Validation in the System Design Phase}
\label{sec:vnv-1}

% TWT analyzed sub-system requirements from \cite[Chapter~5]{subset-026:3.3.0}. The
% requirements have been modeled as colored Petri nets and subjected to
% formal analyses. This activity is part of the System Design
% Verification. 

%\newpage
\subsubsection{Verification of Chapter~5 of Subset~026 (TWT)}
\label{sec:verif-Subset-026}

\paragraph{Contributing project partners}
% Usually, one main partner, perhaps with contributions from others
The work has been performed by TWT.

\paragraph{Process step}
% Classification of the activity according to D2.3a

This activity is part of the verification of the Elaborated System
Requirements which are based on Subset~026 \cite{subset-026:3.3.0}. It
contributes to the System Design Verification Report (1-12). In
formalizing and analyzing the procedures it findings contribute also
to the definition of the Elaborated System Requirements themselves
(1-07).

\paragraph{Object of verification}
% Which openETCS artifact (github link) or other documents/programs
% etc. (provide references)

The object of verification are the procedures defined in Chapter~5 of
Subset~026 \cite[5]{subset-026:3.3.0}. \textit{NN} of the \qq{25}
procedures have been analyzed. 

\paragraph{Available specification}
% The specification against which the object is to be checked. Usually
% coming form some openETCS artifacts (GitHub reference) or background
% material (reference).
The procedures are checked for consistency. They are not checked
against an external specification.
%This example is specific!

\paragraph{Objective}
%In an ordinary development, the main objective would be to verify or validate
%something. Here, in openETCS, it would be quite common to demonstrate
%the applicability of some method/tool, or evaluate its capabilities.
The main objective w.r.t.\ verification is check the procedure definitions for
consistency and some sanity conditions. A by-product are
formalizations which can enter the Elaborated System Requirements (1-07). 


\paragraph{Method/Approach}
% Short description of how the verification/validation is performed
The control flow of the procedures is modeled with colored Petri nets
(CPNs) in the tool \cite{Westergaard2013apn}. Each model is checked
independently by a second person. The necessity of formalization
coming with the modeling uncovers inconsistencies in textual
specifications. With the help of the simulation and checking
facilities of the CPN tools, sanity conditions on the models are
checked.

\paragraph{Means/Tools}
% Could be integrated with the previous paragraph. Assign an
% approrpiate tool class (T1, T2, T3) according to EN 50128. Try to assign a
% maturity level 
%https://github.com/openETCS/validation/blob/master/VerificationAndValidationPlan/V02/VnVUsrStrTmplt-140709-02.pdf
The CPN tools are ...

\paragraph{Results}
% Results related to the objective.
% Refer to appropriate document (preferably GitHub) for more complete description.
The modeling and analysis uncovered 36 inconsistencies, ambiguities
and gaps in the Subset~026 which were reported
in~\cite{specfindings}. \cmmnt{to be revised/completed}

\paragraph{Observations/Comments}
% An optional section where anything can be included which has been
% observed without direct connection


\paragraph{Conclusion}
%What has been achieved, what is missing, what has been learned

\cmmnt{Missing procedures?}
The numerous specification findings illustrate the need for validating
the specification. CPNs are well-suited to model the behavioral
aspects described in Subset-026 chapter 5. The size of the model
clearly indicates the complexity of the procedures, even at the
current level of abstraction. The main benefit comes from the
activity of formalization itself, and of incomplete, but valuable,
simulations. 


\addtocounter{subsection}{2}
% \subsection{Verification and Validation in the Sub-System Architecture Design Phase}
% \label{sec:vnv-2}

% No verification to be attributed to this phase has been performed in openETCS.


% \subsection{Verification and Validation in the SW Specification Phase}
% \label{sec:vnv-3}

% No verification to be attributed to this phase has been performed in openETCS.

\subsection{Verification and Validation in the SW Design Phase}
\label{sec:vnv-4}



\subsubsection{Verification of the openETCS Architecture and Design Specification}
\label{sec:}

\paragraph{Contributing project partners}
This work has been performed by the DLR.

\paragraph{Process step}
This activity is part of the verification of the openETCS SW
Architecture and Design Specification (4-19), ADD. It contributes to the SW
Design Verification Report (4-23).

\paragraph{Object of verification}
The object of verification is D3.5.3, the openETCS Architecture and
Design Specification.

\paragraph{Available specification}
The ADD is checked against
Subset~026 \cite{subset-026:3.3.0}.

\paragraph{Objective}
The objective is to check that the procedures of ETCS OBU are completely,
correctly and consistently mapped to the components of the SW as
described in the ADD document.

\paragraph{Method/Approach}
The verification has been performed by comparing the corresponding
specifications of Subset~026 with the ADD document for each relevant
paragraph.

\paragraph{Means/Tools}
The verification has been performed manually.

\paragraph{Results}
The verification uncovered some minor inconsistencies. These have been
reported to be removed in D3.5.4 which revises D3.5.3. 

\paragraph{Observations/Comments}
% An optional section where anything can be included which has been
% observed without direct connection

\paragraph{Conclusion}
%What has been achieved, what is missing, what has been learned

\newpage

\subsubsection{Model-based Test Generation for the ETCS Ceiling Speed Monitor}
\label{sec:csmunibremen}

\paragraph{Contributing project partners}
% Usually, one main partner, perhaps with contributions from others
Main contribution by University of Bremen, additional contributors: DLR and Siemens 

\paragraph{Process step}
% Classification of the activity according to D2.3a
% Name what is verified and to which report this would contribute.
% Use the numbers, e.g. System Design Verification Report (1-12).
This activity is part of the SW Design (Phase 4). It
contributes to the SW Component Test
Specification (4-22). 

\paragraph{Object of verification}
% Which openETCS artifact (github link) or other documents/programs
% etc. (provide references)
The object of verification are implementations of the ETCS
Ceiling Speed Monitor (CSM).  


\paragraph{Available specification}
% The specification against which the object is to be checked. Usually
% coming form some openETCS artifacts (GitHub reference, process
% artifact number) or background material (reference, artifact number).

The specification of speed and distance monitoring in \cite[Sec.~3.13]{subset-026:3.3.0}.

\paragraph{Objective}
%In an ordinary development, the main objective would be to verify or validate
%something. Here, in openETCS, it would be quite common to demonstrate
%the applicability of some method/tool, or evaluate its capabilities.

The main objective is to evaluate and demonstrate the new input
equivalence class partition test generation method developed by the
team of the University of Bremen. The method guarantees 100 per cent
error detection inside a fault domain, and is expected to provide high
coverage outside the domain. Its results on the CSM are compared with the
relevant system test cases as defined in then ETCS standard conformity
test specification, Subset~076.


\paragraph{Method/Approach}
% Short description of how the verification/validation is performed

A test model specifying the expected behaviour of the CSM has been
developed in SysML, using state machines and block diagrams.  The
model elements have been linked to the associated ETCS system
requirements.  Since this SysML language subset can be associated with
a formal semantics, it is possible to execute algorithms that
automatically generate sets of executable test cases from the
model. These sets of test cases permit to check implementations for
compliance with the model. The tracing information enable to derive
detailed coverage and fault identification information.

The existing SUBSET-076 test cases were formalised using linear
temporal logic (LTL), so that the same test data generation concept
could be applied as for the test cases that were automatically
identified: SUBSET-076 test cases do not provide concrete test data
for every test step, but specify the general constraints from which
concrete data can be elaborated.  This approach also allows to trace
the model coverage achieved by the SUBSET-076 test cases.

All tests were executed against software mutants derived from a
reference implementation, using 3 different mutation generators in
order to avoid a mutation bias. For each testing strategy applied it
was checked
\begin{itemize}
\item which parts of the test model were covered by the test execution, and
\item which fault coverage (percentage of ``killed'' mutants) was achieved.
\end{itemize}





\paragraph{Means/Tools}
% Could be integrated with the previous paragraph. Assign an
% approrpiate tool class (T1, T2, T3) according to EN 50128. Try to assign a
% maturity level 
%https://github.com/openETCS/validation/blob/master/VerificationAndValidationPlan/V02/VnVUsrStrTmplt-140709-02.pdf

The whole approach is fully supported by RT-Tester and its model-based
testing component RTT-MBT. Test cases are described by LTL
formulas. An integrated SMT-solver generate solutions for the LTL
formulas which add  concrete data and makes the test cases
executable. From the SysML test
model, the tool automatically derives LTL
formulas which describe the test cases. For the SUBSET-076 test cases,
the LTL formulas have been provided manually and completed by the
solver. 

\paragraph{Results}
% Results related to the objective.
% Refer to appropriate document (preferably GitHub) for more complete description.
The results can be summarised as follows.
\begin{enumerate}
\item The new equivalence class testing method shows significantly
  higher test strength than all other methods used in the
  comparison. It achieved nearly 100\% fault coverage for mutants
  outside the fault domain (mutants inside the fault domain are always
  killed, due to the guaranteed fault detection properties).

\item The new method is very well suited for software testing and
  HW/SW integration testing, where the high number of test cases
  (approx.~5000 cases) can easily be executed, in particular, because
  the test suite is fully automated. The new method, however, yields
  too many test cases to be applied on system testing level with real
  trains on real tracks.


\item The SUBSET-076 test cases are missing 2 cases for the CSM in order to achieve
requirements coverage. These can be easily identified and added. As a result,
these test comprise 11 cases.

\item With the missing test cases added, the SUBSET-076 achieve only a fault coverage of
62\% -- this would certainly not suffice to obtain certification credit. It is
possible, however, to add an acceptable number of test cases to the SUBSET-076
suite for the CSM which would significantly increase its test strength.


\end{enumerate}


All   results have been published in 
\begin{itemize}
\item Jan Peleska and Wen-ling Huang: Complete model-based equivalence
  class testing. Int J Softw Tools Technol Transfer. Published online:
  21 November 2014. DOI 10.1007/s10009-014-0356-8.

\item Felix H\"ubner, Wen-ling Huang, and Jan Peleska: Experimental
  Evaluation of a Novel Equivalence Class Partition Testing
  Strategy. In Jasmin Christian Blanchette and Nikolai Kosmatov
  (eds.): Tests and Proofs - 9th International Conference, TAP 2015,
  Held as Part of STAF 2015, L'Aquila, Italy, July 22-24,
  2015. Proceedings. Lecture Notes in Computer Science 9154, Springer,
  2015, pp. 155-172, doi 10.1007/978-3-319-21215-9\_10.

\item C{\'e}cile Braunstein, Anne E. Haxthausen, Wen-ling Huang, Felix
  H\"ubner, Jan Peleska, Uwe Schulze, and Linh Vu Hong: Complete
  Model-Based Equivalence Class Testing for the ETCS Ceiling Speed
  Monitor. In S. Merz and J. Pang (eds.): Proceedings of the ICFEM
  2014. Springer, LNCS 8829, pp. 380-395, 2014. DOI
  10.1007/978-3-319-11737-9\_25.


\item Technical Report http://www.informatik.uni-bremen.de/agbs/testingbenchmarks/
\newline
openETCS/ceiling-speed-monitoring/testing\_the\_etcs\_csm.pdf


\item C{\'e}cile Braunstein, Wen-ling Huang, Felix H\"ubner, Jan
  Peleska, and Uwe Schulze: Evaluation of Model-Based Testing
  Strategies for the ETCS Ceiling Speed Monitor.  Submitted to
  Software Testing, Verification and Reliability journal.

Also available as technical report
\end{itemize}

\paragraph{Observations/Comments}
% An optional section where anything can be included which has been
% observed without direct connection

It is interesting to note that typical model-coverage driven test
cases (e.g. transition coverage, MC/DC coverage), while achieving
higher model coverage than the SUBSET-076 tests, do not achieve much
higher fault coverage (approx.~68\%).  The reason is that these test
cases are not invariant under syntactic model transformations: with
another -- through semantically equivalent -- model, higher or lower
test strength would be achieved with the coverage-driven test cases
derived from that model.

In contrast to that, the new equivalence class testing strategy is
elaborated from the {\it semantic} representation of the model and is
therefore invariant (i.e.~always maximal) under all syntactic model
transformations that leave the behavioural semantics unchanged.

Verified Systems International GmbH who maintain the commercial
version of RT-Tester have won the runner-up trophy of the EU
Innovation Radar Innovation Prize\footnote{see {\tt
    https://www.verified.de/publications/papers-2015/\newline
    eu-innovation-radar-price-runner-up-trophy-for-verified-systems-international/}}
for implementing the equivalence class testing strategy described
above in the commercial version of RT-Tester.

\paragraph{Conclusion}
%What has been achieved, what is missing, what has been learned

The new test strategy has shown to provide superior test strength when
compared to SUBSET-076 test cases and conventional model-coverage
driven test cases that are typically provided by other model-based
testing tools. As of today, RT-Tester is the only testing tool where
the new test strategy is implemented.


\subsubsection{Model-based Testing of the ETCS Target Speed Monitor}
\label{sec:targetspeedmonitorbremen}

\paragraph{Contributing project partners}
% Usually, one main partner, perhaps with contributions from others
Main contribution by University of Bremen, additional contributors: DLR and Siemens 

\paragraph{Process step}
% Classification of the activity according to D2.3a
% Name what is verified and to which report this would contribute.
% Use the numbers, e.g. System Design Verification Report (1-12).
This activity is part of the SW Design (Phase 4). It
contributes to the SW Component Test
Specification (4-22). 

\paragraph{Object of verification}
% Which openETCS artifact (github link) or other documents/programs
% etc. (provide references)

The object of verification is an implementation of the target speed monitoring function of the EVC,
see technical report
\begin{itemize}
\item[[~1]] Felix H{\"u}bner, Christoph Hilken, and Jan Peleska.
Combination of Behavioral and Parametric Diagrams for Model-based
Testing -- Application to ETCS Target Speed Monitoring. Submitted to DAC 2016, also available as 
technical openECTS report 2014-11-25.

Available under \url{https://github.com/openETCS/validation/tree/master/VnVUserStories/VnVUserStoryUniBremen/04-Results}
\end{itemize}


\paragraph{Available specification}
% The specification against which the object is to be checked. Usually
% coming form some openETCS artifacts (GitHub reference, process
% artifact number) or background material (reference, artifact number).
ETCS system specification, SUBSET-026-3;
model parts are also available in [1]. The whole target speed monitoring model will be made available on {\tt http://www.mbt-benchmarks.org}.

\paragraph{Objective}
%In an ordinary development, the main objective would be to verify or validate
%something. Here, in openETCS, it would be quite common to demonstrate
%the applicability of some method/tool, or evaluate its capabilities.
For creating a SysML test model of the target speed monitoring function, both time-discrete
(e.g.~trigger of the emergency brakes) and time-continuous (e.g.~time-dependent 
train location, speed, and acceleration) variables 
need to be considered. SysML state machines
are suitable for modelling concurrent real-time behaviour of time-discrete control
functions. For time-continuous aspects, the report [1] describes how to use
parametric constraints and associated diagrams for modelling. It is also explained
how the parametric specifications are made available to the SMT solver creating 
concrete test data from models. As a result, the solver generates data that 
complies with the time-continuous physical constraints of the model. 


\paragraph{Method/Approach}
% Short description of how the verification/validation is performed

Parametric constraints represent a language aspect of the SysML which has not yet
been fully investigated in the research communities. Using so-called constraint
blocks, these constraints can be specified. Typically, parametric constraints 
represent system invariants or -- this is the relevant aspect for the target speed monitor --
physical laws, such as acceleration-dependent speed and speed-dependent location. 
For our application, these laws also comprise the ETCS braking curves modelling the
speed changes of the braking train.
Parametric constraints can be specified using general physical variables; these are 
bound to concrete model variables using parametric diagrams. 

It is shown in [1] how parametric constraints can be used to calculate physically
meaningful train behaviours, that is, meaningful changes of speed and location over
time, taking into account the braking actions. The method follows a 2-step approach: 
first, a model abstraction is created, and  the equivalence class testing 
strategy described in Section~\ref{sec:csmunibremen} is used to identify
test cases with guaranteed fault detection properties. Next, the calculated tests
are refined with respect to time-dependent behaviour, so that still the same 
equivalence classes  are used, but the representatives for location and speed
are selected in a way that complies with the physical laws.

 


\paragraph{Means/Tools}
% Could be integrated with the previous paragraph. Assign an
% approrpiate tool class (T1, T2, T3) according to EN 50128. Try to assign a
% maturity level 
%https://github.com/openETCS/validation/blob/master/VerificationAndValidationPlan/V02/VnVUsrStrTmplt-140709-02.pdf

The method has been implemented in the RT-Tester tool as part of the WP7-related
activities of the University of Bremen team.

\paragraph{Results}
% Results related to the objective.
% Refer to appropriate document (preferably GitHub) for more complete description.

The results show that the method can be automatically performed with acceptable 
computation time. 

\paragraph{Observations/Comments}
% An optional section where anything can be included which has been
% observed without direct connection

To our best knowledge, this is the first SysML-based method
for calculating test data with guaranteed fault detection properties in presence of
both time-discrete and time-continuous observables.


\paragraph{Conclusion}
%What has been achieved, what is missing, what has been learned

The method developed here is highly relevant for testing cyber-physical systems
in general. Verified Systems International GmbH who maintain the commercial version
of RT-Tester has already decided to make this method available in 2016.


\newpage
\subsection{Verification and Validation in the SW Component Phase}
\label{sec:vnv-5}




\subsubsection{Basic Component Verification of Components  Implemented in
  SCADE }
\label{sec:VnV-Code-Basic}


Each component which has been implemented with the SCADE Suite
Advanced Modeler has been subjected to basic verification by the
implementer. The objective of the basic verification is to establish
that the component implements the functionality as required in
standard (non-exceptional) usage situations. This is part of the
process Phase~5, SW Component Implementation and Test. A component
which has passed the basic test may be integrated with other
components (Phase~6, SW Integration). 

The basic test is only a part of the full test according to the SW
Component Test Specification, which requires, among other, code
coverage criteria to be met. That test must be performed on the final
version of the component, and it is to be performed by the Tester.

As an example of the basic component
verifications performed, the one of the component implementing the
Speed and Distance Monitoring from \cite[3.13]{subset-026:3.3.0} is
documented in the following.
\newpage


\subsubsection{Basic Verification of the Implementation of ``Speed and
  Distance Monitoring'' }
\label{sec:}

\paragraph{Contributing project partners}
% Usually, one main partner, perhaps with contributions from others
This work has been performed by the University of Rostock.

\paragraph{Process step}
% Classification of the activity according to D2.3a
% Name what is verified and to which report this would contribute.
% Use the numbers, e.g. System Design Verification Report (1-12).

This activity is part of Phase~5, SW Component Implementation and Test. It
addresses the requirement 5.1 (basic tests performed by the
Implementer). 


\paragraph{Object of verification}
% Which openETCS artifact (github link) or other documents/programs
% etc. (provide references)

The object of verification is the SCADE package
\texttt{SpeedSupervision\_Integration} incorporating the sub packages
\texttt{CalcBrakingCurves}, \texttt{SDM\_Commands},
\texttt{SDM\_GradientAcceleration}, \texttt{SDM\_Models},
\texttt{SDM\_TargetLimits} and \texttt{TargetManagement}, together
with the helper and type package \texttt{SDM\_Types}.

\paragraph{Available specification}
% The specification against which the object is to be checked. Usually
% coming form some openETCS artifacts (GitHub reference, process
% artifact number) or background material (reference, artifact number).

The specification of speed and distance monitoring is given in
\cite[Sec.~3.13]{subset-026:3.3.0}. 
%Subset-026 (v3.3) Figure 28. 

\paragraph{Objective}
%In an ordinary development, the main objective would be to verify or validate
%something. Here, in openETCS, it would be quite common to demonstrate
%the applicability of some method/tool, or evaluate its capabilities.

The objective is to establish that the code performs the expected
functionality in standard (expected) usage scenarios. It is not
meant to be exhaustive. Also, it shall check for performance and
memory usage abnormalities. 

\paragraph{Method/Approach}
% Short description of how the verification/validation is performed

The test has been performed by integrating the package
\texttt{SpeedSupervision\_UnitTest\_Pkg} into the main module which
provides the implementation code with inputs of a usage scenario. The
test simulates a linear movement via generated odometry inputs,
providing constant default national values, reasonable train data and
a generic track that consists of one constant speed profile, a single,
non-extending movement authority and a null-gradient. The correctness
of the implementation reaction is checked mainly manually. 

\paragraph{Means/Tools}
% Could be integrated with the previous paragraph. Assign an
% appropriate tool class (T1, T2, T3) according to EN 50128. Try to assign a
% maturity level 
%https://github.com/openETCS/validation/blob/master/VerificationAndValidationPlan/V02/VnVUsrStrTmplt-140709-02.pdf

The test has been performed with the simulation functions (SCADE Suite
Simulator) of the SCADE Suite Advanced Modeler, Version 6.1. Since the
basic verification is to complemented by a full verification, the tool
qualification level is (only) T1.  

\paragraph{Results}
% Results related to the objective.
% Refer to appropriate document (preferably GitHub) for more complete description.

The test has been applied to each version of the
implementation. Only implementation versions which passed the test had
been given clearance for integration.


% \paragraph{Observations/Comments}
% % An optional section where anything can be included which has been
% % observed without direct connection



% The test case is relatively static, simulating a linear movement via
% generated odometry inputs, providing constant default national values,
% reasonable train data and a generic track that consists of one
% constant speed profile, a single, non-extending movement authority and
% a null-gradient. The test bench does not compare against expected
% values and heavily relies on human inspection. Without a multitude of
% speed profiles (MRSP) this test case cannot cover multiple EBD
% curves. This is up to larger test setups that also provide better data
% visualizations.

\paragraph{Conclusion}
%What has been achieved, what is missing, what has been learned

The test established readiness for integration testing. It does not
cover the full
verification of the code according to the Component Test Specification
(4-22). 
\newpage

\subsubsection{Code Reviews}
\label{sec:code-reviews}

The SW components have been subjected to code reviews to detect design
errors prior to testing, and to complement testing. An example of the
code reviews is presented in the ensuing section. 


\subsubsection{Code Review of the SCADE Package trainData}
\label{sec:VnV-SCADETrainData-DLR}

\paragraph{Contributing project partners}
This work has been performed by the DLR.

\paragraph{Process step}
This activity is part of the verification of the SW components. It
contributes to the SW Component Verification Report (5-26). 

\paragraph{Object of verification}
The SCADE package \texttt{trainData} incorporating the sub packages \texttt{trainData\_
pkg} and \texttt{trainData\_Types\_pkg}.

\paragraph{Available specification}
The package was checked against the ADD document (D3.5.4) and the ETCS
specification Subset-026, \cite[Sec.~3.18.3]{subset-026:3.3.0}.

\paragraph{Objective}
The objective is to establish plausibility that the SCADE package
conforms to the specification.

\paragraph{Method/Approach}
The verification has been performed by comparing implementation given
by the SCADE model with the corresponding specifications of Subset~026
and the ADD document.

\paragraph{Means/Tools}
The verification has been performed using the editing capabilities of
the SCADE Suite to display the model, and to search and navigate.

\paragraph{Results}
The review uncovered some incomplete coverage of the requirements due
to the absence of some implementation packages. The ramifications need
to be analyzed. 

% \paragraph{Observations/Comments}
% % An optional section where anything can be included which has been
% % observed without direct connection

% \paragraph{Conclusion}
% %What has been achieved, what is missing, what has been learned

% \newpage
% \input{VnV-Code-Fraunhofer}

\newpage
\subsection{Verification and Validation in the SW Integration Phase}
\label{sec:vnv-6}

\bgcmmnt{There have been automated integration tests on the SW components.} 

\subsection{Verification and Validation in the SW Validation Phase}
\label{sec:vnv-7}

There have been validations on
\begin{itemize}
\item the integrated software within the \qq{SCADE simulation
    environment}, subjecting the SW with a simulated environment to
  operational use cases.
\item an integration of the SW on a reference hardware, applying
  operational use cases.  
\end{itemize}



\subsubsection{Validation of the Implemented and Integrated Demonstration System}
\label{sec:VnV-Validation-DLR}


\paragraph{Contributing project partners}
The implementation and the validation of the integrated demonstration
system has been performed by the DLR with support of Fraunhofer FOCUS and
GE.


\paragraph{Process step}
This activity is part of the SW Validation (Phase~7). It contributes
to the Overall SW Test Report (7-29).


\paragraph{Object of validation}
The integrated demonstration system is validated. It consists of 2
basic subsystems: the EVC and the DMI. The implementations of both
subsystems are generated from the according operators of the SCADE
model using the Esterel code generator KCG. The complete integrated
demonstration system includes the target platforms and communication
systems as well. The triggering of the EVC as well as of the DMI needs
to be realised by a platform dependent wrapper. This wrapper has also
to handle the communication channels and resources. The wrapper for
EVC and DMI depend completely on the chosen target platform and are
implemented manually. However, the basic structure and basic schemes
are identical since both wrappers have the same tasks.


\paragraph{Available specification}
The modelled SCADE simulation of the EVC and DMI running from
Amsterdam to Utrecht is used as behavioural reference
specification. All driver relevant outputs - such as speeds,
distances, and state changes - are relevant for the validation.


\paragraph{Objective}
Basic assumption for model driven code generation is the correctness
of the code generator. This assumption is also stated for the
KCG-generated model code. Hence, the functionality of the implemented
EVC and the implemented DMI is not verified, since the verification is
already done on model level.

Furthermore, the interaction of each subsystem on a physical hardware
platform needs to be validated for correct functional
behaviour. Especially platform related resource restrictions or timing
issues may influence the overall behaviour of the generated
implementations of the subsystems.


\paragraph{Method/Approach}
The validation is realised by running the test track (Amsterdam -
Utrecht). A concrete sequence of activities is defined in order to
start-up and initiate the distributed system. External inputs,
e.g. for TIU, odometry, and balise information, are provided by a
simulation platform (SEFEV, proprietary software for executing
Subset076 sequences) which is connected via TCP-sockets to the
EVC. The behaviour of the integrated demonstration system is compared
to the behaviour of the SCADE-simulation model. Tolerances for time
and distances have been used as specified in Subset076.

The log files of each subsystem were used in order to check concrete behaviour.


\paragraph{Results}
The validation was done based on Win32-implementations of each
subsystem (EVC, DMI). Both subsystems were executed as single
processes on the same machine. The communication was realised via
TCP-sockets. The correct behaviour of the implementation compared to
the simulated model was shown for a first part of the test drive of
around 4km.


\paragraph{Observations/Comments}
A second implementation of the demonstrations system was realised on
an embedded realtime platform. The EVC was executed on this platform
whereas the DMI needed to be executed on a Win32-platform. The
generated code for EVC and DMI were identical to the code of the
initial Win32-implementation. Only the platform dependent wrappers and
the communication management needed to be adopted.

Due to timing and resource restrictions of the real-time platform,
several synchronisation issues needed to be solved. It can be stated
that the execution times of each part of the subsystem may influence
the overall functional behavior.


\paragraph{Conclusion}
The generated implementation of the SCADE model and the basic wrapping
systems work as expected. Further investigations are necessary in
order to validate runtime and synchronisation effects - mainly on
heterogeneous target platforms.


\section{Conclusion}
\label{sec:conclusion}

\bgcmmnt{The conclusion will be written after the completion of the
  V\&V activities.}

%\nocite{*}

\bibliographystyle{unsrt}

\bibliography{bibliography}


%===================================================
%Do NOT change anything below this line

\end{document}
