\documentclass{template/openetcs_article}
% Use the option "nocc" if the document is not licensed under Creative Commons
%\documentclass[nocc]{template/openetcs_article}
\usepackage{lipsum,url}
\usepackage{supertabular}
\usepackage{multirow}
\usepackage{color, colortbl}
\definecolor{gray}{rgb}{0.8,0.8,0.8}
\usepackage[modulo]{lineno}
\graphicspath{{./template/}{.}{./images/}}
\begin{document}
\frontmatter
\project{openETCS}

%Please do not change anything above this line
%============================



% The document metadata is defined below

%user specified macros


\newcommand{\VV}{Verification \& Validation\xspace}
\newcommand{\vv}{verification \& validation\xspace}

\def\CC{{C\nolinebreak[4]\hspace{-.05em}\raisebox{.4ex}{\tiny\bf ++}}}

\newcommand{\adhesion}{\mbox{\inl{AdhesionFactor}}\xspace}

\newcommand{\bitwalker}{\mbox{\texttt{Bitwalker}}\xspace}

\newcommand{\poke}{\mbox{\texttt{Bitwalker\_Poke}}\xspace}
\newcommand{\peek}{\mbox{\texttt{Bitwalker\_Peek}}\xspace}
\newcommand{\acsl}{\mbox{\textsf{ACSL}}\xspace}
\newcommand{\isoc}{\mbox{\textsf{C}}\xspace}
\newcommand{\framac}{\mbox{\textsf{Frama-C}}\xspace}
\newcommand{\framacwp}{\mbox{\textsf{Frama-C\slash WP}}\xspace}
\newcommand{\why}{\mbox{\textsf{Why}}\xspace}
\newcommand{\wpframac}{\mbox{\textsf{WP}}\xspace}
\newcommand{\altergo}{\mbox{\textsf{Alt-Ergo}}\xspace}
\newcommand{\qed}{\mbox{\textsf{Qed}}\xspace}
\newcommand{\cvc}{\mbox{\textsf{CVC4}}\xspace}
\newcommand{\z}{\mbox{\textsf{Z3}}\xspace}
\newcommand{\coq}{\mbox{\textsf{Coq}}\xspace}
\newcommand{\cealist}{\mbox{\textsf{CEA LIST}}\xspace}
\newcommand{\fokus}{\mbox{\textsf{Fraunhofer FOKUS}}\xspace}

\newcommand{\inl}[1]{\lstinline[style=inline]{#1}}


%Background color of boxes in process graphic
\definecolor{light-gray}{gray}{0.95}

%assign a report number here
\reportnum{OETCS/WP4/D4.4V0.1}


%define your workpackage here
\wp{Work-Package 4: ``Verification \& Validation Strategy''}

%set a title here
\title{openETCS Final Report on Verification and Validation}

%set a subtitle here
\subtitle{}

%set the date of the report here
\date{December 2015}

%document approval
%define the name and affiliation of the people involved in the
%documents approbation here 
\creatorname{Marc Behrens}                                                                                                                                
\creatoraffil{ Deutsches Zentrum für Luft und Raumfahrt e.V.}                    
\techassessorname{[assessor name]}
\techassessoraffil{[affiliation]}

\qualityassessorname{Jan Welte}
\qualityassessoraffil{TU Braunschweig}

\approvalname{Klaus-R\"udiger Hase}
\approvalaffil{DB Netz}

%define a list of authors and their affiliation here


\author{Marc Behrens \and Hardi Hungar}


\affiliation{DLR\\
  Lilienthalplatz 7\\
  38108 Brunswick, Germany
%   \\eMail:hardi.hungar@dlr.de 
}



% define the coverart
\coverart[width=350pt]{openETCS_EUPL}

%define the type of report
\reporttype{Final Report}

\begin{abstract}
%define an abstract here
  This document summarizes the approach, scope and result of the
  verification and validation activities in the project openETCS.
\end{abstract}

%=============================
%Do not change the next three lines
\maketitle

%Modification history
%if you do not need a modification history table for your document
%simply comment out the eight lines below 
%=============================
\section*{Modification History}
\tablefirsthead{
\hline 
\rowcolor{gray} 
Version & Section & Modification / Description & Author \\\hline}
\begin{supertabular}{| m{1.2cm} | m{1.2cm} | m{6.6cm} | m{4cm} |}
 0.0 & all & initial & Marc Behrens \\\hline
 0.1 & all & revision and addition & Hardi Hungar \\\hline
\end{supertabular}

\tableofcontents
\listoffiguresandtables
\newpage
%=============================
%Uncomment the next line if you need line numbers for tracebility when the document is in review
%\linenumbers

%=============================
% The actual document starts below this line
%=============================

%Start here

\section{Introduction}

According to \cite[3.1.48]{EN50128:2011}, verification is an activity
to check whether the output of a development phase meets the
requirements. This concerns formalities, traceability, and, w.r.t.\
the main content, completeness, correctness and consistency. Within
openETCS, examples of each kind of verification have been
performed. Thereby, also new methods and tools have been evaluated and
adapted. 

Validation concerns the compliance of the end result of the
development with the user requirements. This has been done employing
the demonstrator of the EVC software. 

This document summarizes the activities described in more detail in
separate reports. It explains how these separate activities fit into
the development process of openETCS as defined in the deliverable
D2.3a.    

Most verification activities are actually reviews of documents (or
even programs). For general review activities, a process has been
defined in \cite{openETCS:D1.3.1}. 

\newpage







%Examples are below
\section{Verification and Validation in the Development Lifecycle}
\label{sec:Lifecycle}

\begin{figure}[hbt]
  \centering
  \def\svgwidth{.9\textwidth}
  {\tiny
  \input{Prcss2_3a-03.pdf_tex}}
  \caption{openETCS Development Lifecycle}
  \label{fig:lifecycle2}
\end{figure}

Fig.~\ref{fig:lifecycle2} is an overview of the openETCS development
lifecycle, taken from D2.3a. It depicts the process for a complete
development of the EVC software, of which a part has been performed
within the project. Verification, resp., validation, has to be done in
each of the phases of the development.

\section{Overview of Verification and Validation Activities}
\label{sec:overview}

\bgcmmnt{Some sample notes are included in the subsections. To be
  checked for correct assignment to the phases, extended to become
  self-contained summaries of the activities with results and
  contributions. Note: Also evaluating a new verification method is a
  contribution to be mentioned, if this is a side or main effect of
  the activity. Do not forget to add yourself as an author if you contribute.}

\subsection{Verification and Validation in the Planning Phase}
\label{sec:vnv-0}

There have been reviews of the planning documents \cmmnt{compile a
  list}. 

\cmmnt{\textbf{Template Start}}


\subsubsection{Template Verification [Validation] of [what] }
\label{sec:}

\paragraph{Contributing project partners}
% Usually, one main partner, perhaps with contributions from others

\paragraph{Process step}
% Classification of the activity according to D2.3a
% Name what is verified and to which report this would contribute.
% Use the numbers, e.g. System Design Verification Report (1-12).

\paragraph{Object of verification}
% Which openETCS artifact (github link) or other documents/programs
% etc. (provide references)


\paragraph{Available specification}
% The specification against which the object is to be checked. Usually
% coming form some openETCS artifacts (GitHub reference, process
% artifact number) or background material (reference, artifact number).

\paragraph{Objective}
%In an ordinary development, the main objective would be to verify or validate
%something. Here, in openETCS, it would be quite common to demonstrate
%the applicability of some method/tool, or evaluate its capabilities.


\paragraph{Method/Approach}
% Short description of how the verification/validation is performed

\paragraph{Means/Tools}
% Could be integrated with the previous paragraph. Assign an
% approrpiate tool class (T1, T2, T3) according to EN 50128. Try to assign a
% maturity level 
%https://github.com/openETCS/validation/blob/master/VerificationAndValidationPlan/V02/VnVUsrStrTmplt-140709-02.pdf

\paragraph{Results}
% Results related to the objective.
% Refer to appropriate document (preferably GitHub) for more complete description.

\paragraph{Observations/Comments}
% An optional section where anything can be included which has been
% observed without direct connection


\paragraph{Conclusion}
%What has been achieved, what is missing, what has been learned

\cmmnt{\textbf{Template End}}

\subsection{Verification and Validation in the System Design Phase}
\label{sec:vnv-1}

% TWT analyzed sub-system requirements from \cite[Chapter~5]{subset-026:3.3.0}. The
% requirements have been modeled as colored Petri nets and subjected to
% formal analyses. This activity is part of the System Design
% Verification. 

\subsubsection{Verification of Chapter~5 of Subset~026 (TWT)}
\label{sec:verif-Subset-026}

\paragraph{Contributing project partners}
% Usually, one main partner, perhaps with contributions from others
The work has been performed by TWT.

\paragraph{Process step}
% Classification of the activity according to D2.3a

This activity is part of the verification of the Elaborated System
Requirements which are based on Subset~026 \cite{subset-026:3.3.0}. It
contributes to the System Design Verification Report (1-12). In
formalizing and analyzing the procedures it findings contribute also
to the definition of the Elaborated System Requirements themselves
(1-07).

\paragraph{Object of verification}
% Which openETCS artifact (github link) or other documents/programs
% etc. (provide references)

The object of verification are the procedures defined in Chapter~5 of
Subset~026 \cite[5]{subset-026:3.3.0}. \textit{NN} of the \qq{25}
procedures have been analyzed. 

\paragraph{Available specification}
% The specification against which the object is to be checked. Usually
% coming form some openETCS artifacts (GitHub reference) or background
% material (reference).
The procedures are checked for consistency. They are not checked
against an external specification.
%This example is specific!

\paragraph{Objective}
%In an ordinary development, the main objective would be to verify or validate
%something. Here, in openETCS, it would be quite common to demonstrate
%the applicability of some method/tool, or evaluate its capabilities.
The main objective w.r.t.\ verification is check the procedure definitions for
consistency and some sanity conditions. A by-product are
formalizations which can enter the Elaborated System Requirements (1-07). 


\paragraph{Method/Approach}
% Short description of how the verification/validation is performed
The control flow of the procedures is modeled with colored Petri nets
(CPNs) in the tool \cite{Westergaard2013apn}. Each model is checked
independently by a second person. The necessity of formalization
coming with the modeling uncovers inconsistencies in textual
specifications. With the help of the simulation and checking
facilities of the CPN tools, sanity conditions on the models are
checked.

\paragraph{Means/Tools}
% Could be integrated with the previous paragraph. Assign an
% approrpiate tool class (T1, T2, T3) according to EN 50128. Try to assign a
% maturity level 
%https://github.com/openETCS/validation/blob/master/VerificationAndValidationPlan/V02/VnVUsrStrTmplt-140709-02.pdf
The CPN tools are ...

\paragraph{Results}
% Results related to the objective.
% Refer to appropriate document (preferably GitHub) for more complete description.
The modeling and analysis uncovered 36 inconsistencies, ambiguities
and gaps in the Subset~026 which were reported
in~\cite{specfindings}. \cmmnt{to be revised/completed}

\paragraph{Observations/Comments}
% An optional section where anything can be included which has been
% observed without direct connection


\paragraph{Conclusion}
%What has been achieved, what is missing, what has been learned

\cmmnt{Missing procedures?}
The numerous specification findings illustrate the need for validating
the specification. CPNs are well-suited to model the behavioral
aspects described in Subset-026 chapter 5. The size of the model
clearly indicates the complexity of the procedures, even at the
current level of abstraction. The main benefit comes from the
activity of formalization itself, and of incomplete, but valuable,
simulations. 




\subsubsection{Verification of Management of Radio Communication function (Systerel) }
\label{sec:}

\paragraph{Contributing project partners}
% Usually, one main partner, perhaps with contributions from others
The work has been performed by Systerel

\paragraph{Process step}
% Classification of the activity according to D2.3a
% Name what is verified and to which report this would contribute.
% Use the numbers, e.g. System Design Verification Report (1-12).


This activity contributes to:
\begin{itemize}
\item the Elaborated System Requirements (1.07)
\item the Sub-System Requirement Specification (1.10)
\item the System Design Verification Report  (1.12)  
\end{itemize}



\paragraph{Object of verification}
% Which openETCS artifact (github link) or other documents/programs
% etc. (provide references)

The object of verification is the Event-B model for the communication
establishing at {\url{https://github.com/openETCS/model-evaluation/tree/master/model/Event_B_Systerel/Subset_026_comm_session}}. It
is from the strictly formal modeling phase and represents the communication
session management of the OBU.


\paragraph{Available specification}
% The specification against which the object is to be checked. Usually
% coming form some openETCS artifacts (GitHub reference, process
% artifact number) or background material (reference, artifact number).

The model implements the requirements for the communication session management
as described in Subset-026 chapter 3.5.

This section describes the establishing, maintaining and termination of a
communication session of the OBU with on-track systems.


\paragraph{Objective}
%In an ordinary development, the main objective would be to verify or validate
%something. Here, in openETCS, it would be quite common to demonstrate
%the applicability of some method/tool, or evaluate its capabilities.


One goal is the development of a strictly formal, fully proven model of the
communication session management and to provide evidence of covering the
necessary requirements of Subset-026 as well as proving correctness of the model
wrt.\ the requirements and attaining a good coverage of the model wrt.\ the
requirements.

The second goal is to correctly implement the applicable safety requirements
identified by the safety analysis. Both functional and safety requirements
should be traced in the model and a requirement document in a standardized
format.

The formal model will represent the described functionality on the system level,
the correct functioning can be validated by step-wise simulation and
model-checking of deadlock-freeness.


\paragraph{Method/Approach}
% Short description of how the verification/validation is performed

At first, the basic functionality described in the chapter 3.5 that are
identified. These serve as basis for a first abstract model, which is refined
iteratively, adding the desired level of detail. The elements of Subset-026 are
traced using links from Event-B to the ProR file in ReqIf format. Requirements
are formalized as invariants and proven where applicable.

\paragraph{Means/Tools}
% Could be integrated with the previous paragraph. Assign an
% approrpiate tool class (T1, T2, T3) according to EN 50128. Try to assign a
% maturity level 
%https://github.com/openETCS/validation/blob/master/VerificationAndValidationPlan/V02/VnVUsrStrTmplt-140709-02.pdf


The means used are:
\begin{itemize}
\item open source Rodin tool (\url{http://www.event-b.org/}), including plug-ins
  (for details
  see~\url{https://github.com/openETCS/model-evaluation/blob/master/model/Event_B_Systerel/Subset_026_comm_session/latex/subset_3_5.pdf})
\item ProR requirements modeling tool~\url{http://www.pror.org}
\item open source ProB model checker and B model
  simulator~\url{http://www.stups.uni-duesseldorf.de/ProB/index.php5/Main_Page}
\item open source CVC3 (\url{http://www.cs.nyu.edu/acsys/cvc3/}), verIT
  (\url{www.verit-solver.org}) and Alt-Ergo (\url{http://alt-ergo.lri.fr}) SMT
  solvers
\end{itemize}

All the tools are on class T1.

\paragraph{Results}
% Results related to the objective.
% Refer to appropriate document (preferably GitHub) for more complete description.


\begin{itemize}
\item The result is a fully formal model of the communication session management
  as described in chapter 3.5 of Subset-026.
\item Each implemented element of this section is linked to the ProR
  requirements file, both specification elements that describe how something has
  to be done, as well as requirements that describe what must be achieved.
\item The model can be simulated / animated, either with the AnimB or the ProB
  plug-in, validating the functional capabilities.
\item The safety requirements are formalized as invariants in predicate logic,
  their proofs are for the most part fully automatic.
\item It was found that while the Subset-026 communication management explicitly
  allows multiple communication partners (see RBC handover), there is no
  explicit limit of established communication connections given in chapter 3.5.
\item A complete covering of the elements of Subset-026 was not realized, e.g.,\
  there is a representation of the contents of a message, but its explicit
  format is not implemented. This is considered an implementation detail without
  influence for a system level analysis. In general, Event-B models will not be
  refined up to the implementation level.
\end{itemize}

See {\url{https://github.com/openETCS/validation/blob/master/Reports/D4.3/D4.3.1-Final-VV-report-on-model/D4.3.1.pdf}} and {\url{https://github.com/openETCS/validation/blob/master/VnVUserStories/VnVUserStorySysterel/04-Results/d-EventB-VnV/EventB-Rodin-VnV.pdf}}.



\paragraph{Observations/Comments}
% An optional section where anything can be included which has been
% observed without direct connection


\paragraph{Conclusion}
%What has been achieved, what is missing, what has been learned

Having an abstract formal model of the implemented functionality which can be
simulated, allows for interesting insights into the overall functioning of a
system. Formalized requirements are very helpful in both the identification of
ambiguous requirements and in their clarification.

The elements of Subset-026 are of very different nature. Some describe rather
low-level specification details, other describe ``real'' requirements. Without
an analysis as done with this Event-B model, it can be difficult to decide which
elements must be considered on a system level analysis and which on the lower
implementation level.



\subsection{Verification and Validation in the Sub-System Architecture Design Phase}
\label{sec:vnv-2}

The DLR verified the Sub-System Architecture Design
\cmmnt{citations}. \cmmnt{\qq{correct phase}}

\subsection{Verification and Validation in the SW Specification Phase}
\label{sec:vnv-3}



\subsubsection{Verification of Procedure On-Sight (Systerel) }
\label{sec:}

\paragraph{Contributing project partners}
% Usually, one main partner, perhaps with contributions from others
The work has been performed by Systerel

\paragraph{Process step}
% Classification of the activity according to D2.3a
% Name what is verified and to which report this would contribute.
% Use the numbers, e.g. System Design Verification Report (1-12).


This activity contributes to:
\begin{itemize}
\item the Software Requirement Specification  (3.16)
\item the Software Specification Verification Report  (3.18)  
\item the Software Architecture and Design Specification (4.19)
\item the Software Design Verification Report  (4.23)  
\item the Software Components (5.24)
\item the Software Component Verification Report  (5.26)  
\end{itemize}



\paragraph{Object of verification}
% Which openETCS artifact (github link) or other documents/programs
% etc. (provide references)

The object of verification is the Classical B model for the procedure On-Sight at {\url{https://github.com/openETCS/model-evaluation/tree/master/model/Classical_B_Systerel/obu_classicalB}}. 


\paragraph{Available specification}
% The specification against which the object is to be checked. Usually
% coming form some openETCS artifacts (GitHub reference, process
% artifact number) or background material (reference, artifact number).

The model implements the requirements for the The Procedure On-Sight, as described in {\itshape System Requirements Specification, Chapter 5}.



\paragraph{Objective}
%In an ordinary development, the main objective would be to verify or validate
%something. Here, in openETCS, it would be quite common to demonstrate
%the applicability of some method/tool, or evaluate its capabilities.


The goal is to produce a B model which implement the On-Sight procedure.
The B model shall be formally proved (verified) and shall allow to generate an executable C code.



\paragraph{Method/Approach}
% Short description of how the verification/validation is performed

At first, a formal model is defined with the B method using the AtelierB  tool.
Then the model is formally verified by proof and model-checking.
Functional properties can also be defined and validate by proof or model-checking on the B model.

Finally, C code is automatically translated from the B model.

\paragraph{Means/Tools}
% Could be integrated with the previous paragraph. Assign an
% approrpiate tool class (T1, T2, T3) according to EN 50128. Try to assign a
% maturity level 
%https://github.com/openETCS/validation/blob/master/VerificationAndValidationPlan/V02/VnVUsrStrTmplt-140709-02.pdf


The means used are:
\begin{itemize}
\item Atelier B to  design, check, verify and prove the model (qualified by industrial railway actors)
\item ProB to perform model-checking
\item Atelierb translators to produce C code (Code translator shall be T3 level to obtain certified code).
\end{itemize}


\paragraph{Results}
% Results related to the objective.
% Refer to appropriate document (preferably GitHub) for more complete description.


\begin{itemize}
\item The result is a fully formal model of the procedure On-Sight.
\item The C code has been automatically  translated.
\item The model  is formally proved.
\item Some properties have been checked by model checking.
\end{itemize}

See {\url{https://github.com/openETCS/validation/blob/master/Reports/D4.3/D4.3.1-Final-VV-report-on-model/D4.3.1.pdf}} and {\url{https://github.com/openETCS/validation/blob/master/VnVUserStories/VnVUserStorySysterel/04-Results/b-ClassicalB-VnV/BmodelVnV.pdf}}.



\paragraph{Observations/Comments}
% An optional section where anything can be included which has been
% observed without direct connection


\paragraph{Conclusion}
%What has been achieved, what is missing, what has been learned

The B~method, along with its verification processes and tools, meets the goals and activities of the openETCS project in terms of quality, rigor, safety and credibility.\\
There is yet to develop open-source POG and build a framework for proving, but this is compensated by the fact that work on the subject is ongoing, and ProB is an effective tool for verification.



\subsection{Verification and Validation in the SW Design Phase}
\label{sec:vnv-4}

Model-based testing applied to design models  \cmmnt{\qq{U Bremen}}




\subsubsection{Verification of Modes and Levels Management function (Systerel) }
\label{sec:}

\paragraph{Contributing project partners}
% Usually, one main partner, perhaps with contributions from others
The work has been performed by Systerel

\paragraph{Process step}
% Classification of the activity according to D2.3a
% Name what is verified and to which report this would contribute.
% Use the numbers, e.g. System Design Verification Report (1-12).


This activity contributes to:
\begin{itemize}
\item the Software Requirement Specification  (3.16)
\item the Software Specification Verification Report  (3.18)  
\item the Software Architecture and Design Specification (4.19)
\item the Software Design Verification Report  (4.23)  
\item the Software Components (5.24)
\item the Software Component Verification Report  (5.26)  
\end{itemize}



\paragraph{Object of verification}
% Which openETCS artifact (github link) or other documents/programs
% etc. (provide references)

The object of verification is the Scade model for the modes and levels management function at {\url{https://github.com/openETCS/modeling/tree/master/model/Scade/System/ObuFunctions/ManageLevelsAndModes}}. 


\paragraph{Available specification}
% The specification against which the object is to be checked. Usually
% coming form some openETCS artifacts (GitHub reference, process
% artifact number) or background material (reference, artifact number).

The model implements the requirements of modes and levels management function, as described in {\itshape System Requirements Specification, Chapter 4 and 5}.
All the chapter are not covered, only the part related to  the definition of current mode and level.


\paragraph{Objective}
%In an ordinary development, the main objective would be to verify or validate
%something. Here, in openETCS, it would be quite common to demonstrate
%the applicability of some method/tool, or evaluate its capabilities.


The goal is to produce a Scade model, to generate automatically executable C code, and to  verify properties on Scade model by model-checking.



\paragraph{Method/Approach}
% Short description of how the verification/validation is performed

At first, a formal model is defined with the Scade suite tool.
Then functional properties are formally verified on the model by  model-checking.

Finally, C code is automatically translated from the Scade model.

\paragraph{Means/Tools}
% Could be integrated with the previous paragraph. Assign an
% approrpiate tool class (T1, T2, T3) according to EN 50128. Try to assign a
% maturity level 
%https://github.com/openETCS/validation/blob/master/VerificationAndValidationPlan/V02/VnVUsrStrTmplt-140709-02.pdf


The means used are:
\begin{itemize}
\item Scade suite to  design, check and simulate the model 
\item Systerel Smart Solver (S3) to perform model-checking (can be certified as T2 tool)
\item KCG translator to produce C code (Code translator shall be T3 level to obtain certified code).
\end{itemize}


\paragraph{Results}
% Results related to the objective.
% Refer to appropriate document (preferably GitHub) for more complete description.


\begin{itemize}
\item The result is a Scade model of the mode and level management function integrated in the whole EVC scade model.
\item The C code has been automatically  generated.
\item Some properties have been checked by model checking.
\end{itemize}

See {\url{https://github.com/openETCS/validation/blob/master/Reports/D4.3/D4.3.1-Final-VV-report-on-model/D4.3.1.pdf}} and {\url{https://github.com/openETCS/validation/blob/master/VnVUserStories/VnVUserStorySysterel/04-Results/e-Scade_S3/Scade_S3_VnV.pdf}}.



\paragraph{Observations/Comments}
% An optional section where anything can be included which has been
% observed without direct connection


\paragraph{Conclusion}
%What has been achieved, what is missing, what has been learned



Benefits of formal  methods in an a posteriori verification process of critical systems has been recognized by our industrial customers:

\begin{itemize}
\item Contrarily to a human generated test-based verification solution, a formal safety verification is
intrinsically complete. It is equivalent to search for every possible falsification.
\item It clearly identifies the complete list of assumptions upon which the safety relies.
\item A certified solution allows for a reduction of the testing and review efforts (only the generic safety
specification has to be reviewed).
\item The use of formal verification in the qualification of critical software sends a strong
and positive message to the market, and is sometime even a requirement for some customers.
\end{itemize}



\subsection{Verification and Validation in the SW Component Phase}
\label{sec:vnv-5}

\begin{itemize}
\item Dedicated tests on single components \cmmnt{\qq{DB}}
\item Formal code verification (FRAMA C on the bitwalker)
\end{itemize}


\subsection{Verification and Validation in the SW Integration Phase}
\label{sec:vnv-6}

Automatized integration tests on the SW components. 

\subsection{Verification and Validation in the SW Validation Phase}
\label{sec:vnv-7}

There have been validations on
\begin{itemize}
\item the integrated software within the \qq{SCADE simulation
    environment}, subjecting the SW with a simulated environment to
  operational use cases.
\item an integration of the SW on a reference hardware, applying
  operational use cases.  
\end{itemize}

\section{Conclusion}
\label{sec:conclusion}


%\nocite{*}

\bibliographystyle{unsrt}

\bibliography{bibliography}



%===================================================
%Do NOT change anything below this line

\end{document}
