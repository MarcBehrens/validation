

\subsubsection{Verification of Management of Radio Communication function (Systerel) }
\label{sec:}

\paragraph{Contributing project partners}
% Usually, one main partner, perhaps with contributions from others
The work has been performed by Systerel

\paragraph{Process step}
% Classification of the activity according to D2.3a
% Name what is verified and to which report this would contribute.
% Use the numbers, e.g. System Design Verification Report (1-12).


This activity contributes to:
\begin{itemize}
\item the Elaborated System Requirements (1.07)
\item the Sub-System Requirement Specification (1.10)
\item the System Design Verification Report  (1.12)  
\end{itemize}



\paragraph{Object of verification}
% Which openETCS artifact (github link) or other documents/programs
% etc. (provide references)

The object of verification is the Event-B model for the communication
establishing at {\url{https://github.com/openETCS/model-evaluation/tree/master/model/Event_B_Systerel/Subset_026_comm_session}}. It
is from the strictly formal modeling phase and represents the communication
session management of the OBU.


\paragraph{Available specification}
% The specification against which the object is to be checked. Usually
% coming form some openETCS artifacts (GitHub reference, process
% artifact number) or background material (reference, artifact number).

The model implements the requirements for the communication session management
as described in Subset-026 chapter 3.5.

This section describes the establishing, maintaining and termination of a
communication session of the OBU with on-track systems.


\paragraph{Objective}
%In an ordinary development, the main objective would be to verify or validate
%something. Here, in openETCS, it would be quite common to demonstrate
%the applicability of some method/tool, or evaluate its capabilities.


One goal is the development of a strictly formal, fully proven model of the
communication session management and to provide evidence of covering the
necessary requirements of Subset-026 as well as proving correctness of the model
wrt.\ the requirements and attaining a good coverage of the model wrt.\ the
requirements.

The second goal is to correctly implement the applicable safety requirements
identified by the safety analysis. Both functional and safety requirements
should be traced in the model and a requirement document in a standardized
format.

The formal model will represent the described functionality on the system level,
the correct functioning can be validated by step-wise simulation and
model-checking of deadlock-freeness.


\paragraph{Method/Approach}
% Short description of how the verification/validation is performed

At first, the basic functionality described in the chapter 3.5 that are
identified. These serve as basis for a first abstract model, which is refined
iteratively, adding the desired level of detail. The elements of Subset-026 are
traced using links from Event-B to the ProR file in ReqIf format. Requirements
are formalized as invariants and proven where applicable.

\paragraph{Means/Tools}
% Could be integrated with the previous paragraph. Assign an
% approrpiate tool class (T1, T2, T3) according to EN 50128. Try to assign a
% maturity level 
%https://github.com/openETCS/validation/blob/master/VerificationAndValidationPlan/V02/VnVUsrStrTmplt-140709-02.pdf


The means used are:
\begin{itemize}
\item open source Rodin tool (\url{http://www.event-b.org/}), including plug-ins
  (for details
  see~\url{https://github.com/openETCS/model-evaluation/blob/master/model/Event_B_Systerel/Subset_026_comm_session/latex/subset_3_5.pdf})
\item ProR requirements modeling tool~\url{http://www.pror.org}
\item open source ProB model checker and B model
  simulator~\url{http://www.stups.uni-duesseldorf.de/ProB/index.php5/Main_Page}
\item open source CVC3 (\url{http://www.cs.nyu.edu/acsys/cvc3/}), verIT
  (\url{www.verit-solver.org}) and Alt-Ergo (\url{http://alt-ergo.lri.fr}) SMT
  solvers
\end{itemize}

All the tools are on class T1.

\paragraph{Results}
% Results related to the objective.
% Refer to appropriate document (preferably GitHub) for more complete description.


\begin{itemize}
\item The result is a fully formal model of the communication session management
  as described in chapter 3.5 of Subset-026.
\item Each implemented element of this section is linked to the ProR
  requirements file, both specification elements that describe how something has
  to be done, as well as requirements that describe what must be achieved.
\item The model can be simulated / animated, either with the AnimB or the ProB
  plug-in, validating the functional capabilities.
\item The safety requirements are formalized as invariants in predicate logic,
  their proofs are for the most part fully automatic.
\item It was found that while the Subset-026 communication management explicitly
  allows multiple communication partners (see RBC handover), there is no
  explicit limit of established communication connections given in chapter 3.5.
\item A complete covering of the elements of Subset-026 was not realized, e.g.,\
  there is a representation of the contents of a message, but its explicit
  format is not implemented. This is considered an implementation detail without
  influence for a system level analysis. In general, Event-B models will not be
  refined up to the implementation level.
\end{itemize}

See {\url{https://github.com/openETCS/validation/blob/master/Reports/D4.3/D4.3.1-Final-VV-report-on-model/D4.3.1.pdf}} and {\url{https://github.com/openETCS/validation/blob/master/VnVUserStories/VnVUserStorySysterel/04-Results/d-EventB-VnV/EventB-Rodin-VnV.pdf}}.



\paragraph{Observations/Comments}
% An optional section where anything can be included which has been
% observed without direct connection


\paragraph{Conclusion}
%What has been achieved, what is missing, what has been learned

Having an abstract formal model of the implemented functionality which can be
simulated, allows for interesting insights into the overall functioning of a
system. Formalized requirements are very helpful in both the identification of
ambiguous requirements and in their clarification.

The elements of Subset-026 are of very different nature. Some describe rather
low-level specification details, other describe ``real'' requirements. Without
an analysis as done with this Event-B model, it can be difficult to decide which
elements must be considered on a system level analysis and which on the lower
implementation level.

