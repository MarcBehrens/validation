
\chapter{Low-level bitstream operations}
\label{cha:low-level bitstream}

In this appendix, we describe the implementation of the
low-level bitstream operations.
%
They were used to implement the bit sequence abstraction level, cf.\
Section~\ref{sec:bit sequences}.
%
Since a write operation moves bits from a \lstinline{uint64_t} value into an array of
\lstinline{uint8_t} values, and a read operations moved them the other way round,
we need bit operations on both data types.
%
They are given in
Subsection~\ref{subsec:low-level 8 array} (array of \lstinline{uint8_t}), 
\ref{subsec:low-level 8} (single \lstinline{uint8_t}), and
\ref{subsec:low-level 64} (single \lstinline{uint64_t}) below.













\section{Reading and writing individual bits}


\subsection{8 bit arrays}
\label{subsec:low-level 8 array}









The operation \lstinline{TestBit8Array(addr,size,pos)} returns the
\lstinline{pos}$^{\mbox{\scriptsize th}}$ bit
within the array at \lstinline{addr}
of byte-size \lstinline{size}.\footnote{
	This parameter isn't actually used in the code, but merely in the contract.
}
Its contract and its implementation is shown in Listing~\ref{lst:TestBit8Array}.
%
See Listing~\ref{lst:bit predicate defs} for the definition of the predicate
\lstinline{Bit8Array}.
%
The array bits are counted starting with the most significant one of the first byte,
cf.\ Figure~\ref{fig:bit coords} below.
%
\lstinline{TestBit8(bytevalue,bitadr)} returns the 
\lstinline{bitadr}$^{\mbox{\scriptsize th}}$ bit
within \lstinline{bytevalue}, its is discussed in Appendix~\ref{subsec:low-level 8} below.








\begin{listing}[hbt]
\begin{minipage}{0.99\textwidth}
\begin{lstlisting}[style=acsl-block]
/*@
    requires valid:  \valid_read(addr + (0..size-1));
    requires size:   8 * size <= UINT32_MAX;
    requires pos:    pos < 8 * size;

    assigns \nothing;

    ensures result:  \result != 0 <==> Bit8Array(addr, pos);
*/
static inline int TestBit8Array(uint8_t*  addr, uint32_t size, uint32_t pos)
{
    return TestBit8(addr[pos / 8], pos % 8);
}
\end{lstlisting}
\end{minipage}
\caption{\label{lst:TestBit8Array}Reading a bit of an \inl{uint8_t} array}
\end{listing}










Similarly, the operation \lstinline{SetBit8Array(addr,size,pos,flag)} sets the
\lstinline{pos}$^{\mbox{\scriptsize th}}$ bit within the array at \lstinline{addr} of
byte-size \lstinline{size} to \lstinline{flag}.
%
Its contract is shown in Listing~\ref{lst:SetBit8Array}.
%
It requires
%
\begin{itemize}
\item the whole array to be accessible for update (requirement ``\lstinline{valid}''),
\item each possible bit position in the array to fit into a \lstinline{uint32_t}
	(``\lstinline{size}''), and
\item the given \lstinline{pos} to be a valid bit position in the array
	(``\lstinline{pos}'').
\end{itemize}
%
The \lstinline{assigns} clause allows the operation to change the contents of the array,
but no other memory locations.
%
On completion, the operation shall guarantee
\begin{itemize}
\item that the value of \lstinline{flag}\footnote{
	Any non-zero \lstinline{flag} value is treated like 1.
	This is ensured by teh contract of the called operation \lstinline{SetBit8},
	cf. Appendix{subsec:low-level 8}.
	}
	is actually stored at the designated bit
	position (``\lstinline{ensures middle}''), and
\item that all other bits remain unchanged 
	(``\lstinline{left}'', ``\lstinline{right}'').
\end{itemize}
%
Two fairly sophisicated hints had to be provided as assertions in the body in order for
the provers to establish the contract's post-conditions.









\begin{listing}[hbt]
\begin{minipage}{0.99\textwidth}
\begin{lstlisting}[style=acsl-block]
/*@
    requires valid:  \valid(addr + (0..size-1));
    requires size:   8 * size <= UINT32_MAX;
    requires pos:    pos < 8 * size;

    assigns addr[0..size-1];

    ensures left:   Unchanged{Here,Old}(addr, 0, pos);
    ensures middle: Bit8Array(addr, pos) <==> (flag != 0);
    ensures right:  Unchanged{Here,Old}(addr, pos + 1, 8 * size);
*/
static inline void SetBit8Array(uint8_t* addr, uint32_t size, uint32_t pos, int flag)
{
    uint32_t i = pos / 8u;
    uint32_t k = pos % 8u;

    addr[i] = SetBit8(addr[i], k, flag);

    // The following assertion claims that in byte with index "pos/8"
    // the bits with indices different from "k" do not change
    /*@
      assert bits_in_byte:
        \forall integer j; (0 <= j < 8  && j != k) ==>
        (Bit8(addr[pos/8], j) <==> \at(Bit8(addr[pos/8], j), Pre));
    */

    // The following assertion claims that in every byte
    // with an index that is different from "pos/8" no bit is changed.

    /*@
        assert other_bytes:
        \forall integer l, j; (0 <= l < size  &&  l != pos/8  &&  0 <= j < 8) ==>
          (Bit8(addr[l], j) <==> \at(Bit8(addr[l], j), Pre));
    */

}
\end{lstlisting}
\end{minipage}
\caption{\label{lst:SetBit8Array}Writing a bit of an \inl{uint8_t} array}
\end{listing}








\FloatBarrier

\subsection{8 bits}
\label{subsec:low-level 8}









\begin{listing}[hbt]
\begin{minipage}{0.99\textwidth}
\begin{lstlisting}[style=acsl-block]
/*@
    requires pre:  pos < 8;

    assigns \nothing;

    ensures pos:  \result != 0 <==> Bit8(value, pos);
*/
static inline int TestBit8(uint8_t value, uint32_t pos)
{
    uint8_t mask = ((uint8_t) 1) << (7u - pos);
    uint8_t flag = value & mask;

    return flag != 0;
}
\end{lstlisting}
\end{minipage}
\caption{\label{lst:TestBit8}Reading a bit of \inl{uint8_t}}
\end{listing}










The operation \lstinline{TestBit8(value,pos)} returns the
\lstinline{pos}$^{\mbox{\scriptsize th}}$
bit of \lstinline{value}.
%
Its contract is shown in Listing~\ref{lst:TestBit8}.
%
\lstinline{pos} must not exceed 7 (requirement ''\lstinline{pre}''),
no memory may be modified (\lstinline{assigns}), and the result is non-zero if, and only
if, the specified bit is set.
%
The shown implementation additionally guarantees that the result is zero or one, which
is not specified in the contract since this property isn't needed.
%
Returning just \lstinline{flag} rather than \lstinline{flag!=0u} would satisfy the
contract also, and would be slightly faster.

The definition of predicate \lstinline{Bit8} is shown in 
Listing~\ref{lst:bit predicate defs}.
%
It relies on the Frama-C library predicate \lstinline{BitTest}, performing a coordinate
transformation to fits Frama-C's notion of bit positions with the OpenETCS project's
notion, cf.\ Figure~\ref{fig:bit coords}.
%
In this report, we preferably use the terms ``least'' and ``most significant bit(s)'' to
designate a (range of) bit position(s) indenpendent of the coordinate system.






\begin{listing}[hbt]
\begin{minipage}{0.99\textwidth}
\begin{lstlisting}[style=acsl-block]
/*@
   predicate Bit8{A}(uint8_t v, integer n)  = BitTest(v, 7 - n);

   predicate Bit64{A}(uint64_t v, integer n) = BitTest(v, 63 - n);

   predicate Bit8Array{A}(uint8_t* a, integer n) = Bit8(a[n / 8],n % 8);
*/
\end{lstlisting}
\end{minipage}
\caption{\label{lst:bit predicate defs}Definition of bit test predicates}
\end{listing}








\begin{figure}
\begin{center}
\vspace*{2cm}
--- Insert drawing --- 
\vspace*{2cm}
\caption{\label{fig:bit coords}
        Bit coordinates in Frama-C and in the OpenETCS project}
\end{center}
\end{figure}






Dual to \lstinline{TestBit8}, the operation \lstinline{SetBit8(value,pos,flag)}
returns \lstinline{value}, with the \lstinline{pos}$^{\mbox{\scriptsize th}}$ bit
set to \lstinline{flag}.
%
Its contract is shown in Listing~\ref{lst:SetBit8}.
%
Again, \lstinline{pos} mustn't exceed 7 (requirement ``\lstinline{pre}''),
no memory may be modified (\lstinline{assigns} clause),
the return value coincides with \lstinline{value}, except possibly at \lstinline{pos}
(ensures ``\lstinline{left}'' and ``\lstinline{right}''),
and \lstinline{flag} is written to the approriate bit of \lstinline{value}
(``\lstinline{pos}'').
%
The implementation branches on the value of \lstinline{flag}, and clears or sets the
appropriate bit in the usual way.
%
Note that both our contract and our implementation enable us to set a bit by supplying a
\lstinline{flag} value of e.g.\ 2, whereas the code 
\lstinline{mask=flag<<(7-pos); return (value&~mask)|mask;} does not.





\begin{listing}[hbt]
\begin{minipage}{0.99\textwidth}
\begin{lstlisting}[style=acsl-block]
/*@
    requires pre: pos < 8;

    assigns \nothing;

    ensures left:   EqualBits8(\result, value, 0,  pos);
    ensures pos:    Bit8(\result, pos) <==> (flag != 0);
    ensures right:  EqualBits8(\result, value, pos + 1,  8);
*/
static inline uint8_t SetBit8(uint8_t value, uint32_t pos, int flag)
{
    uint8_t mask = ((uint8_t) 1) << (7u - pos);

    return (flag == 0) ? (value & ~mask) : (value | mask);
}
\end{lstlisting}
\end{minipage}
\caption{\label{lst:SetBit8}Writing a bit of \inl{uint8_t}}
\end{listing}























\FloatBarrier

\subsection{64 bits}
\label{subsec:low-level 64}



The operations to read and write a bit of a \lstinline{uint64_t} are closely similar to
those working on a \lstinline{uint8_t}.
%
They are shown in Listing~\ref{lst:TestBit64}
and \ref{lst:SetBit64} without repeating the comments given
in Appendix~\ref{subsec:low-level 8} for the 8 bit version; see in particular
Listing~\ref{lst:bit predicate defs} for the employed ACSL predicates.





\begin{listing}[hbt]
\begin{minipage}{0.99\textwidth}
\begin{lstlisting}[style=acsl-block]
/*@
   requires pre: pos < 64;

   assigns \nothing;

   ensures set_bit: \result != 0 <==> Bit64(value, pos);
*/
int TestBit64(uint64_t value, uint32_t pos)
{
    uint64_t mask = ((uint64_t) 1) << (63u - pos);
    uint64_t flag = value & mask;

    return flag != 0u;
}
\end{lstlisting}
\end{minipage}
\caption{\label{lst:TestBit64}Reading a bit of \inl{uint64_t}}
\end{listing}









As a redundant postcondition of the \lstinline{SetBit64} operation in
Listing~\ref{lst:SetBit64},
``\lstinline{upper}'' guarantees that the leading zeros in
\lstinline{value} are kept in the result, up to, but excluding position \lstinline{pos}.
%
It was needed to enable the provers to verify code that uses \lstinline{SetBit64}.




\begin{listing}[hbt]
\begin{minipage}{0.99\textwidth}
\begin{lstlisting}[style=acsl-block]
/*@
    requires pre: pos < 64;

    assigns \nothing;

    ensures left:     EqualBits64(\result, value, 0,  pos);
    ensures set_bit:  flag != 0  <==>  Bit64(\result, pos);
    ensures right:    EqualBits64(\result, value, pos + 1,  64);
    ensures upper:    \forall integer i; i >= 64 - pos ==>
                         (UpperBitsNotSet(value, i) ==> UpperBitsNotSet(\result, i));
*/
uint64_t SetBit64(uint64_t value, uint32_t pos, int flag)
{
    uint64_t mask = ((uint64_t) 1u) << (63 - pos);

    return (flag == 0) ? (value & ~mask) : (value | mask);
}
\end{lstlisting}
\end{minipage}
\caption{\label{lst:SetBit64}Writing a bit of \inl{uint64_t}}
\end{listing}




The operation \lstinline{UpperBitsNotSet64(value,length)} succeeds, i.e.\ returns a
non-zero value, if, and only if, all bits of \lstinline{value} except the least significant
\lstinline{length} ones are zero.
%
It is used for \fxfatal{ja, wofuer denn eigentlich?}





\begin{listing}[hbt]
\begin{minipage}{0.99\textwidth}
\begin{lstlisting}[style=acsl-block]
/*@
    requires pre: length <= 64;

    assigns \nothing;

    ensures  not_set: \result <==> UpperBitsNotSet(value, length);
*/
int UpperBitsNotSet64(uint64_t value, uint32_t length);
\end{lstlisting}
\end{minipage}
\caption{\label{lst:UpperBitsNotSet64}Test that upper bits are not set}
\end{listing}
% BODY OMITTED:
%{
%    if (length == 64)
%    {
%        return 1;
%    }
%    else
%    {
%        const uint64_t MaxValue = ((uint64_t) 1) << length;
%        // assert equiv: UpperBitsNotSet(value, length) <==> value < MaxValue;
%
%        return value < MaxValue;
%    }
%}










\FloatBarrier

\section{Formalization of bit operations in \framac}




\fxfatal{TODO}

describe \inl{BitTest}





\begin{lstlisting}[style=acsl-block]
predicate   BitTest(integer v, integer n);
\end{lstlisting}






\begin{lstlisting}[style=acsl-block]
predicate Bit8{A}(uint8_t v, integer n)  = BitTest(v, 7 - n);

predicate Bit64{A}(uint64_t v, integer n) = BitTest(v, 63 - n);

predicate Bit8Array{A}(uint8_t* a, integer n) = Bit8(a[n / 8],n % 8);
\end{lstlisting}










\begin{lstlisting}[style=acsl-block]
predicate
  UpperBitsNotSet{A}(integer value, integer length) =
    \forall integer i; length <= i ==> !BitTest(value, i);
\end{lstlisting}











\begin{listing}[hbt]
\begin{minipage}{0.99\textwidth}
\begin{lstlisting}[style=acsl-block]

\end{lstlisting}
\end{minipage}
\caption{caption}
\end{listing}

