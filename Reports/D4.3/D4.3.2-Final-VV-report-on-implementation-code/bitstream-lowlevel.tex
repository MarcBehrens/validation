
\chapter{Low-level bitstream operations}
\label{cha:low-level bitstream}

In this appendix, we describe the implementation of the
low-level bitstream operations.
%
They were used to implement the bit sequence abstraction level, cf.\
Chapter~\ref{cha:bitstream}.
%
Since a write operation moves bits from a \lstinline{uint64_t} value
into an array of \lstinline{uint8_t} values, and a read operations
moved them the other way round,
we need bit operations on both data types.
%
They are given in
Subsection~\ref{subsec:low-level 8 array} 
for an array of \lstinline{uint8_t}, 
in Subsection~\ref{subsec:low-level 8} for a single \lstinline{uint8_t},
and
in Subsection~\ref{subsec:low-level 64} for single \lstinline{uint64_t}.













\section{Reading and writing individual bits}


\subsection{8 bit arrays}
\label{subsec:low-level 8 array}


In this section, we discuss the operations for read and write of a
single bit from/into a byte array.

The operation \lstinline{TestBit8Array(addr,size,pos)} returns the
\lstinline{pos}$^{\mbox{\scriptsize th}}$ bit
within the array at \lstinline{addr}
of byte-size \lstinline{size}.\footnote{
	This parameter isn't actually used in the code, but merely
	in the contract.
}
Its contract and its implementation is shown in
Listing~\ref{lst:TestBit8Array}.
%
See Listing~\ref{lst:bit predicate defs} for the definition of the predicate
\lstinline{Bit8Array}.
%
The array bits are counted starting with the most significant one of
the first byte,
cf.\ Figure~\ref{fig:bit coords} below.
%
A call to \lstinline{TestBit8(bytevalue,bitadr)} returns the 
\lstinline{bitadr}$^{\mbox{\scriptsize th}}$ bit
within \lstinline{bytevalue}, this operation is discussed in
Appendix~\ref{subsec:low-level 8} below.








\begin{listing}[hbt]
\begin{minipage}{0.99\textwidth}
\begin{lstlisting}[style=acsl-block]
/*@
    requires valid:  \valid_read(addr + (0..size-1));
    requires size:   8 * size <= UINT32_MAX;
    requires pos:    pos < 8 * size;

    assigns \nothing;

    ensures result:  \result != 0 <==> Bit8Array(addr, pos);
*/
static inline int TestBit8Array(uint8_t*  addr, uint32_t size, uint32_t pos)
{
    return TestBit8(addr[pos / 8], pos % 8);
}
\end{lstlisting}
\end{minipage}
\caption{\label{lst:TestBit8Array}Reading a bit of an \inl{uint8_t} array}
\end{listing}










Similarly, the operation \lstinline{SetBit8Array(addr,size,pos,flag)}
sets the
\lstinline{pos}$^{\mbox{\scriptsize th}}$ bit within the array at
\lstinline{addr} of
byte-size \lstinline{size} to \lstinline{flag}.
%
Its contract is shown in Listing~\ref{lst:SetBit8Array}.
%
It requires
%
\begin{itemize}
\item the whole array to be accessible for update (requirement
	property ``\lstinline{valid}''),
\item each possible bit position in the array to fit into a
	\lstinline{uint32_t} (property ``\lstinline{size}''), and
\item the given \lstinline{pos} to be a valid bit position in the array
	(property ``\lstinline{pos}'').
\end{itemize}
%
The \lstinline{assigns} clause allows the operation to change the
contents of the array,
but no other memory locations.
%
On completion, the operation shall guarantee
\begin{itemize}
\item that the value of \lstinline{flag}\footnote{
		Any non-zero \lstinline{flag} value is treated like 1.
		This is ensured by the contract of the called operation
		\lstinline{SetBit8}, cf.\ Appendix~\ref{subsec:low-level 8}.
	}
	is actually stored at the designated bit
	position (postcondition property ``\lstinline{middle}'';
	the call \lstinline{Bit8Array()} succeeds if, and inly if, the
	\lstinline{pos}$^{\mbox{\scriptsize th}}$ bit within the byte
	array at \lstinline{addr} is set, 
	cf.\ Listing~\ref{lst:bit predicate defs} in
	Appendix~\ref{sec:bit operations in framac}), and
\item that all other bits remain unchanged 
	(properties ``\lstinline{left}'', ``\lstinline{right}'').
\end{itemize}
%
Two fairly sophisicated hints had to be provided as assertions in
the body in order for
the provers to establish the contract's post-conditions.









\begin{listing}[hbt]
\begin{minipage}{0.99\textwidth}
\begin{lstlisting}[style=acsl-block]
/*@
    requires valid:  \valid(addr + (0..size-1));
    requires size:   8 * size <= UINT32_MAX;
    requires pos:    pos < 8 * size;

    assigns addr[0..size-1];

    ensures left:   Unchanged{Here,Old}(addr, 0, pos);
    ensures middle: Bit8Array(addr, pos) <==> (flag != 0);
    ensures right:  Unchanged{Here,Old}(addr, pos + 1, 8 * size);
*/
static inline void SetBit8Array(uint8_t* addr, uint32_t size, uint32_t pos, int flag)
{
    uint32_t i = pos / 8u;
    uint32_t k = pos % 8u;

    addr[i] = SetBit8(addr[i], k, flag);

    // The following assertion claims that in byte with index "pos/8"
    // the bits with indices different from "k" do not change
    /*@
      assert bits_in_byte:
        \forall integer j; (0 <= j < 8  && j != k) ==>
        (Bit8(addr[pos/8], j) <==> \at(Bit8(addr[pos/8], j), Pre));
    */

    // The following assertion claims that in every byte
    // with an index that is different from "pos/8" no bit is changed.

    /*@
        assert other_bytes:
        \forall integer l, j; (0 <= l < size  &&  l != pos/8  &&  0 <= j < 8) ==>
          (Bit8(addr[l], j) <==> \at(Bit8(addr[l], j), Pre));
    */

}
\end{lstlisting}
\end{minipage}
\caption{\label{lst:SetBit8Array}Writing a bit of an \inl{uint8_t} array}
\end{listing}








\FloatBarrier

\subsection{8 bits}
\label{subsec:low-level 8}









\begin{listing}[hbt]
\begin{minipage}{0.99\textwidth}
\begin{lstlisting}[style=acsl-block]
/*@
    requires pre:  pos < 8;

    assigns \nothing;

    ensures pos:  \result != 0 <==> Bit8(value, pos);
*/
static inline int TestBit8(uint8_t value, uint32_t pos)
{
    uint8_t mask = ((uint8_t) 1) << (7u - pos);
    uint8_t flag = value & mask;

    return flag != 0;
}
\end{lstlisting}
\end{minipage}
\caption{\label{lst:TestBit8}Reading a bit of \inl{uint8_t}}
\end{listing}










The operation \lstinline{TestBit8(value,pos)} returns the
\lstinline{pos}$^{\mbox{\scriptsize th}}$
bit of \lstinline{value}.
%
Its contract is shown in Listing~\ref{lst:TestBit8}.
%
\begin{itemize}
\item The value of \lstinline{pos} must not exceed 7 
	(requirement property ''\lstinline{pre}''),
\item no memory may be modified (\lstinline{assigns}), and 
\item the result is non-zero if, and only
	if, the specified bit is set (postcondition property
	``\lstinline{pos}''; the call \lstinline{Bit8(value,pos)} succeeds 
	if, and only if, the \lstinline{pos}$^{\mbox{\scriptsize th}}$ of
	the byte \lstinline{value} is set, 
	cf.\ Listing~\ref{lst:bit predicate defs} in
	Appendix~\ref{sec:bit operations in framac}).
\end{itemize}
%
The shown implementation additionally guarantees that the result is
zero or one, which
is not specified in the contract since this property isn't needed.
%
Returning just \lstinline{flag} rather than \lstinline{flag!=0u}
would satisfy the
contract also, and would be slightly faster.






Dual to \lstinline{TestBit8}, the operation
\lstinline{SetBit8(value,pos,flag)}
returns \lstinline{value}, 
with the \lstinline{pos}$^{\mbox{\scriptsize th}}$ bit
set to \lstinline{flag}.
%
Its contract is shown in Listing~\ref{lst:SetBit8}.
%
\begin{itemize}
\item Again, the value of \lstinline{pos} mustn't exceed 7 
	(requirement property ``\lstinline{pre}''),
\item no memory may be modified (\lstinline{assigns} clause),
\item the return value coincides with \lstinline{value}, except possibly
	at \lstinline{pos} (postcondition properties ``\lstinline{left}'' 
	and ``\lstinline{right}''; a call
	\lstinline{EqualBits8(x,y,first,last)} succeeds if, and
	only if, the \lstinline{uint8_t} values \lstinline{x} and
	\lstinline{y} agree on all bits in range
	[\lstinline{first}\ldots\lstinline{last}), cf.\ also
	Listing~\ref{lst:EqualBits64} in 
	Appendix~\ref{sec:bit operations in framac}), and 
\item \lstinline{flag} is written to the approriate bit of 
	\lstinline{value} (property ``\lstinline{pos}'').
\end{itemize}
%
The implementation branches on the value of \lstinline{flag}, and
clears or sets the
appropriate bit in the usual way.
%
Note that both our contract and our implementation enable us to set
a bit by supplying a
\lstinline{flag} value of e.g.\ 2, whereas the code 
``\lstinline{mask=flag<<(7-pos);return(value&~mask)|mask}'' does not.





\begin{listing}[hbt]
\begin{minipage}{0.99\textwidth}
\begin{lstlisting}[style=acsl-block]
/*@
    requires pre: pos < 8;

    assigns \nothing;

    ensures left:   EqualBits8(\result, value, 0,  pos);
    ensures pos:    Bit8(\result, pos) <==> (flag != 0);
    ensures right:  EqualBits8(\result, value, pos + 1,  8);
*/
static inline uint8_t SetBit8(uint8_t value, uint32_t pos, int flag)
{
    uint8_t mask = ((uint8_t) 1) << (7u - pos);

    return (flag == 0) ? (value & ~mask) : (value | mask);
}
\end{lstlisting}
\end{minipage}
\caption{\label{lst:SetBit8}Writing a bit of \inl{uint8_t}}
\end{listing}





















\FloatBarrier

\subsection{64 bits}
\label{subsec:low-level 64}



The operations to read and write a bit of a \lstinline{uint64_t}
are closely similar to
those working on a \lstinline{uint8_t}.
%
They are shown in Listing~\ref{lst:TestBit64}
and \ref{lst:SetBit64} without repeating the comments given
in Appendix~\ref{subsec:low-level 8} for the 8 bit version.
%
See Listing~\ref{lst:bit predicate defs} for the employed ACSL predicates.





\begin{listing}[hbt]
\begin{minipage}{0.99\textwidth}
\begin{lstlisting}[style=acsl-block]
/*@
   requires pre: pos < 64;

   assigns \nothing;

   ensures set_bit: \result != 0 <==> Bit64(value, pos);
*/
int TestBit64(uint64_t value, uint32_t pos)
{
    uint64_t mask = ((uint64_t) 1) << (63u - pos);
    uint64_t flag = value & mask;

    return flag != 0u;
}
\end{lstlisting}
\end{minipage}
\caption{\label{lst:TestBit64}Reading a bit of \inl{uint64_t}}
\end{listing}








Listing~\ref{lst:SetBit64} shows the operation \lstinline{SetBit64}.
%
Note that it has a redundant postcondition, viz.\ property
``\lstinline{upper}'', which guarantees that the leading zeros in
\lstinline{value} are kept in the result, up to, but excluding
position \lstinline{pos}.
%
This property was needed to enable the provers to verify code that uses
\lstinline{SetBit64}.




\begin{listing}[hbt]
\begin{minipage}{0.99\textwidth}
\begin{lstlisting}[style=acsl-block]
/*@
    requires pre: pos < 64;

    assigns \nothing;

    ensures left:     EqualBits64(\result, value, 0,  pos);
    ensures set_bit:  flag != 0  <==>  Bit64(\result, pos);
    ensures right:    EqualBits64(\result, value, pos + 1,  64);
    ensures upper:    \forall integer i; i >= 64 - pos ==>
                         (UpperBitsNotSet(value, i) ==> UpperBitsNotSet(\result, i));
*/
uint64_t SetBit64(uint64_t value, uint32_t pos, int flag)
{
    uint64_t mask = ((uint64_t) 1u) << (63 - pos);

    return (flag == 0) ? (value & ~mask) : (value | mask);
}
\end{lstlisting}
\end{minipage}
\caption{\label{lst:SetBit64}Writing a bit of \inl{uint64_t}}
\end{listing}




The operation \lstinline{UpperBitsNotSet64(value,length)} succeeds,
i.e.\ returns a
non-zero value, if, and only if, all bits of \lstinline{value} except
the least significant
\lstinline{length} ones are zero.
%
It is used in the implementation of packet writing functions like
\lstinline{AdhesionFactor_EncodeBit} 
%\fxfatal{oder doch AdhesionFactor\_EncodeBit ?}
(see Section~\ref{sec:adhesionfactor-encodebit})
to check that no non-zero bits from the packet structure (like
\lstinline{struct AdhesionFactor}) are ignored due to space
limitations in the bitstream.





\begin{listing}[hbt]
\begin{minipage}{0.99\textwidth}
\begin{lstlisting}[style=acsl-block]
/*@
    requires pre: length <= 64;

    assigns \nothing;

    ensures  not_set: \result <==> UpperBitsNotSet(value, length);
*/
int UpperBitsNotSet64(uint64_t value, uint32_t length);
\end{lstlisting}
\end{minipage}
\caption{\label{lst:UpperBitsNotSet64}Test that upper bits are not set}
\end{listing}
% BODY OMITTED:
%{
%    if (length == 64)
%    {
%        return 1;
%    }
%    else
%    {
%        const uint64_t MaxValue = ((uint64_t) 1) << length;
%        // assert equiv: UpperBitsNotSet(value, length) <==> value < MaxValue;
%
%        return value < MaxValue;
%    }
%}











\FloatBarrier

\section{Formalization of bit operations in \framac}
\label{sec:bit operations in framac}






The definition of predicate \lstinline{Bit8} is shown in 
Listing~\ref{lst:bit predicate defs}.
%
It relies on the Frama-C library predicate \lstinline{BitTest},
performing a coordinate
transformation to fit Frama-C's notion of bit positions with the
OpenETCS project's
notion, cf.\ Figure~\ref{fig:bit coords}.





\begin{listing}[hbt]
\begin{minipage}{0.99\textwidth}
\begin{lstlisting}[style=acsl-block]
/*@
   predicate Bit8{A}(uint8_t v, integer n)  = BitTest(v, 7 - n);

   predicate Bit64{A}(uint64_t v, integer n) = BitTest(v, 63 - n);

   predicate Bit8Array{A}(uint8_t* a, integer n) = Bit8(a[n / 8],n % 8);
*/
\end{lstlisting}
\end{minipage}
\caption{\label{lst:bit predicate defs}Definition of bit test predicates}
\end{listing}













\begin{figure}
\begin{center}
\vspace*{2cm}
--- Insert drawing --- 
\vspace*{2cm}
\caption{\label{fig:bit coords}
        Bit coordinates in Frama-C and in the OpenETCS project}
\end{center}
\end{figure}






The predicate \lstinline{UpperBitsNotSet(value,length)} succeeds if,
and only if,
all but possibly 
the least significant \lstinline{length} bits of \lstinline{value} are zero.
%
Its definition is shown in Listing~\ref{lst:UpperBitsNotSet integer}.







\begin{listing}[hbt]
\begin{minipage}{0.99\textwidth}
\begin{lstlisting}[style=acsl-block]
predicate
  UpperBitsNotSet{A}(integer value, integer length) =
    \forall integer i; length <= i ==> !BitTest(value, i);
\end{lstlisting}
\end{minipage}
\caption{\label{lst:UpperBitsNotSet integer}
	Definition of the low-level predicate \lstinline{UpperBitsNotSet}}
\end{listing}


Listing~\ref{lst:EqualBits64} shows the predicate
\lstinline{EqualBits64} that was used in the 64-bit
operations' contracts.
%
The call \lstinline{EqualBits64(x,y,first,last)} succeeds if, and inly
if, the \lstinline{uint64_t} values  \lstinline{x} and  \lstinline{y}
agree on all bits in range [\lstinline{first}\ldots\lstinline{{last}).
%
The predicate \lstinline{EqualBits8}, used in
Appendix~\ref{subsec:low-level 8}, is defined in similar way; its
definition need not be shown here.










\begin{listing}[hbt]
\begin{minipage}{0.99\textwidth}
\begin{lstlisting}[style=acsl-block]
/*@
   predicate
     EqualBits64{A}(uint64_t x, uint64_t y, integer first, integer last) =
        \forall integer i; 64 - last <= i < 64 - first 
          ==> (BitTest(x, i) <==> BitTest(y, i));
\end{lstlisting}
\end{minipage}
\caption{\label{lst:EqualBits64}
        Definition of the low-level predicate \lstinline{EqualBits64}}
\end{listing}








In order for the provers to find all low-level validation proofs, we
needed to supply three redundand properties about
\lstinline{EqualBits64} and \lstinline{UpperBitsNotSet}; they are shown
in Listing~\ref{lst:axioms64}.
%
Axiom \lstinline{equal_bits64_0} states that two \lstinline{uint64_t}
values must be equal, if they agree on all 64 bit positions.
%
Axiom \lstinline{upper_bits_less_than} states that in a nonnegative number
less than $2^n$ all bits are zero, except for possibly
the least significant \lstinline{n} ones.
%
The neccessity of these extra axioms might indicate an
incompleteness in Frama-C's actual bit-operator theory; this is currently
investigated.
%
Axiom \lstinline{equal_bits64_1} is just a (relaxed)
rephrasing of the definition
in Listing~\ref{lst:EqualBits64}, using a different index scheme.
%
It is necessary due to the provers' weakness in applying index
transformations.










\begin{listing}[hbt]
\begin{minipage}{0.99\textwidth}
\begin{lstlisting}[style=acsl-block]
   axiomatic BitProperties
   {
     axiom equal_bits64_0:
       \forall uint64_t x, y;
         EqualBits64(x, y, 0, 64) ==> x == y;

     axiom equal_bits64_1:
       \forall uint64_t x, y, integer p, q;
         (\forall integer k; p <= k < q 
           ==> \let j = q-1-k; (BitTest(x, j) <==> BitTest(y, j)))
         ==> EqualBits64(x, y, 64-(q-p), 64);

      axiom upper_bits_less_than:
        \forall integer x, n; x >= 0 ==> n >= 0 ==>
           (UpperBitsNotSet(x, n) <==> x < (1 << n));
   }
*/
\end{lstlisting}
\end{minipage}
\caption{\label{lst:axioms64}ACSL axioms used in 64-bit contracts}
\end{listing}








For a nonnegative integer \lstinline{v},
the predicate \lstinline{BitTest(v,n)} succeeds if, and only if, the
\lstinline{n}$^{\mbox{\scriptsize th}}$ bit is set in the binary
representation of
\lstinline{v}, i.e.\ iff the truncating integer division of \lstinline{v} by
$2^{\mbox{\scriptsize\lstinline{n}}}$ yields an odd number.
%
This predicate comes with the standard library of the Frama-C system,
however, without any detailled documentation.
%
Its declaration is shown in Listing~\ref{lst:BitTest}.


\begin{listing}[hbt]
\begin{minipage}{0.99\textwidth}
\begin{lstlisting}[style=acsl-block]
/*@
   predicate   BitTest(integer v, integer n);
*/
\end{lstlisting}
\end{minipage}
\caption{\label{lst:BitTest}
	The Frama-C library predicate \lstinline{BitTest}}
\end{listing}











%	\begin{listing}[hbt]
%	\begin{minipage}{0.99\textwidth}
%	\begin{lstlisting}[style=acsl-block]
%	
%	\end{lstlisting}
%	\end{minipage}
%	\caption{caption}
%	\end{listing}

