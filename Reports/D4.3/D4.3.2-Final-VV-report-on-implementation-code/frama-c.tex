
\chapter{An introduction to formal verification with \framacwp}
\label{sec:frama-c}

Frama-C is a platform dedicated to source-code analysis of C software.
It has a plug-in architecture and can thus be easily extended to 
different kinds of analyses.
The WP plugin of Frama-C allows one to formally verify that a piece of
C code satisfies its specification.
This implies, of course, that the user provides a \emph{formal specification}
of what the implementation is supposed to do.
Frama-C comes with its own specification language ACSL which stands for
\emph{ANSI\slash ISO C Specification Language}.
In order to help potential users to master ACSL we discuss in this chapter 
a very simple C function \inl{abs_int} that implements the computation of
the absolute value for objects of type~\inl{int}.

\begin{itemize}
\item
In Section~\ref{sec:first-steps} we will present a straightforward
specification of \inl{abs_int}.
We discuss the reasons why \framacwp is not able to verify that our
implementation satisfies this specification in Section~\ref{sec:framac-failure}.

\item 
In Section~\ref{sec:contract-sharpening} we provide a more precise
specification that can be verified by \framacwp.
In Section~\ref{sec:separating-spec-impl} we explain
how \framac supports---by allowing the separation of the specification from the 
implementation---good software engineering practices.

\item
Sections~\ref{sec:modular-verification} and~\ref{sec:side-effects}
discuss, respectively,
how \framacwp supports \emph{modular verification} and the
formal treatment of \emph{side effects}.
\end{itemize}

\clearpage

\section{First steps}
\label{sec:first-steps}

We will consider the function that computes the absolute value $|x|$
of an integer $x$.
In order to avoid name clashes with the function \inl{abs} in C standard library
we use the name \inl{abs_int}.

The mathematical definition of absolute value is very simple
\begin{align}
\label{eq:abs}
   |x| &= \left\{
            \begin{array}{rl}
               x  & \text{if $x \geq 0$} \\
               -x & \text{if $x < 0$}
            \end{array}
          \right.
\end{align}

A straightforward implementation of \inl{abs_int} is shown in Listing~\ref{lst:abs}.

\begin{listing}[hbt]
\begin{minipage}{\textwidth}
\lstinputlisting[style=acsl-block ]{./Abs/abs.c}
\end{minipage}
\caption{\label{lst:abs} An implementation of the absolute value function}
\end{listing}

In order to demonstrate that this implementation is correct we have to provide
a formal specification.
Listing~\ref{lst:abs1} shows our first attempt for an ACSL specification of \inl{abs_int} that
is based on the mathematical definition of the function $x \mapsto |x|$ in Equation~\ref{eq:abs}.

\begin{listing}[hbt]
\begin{minipage}{\textwidth}
\lstinputlisting[style=acsl-block ]{./Abs/abs1.c}
\end{minipage}
\caption{\label{lst:abs1} A first attempt to formally specify \inl{abs_int}}
\end{listing}

\FloatBarrier

The first thing to note is that ACSL specifications---or \emph{contracts}---are placed in special C comments
(they start with \inl{/*@}).
Thus, they do not interfere with the executable code.
The \inl{ensures} clause in the specification expresses \emph{postconditions},
that is, properties that should be guaranteed \emph{after} the execution
of \inl{abs_int}.
The ACSL reserved word \inl{\\result} is used to refer to the return value of a C function.
Note that we use the usual C operators \inl{==} and \inl{<=} to express equalities and inequalities
in the specification.
There is also an additional operator \inl{==>} which expresses logical implication.

\section{Why can \framacwp not verify such a simple function?}
\label{sec:framac-failure}

Although the specification and implementation in Listing~\ref{lst:abs1} look perfectly right, 
\framacwp cannot verify that the implementation actually satisfies its specification.


The reason becomes clear if we look at some actual return values of \inl{abs_int}.
Listing~\ref{lst:test_abs} shows our test code whose output is
listed in Table~\ref{tbl:test_abs_output}.

\begin{listing}[hbt]
\begin{minipage}{\textwidth}
\lstinputlisting[style=acsl-block ]{./Abs/test_abs.c}
\end{minipage}
\caption{\label{lst:test_abs} Some simple test cases for \inl{abs_int}}
\end{listing}

%\clearpage

\begin{table}[hbt]
\begin{center}
\begin{tabular}{|r|r|c|}
\hline
\inl{x} &  \inl{abs_int(x)} & Remark \\ \hline\hline
0	&	0 & \checkmark \\ \hline
1	&	1 & \checkmark \\ \hline
10	&	10 & \checkmark \\ \hline
2147483647	&	2147483647 & \checkmark \\ \hline
-1	&	1 & \checkmark \\ \hline
-10	&	10 & \checkmark \\ \hline
-2147483648	&	-2147483648 & $\lightning$ \\ \hline
\end{tabular}
\end{center}
\caption{\label{tbl:test_abs_output} Test results for \inl{abs_int}}
\end{table}

\FloatBarrier

The offending value is in the last line of Table~\ref{tbl:test_abs_output}
which basically states that \inl{abs_int(INT_MIN)} equals \inl{INT_MIN}
whereas it should equal \inl{-INT_MIN}.
The problem is that the type \inl{int} only present a 
finite subset of the (mathematical) integers.
Many computers use a two's-complement representation of integers
which covers the range $[-2^{31}\ldots 2^{31}-1]$ on a 32-bit machine.
On such a machine \inl{-INT_MIN} cannot be  represented by a value
of the type~\inl{int}.

In a specification, \framacwp interprets integers as mathematical entities.
Consequently, there is no such thing as an \emph{arithmetic overflow} when
adding or multiplying them.
In other words,
\framacwp is perfectly right not being able to verify that \inl{abs_int}
satisfies the contract in Listing~\ref{lst:abs1}. Indeed, the 
implementation does not respect the given specification.


\section{Sharpening the contract of \inl{abs_int}}
\label{sec:contract-sharpening}

It is of course well known that the operation \inl{-x} can overflow
and it is the fact that \framacwp can detect such overflows that 
helps to prevent incorrect verification results.

The GNU Standard C Library clearly states that the absolute value of
\inl{INT_MIN} is undefined.\footnote{%
  See \url{http://www.gnu.org/software/libc/manual/html_node/Absolute-Value.html}
}
Under \textsf{OSX}, the manual page of \inl{abs} mentions under the field of ``Bugs'':
%
\begin{small}
\begin{verbatim}
    The absolute value of the most negative integer remains negative.
\end{verbatim}
\end{small}

Thus, our formal specification should exclude the value \inl{INT_MIN}
from the set of admissible value to which \inl{abs_int} can be applied.
In ACSL, we can use the \inl{requires} clause to express \emph{preconditions}
of a function.
Listing~\ref{lst:abs1a} shows an extended contract of \inl{abs_int}
that takes the limitations of the type \inl{int} into account.

\begin{listing}[hbt]
\begin{minipage}{\textwidth}
\lstinputlisting[style=acsl-block ]{./Abs/abs1a.c}
\end{minipage}
\caption{\label{lst:abs1a} Taking integer overflows into account}
\end{listing}

\FloatBarrier

\framacwp is now capable to verify that the implementation of
\inl{abs_int} satisfies the specification of Listing~\ref{lst:abs1a}.

There is an important lesson that can be learned here:
\begin{framed}
\label{lesson}
Sometimes developers provide source code and imagine that a tool
like \framacwp can verify the correctness of their implementation.
In order to fulfill its task, however, \framacwp needs an ACSL specification. 
Such a specification---which must be based on a reasonably precise description of the
admissible inputs and expected behavior---has to come from the \emph{requirements}
of the software and is not magically discovered from the source code by \framacwp.
The code does what it does. 
In order to verify that the code does what someone expects, these expectations
must be clearly expressed, that is, they must be formally specified.
\end{framed}

%\clearpage 

Of course, it might not always be the goal to verify the complete functionality of a
piece of software.
Sometimes, it is enough to ensure that individual software components
cause no runtime errors, that is, arithmetic overflows or invalid pointer accesses.
\framacwp can also be used in this situation.
Under the terms of the following minimal specification in 
Listing~\ref{lst:abs1b}, \framacwp can verify that no runtime error will occur.

\begin{listing}[hbt]
\begin{minipage}{\textwidth}
\lstinputlisting[style=acsl-block ]{./Abs/abs1b.c}
\end{minipage}
\caption{\label{lst:abs1b} Minimal contract to ensure the absence of runtime errors in \inl{abs_int}}
\end{listing}

%\clearpage

\section{Separating specification and implementation}
\label{sec:separating-spec-impl}

Before we continue exploring more advanced specification and verification
capabilities of \framacwp we turn to a simple software engineering question.

It is common practice to put function prototypes into ``\inl{.h}'' files and
keep the implementation in files ending in~``\inl{.c}''.
\framacwp supports this separation of specification and implementation.
Listing~\ref{lst:abs2-h} shows the file \inl{abs2.h} which contains
a declaration of \inl{abs_int} together with an attached ACSL specification.

\begin{listing}[hbt]
\begin{minipage}{\textwidth}
\lstinputlisting[style=acsl-block ]{./Abs/abs2.h}
\end{minipage}
\caption{\label{lst:abs2-h} Specifying a function prototype in a header file}
\end{listing}

Listing~\ref{lst:abs2-c} shows the specification of \inl{abs_int} in a~\inl{.c} file.
Note that the file \inl{abs2.h} with the specification is included by this file.
\framacwp can verify that this implementation satisfies the contract in
Listing~\ref{lst:abs2-h}.



\begin{listing}[hbt]
\begin{minipage}{\textwidth}
\lstinputlisting[style=acsl-block ]{./Abs/abs2.c}
\end{minipage}
\caption{\label{lst:abs2-c} Implementation at a different location than the specification}
\end{listing}

We remark, that the definition of a very small function like \inl{abs_int} would normally
be placed in a header file so that a compiler can inline the function definition
at the call site.

%\clearpage

\section{Modular verification}
\label{sec:modular-verification}

We now look at a simple example in which our function \inl{abs_int} is used.
More precisely, we include in Listing~\ref{lst:use_abs2-1} the
header file from Listing~\ref{lst:abs2-h} which contains an ACSL specification of \inl{abs_int}.

\begin{listing}[hbt]
\begin{minipage}{\textwidth}
\lstinputlisting[style=acsl-block ]{./Abs/use_abs2_1.c}
\end{minipage}
\caption{\label{lst:use_abs2-1} A simple example of modular verification}
\end{listing}

\FloatBarrier

When \framacwp tries to verify the code in Listing~\ref{lst:use_abs2-1},
then it actually tries to establish whether at each program location where
it is called the \emph{preconditions} of \inl{abs_int} are satisfied.
Based on the specification of \inl{abs_int},
\framacwp can indeed verify that for the first three calls the preconditions are fulfilled.
For the last call this verification fails because the value \inl{INT_MIN}
is explicitly excluded by the specification in Listing~\ref{lst:abs2-h}.

Note that the \emph{implementation} of \inl{abs_int}
does not play any role in determining whether it is safe to
call the function in a particular context.
This is what we call \emph{modular verification}: a function can be verified in
isolation whereas code that calls the function only uses the function contract.

This also means that in a situation as in Listing~\ref{lst:use_abs2-2},
where nothing is known about the argument of \inl{abs_int}, 
\framacwp cannot establish that the precondition of \inl{abs_int} is satisfied
or, in other words, that \inl{x > INT_MIN} holds.

\begin{listing}[hbt]
\begin{minipage}{\textwidth}
\lstinputlisting[style=acsl-block ]{./Abs/use_abs2_2.c}
\end{minipage}
\caption{\label{lst:use_abs2-2} Another example of modular verification}
\end{listing}

\FloatBarrier

If, on the other hand, we have precise information on the arguments at call site, then \framacwp can exploit the specification of 
\inl{abs_int} in order derive some interesting properties.
As an example, we consider the code fragment in Listing~\ref{lst:use_abs2-3}.
Here, \framacwp can verify that the assertion after 
the call of \inl{abs_int} is correct.


\begin{listing}[hbt]
\begin{minipage}{\textwidth}
\lstinputlisting[style=acsl-block ]{./Abs/use_abs2_3.c}
\end{minipage}
\caption{\label{lst:use_abs2-3} A more complex example of modular verification}
\end{listing}

Note that this assertion is a \emph{static} one, that is, it is
an ACSL annotation that resides inside a comment and does not affect
the execution of the code in Listing~\ref{lst:use_abs2-3}.
Also note that unlike in C~code, \emph{relation chains} with their usual mathematical meaning can be used both in function contracts and assertions.

%\clearpage

\section{Dealing with side effects}
\label{sec:side-effects}

Listing~\ref{lst:abs3a1} shows an implementation of \inl{abs_int}
that writes as a side effect the argument~\inl{x} to a global variable~\inl{a}.
A natural question is to ask whether this implementation with a side effect
also satisfies the specification.

\begin{listing}[hbt]
\begin{minipage}{\textwidth}
\lstinputlisting[style=acsl-block ]{./Abs/abs3a1.c}
\end{minipage}
\caption{\label{lst:abs3a1} An implementation with side effects}
\end{listing}

\FloatBarrier

Before we answer this question we consider various uses for side effects.
There are of course legitimate uses for side effects.
The assignment to a memory location outside the scope of the function
might be meaningful because an error condition is reported or because
some data are logged as in Listing~\ref{lst:abs3logging1}.

\begin{listing}[hbt]
\begin{minipage}{\textwidth}
\lstinputlisting[style=acsl-block ]{./Abs/abs3logging1.c}
\end{minipage}
\caption{\label{lst:abs3logging1} Calling a logging function from \inl{abs_int}}
\end{listing}

If \framacwp attempts to verify the code in Listing~\ref{lst:abs3logging1},
then it issues the following warning:
%
\begin{small}
\begin{verbatim}
    Neither code nor specification for function logging,
    generating default assigns from the prototype
\end{verbatim}
\end{small}
%
Thus, it points out that the called function \inl{logging} should have a proper
specification that clearly indicates its side effects.

There are, on the other hand, also good reasons to minimize or even forbid side 
effects:

\begin{itemize}
\item
Imagine a malicious password checking function that writes the password to
a global variable.

\item
Another reason is that side effects can make it harder to understand what 
the real consequences of a function call are.
In particular, one must be concerned about unintended consequences that
are caused by side effects
The norm IEC 61508 therefore requests in the context of software module testing
and integration testing:

\begin{quote}
To show that all software modules,
elements and subsystems interact correctly
to perform their intended function and do not perform unintended functions
(see also.~\cite[\S 7.4.7.2,\S 7.7.2.9]{IEC-61508-3})
\end{quote}

Of course, it is quite difficult to ensure by testing alone that something does \emph{not} happen.
\end{itemize}

To come back to our question about Listing~\ref{lst:abs3a1} it is important
to understand that \framacwp verifies that the implementation shown there
satisfies the specification.

If one wishes to forbid that a function changes global variables
one can use an \inl{assigns \\nothing} clause as shown in Listing~\ref{lst:abs3a2}.
\framacwp will then point out that this implementation prevents
the verification of the assigns clause.

\begin{listing}[hbt]
\begin{minipage}{\textwidth}
\lstinputlisting[style=acsl-block ]{./Abs/abs3a2.c}
\end{minipage}
\caption{\label{lst:abs3a2} Specifying the absence of side effects}
\end{listing}


\FloatBarrier

Of course, an all-or-nothing-approach to side effects is not very helpful
for the verification of real-life software.
Listing~\ref{lst:abs3a3} shows how the \inl{assigns} clause of a
specification can name the exact memory location that the
function is allowed to modify.

\begin{listing}[hbt]
\begin{minipage}{\textwidth}
\lstinputlisting[style=acsl-block ]{./Abs/abs3a3.c}
\end{minipage}
\caption{\label{lst:abs3a3} Finer control of side effects}
\end{listing}

Note however that \inl{assigns a} does not imply that a write to \inl{a}
necessarily occurs during the execution of \inl{abs}. On the other hand, any
other memory location must stay unchanged between the initial state
and the final state of \inl{abs}.

