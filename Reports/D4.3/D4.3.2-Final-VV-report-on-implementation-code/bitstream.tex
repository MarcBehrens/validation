

\chapter{The Bit Stream Layer}
\label{sec:bitstream}

\section{Formalization of bit operations in \framac}

describe \inl{BitTest}

\begin{lstlisting}[style=acsl-block]
predicate   BitTest(integer v, integer n);
\end{lstlisting}


\begin{lstlisting}[style=acsl-block]
predicate Bit8{A}(uint8_t v, integer n)  = BitTest(v, 7 - n);

predicate Bit64{A}(uint64_t v, integer n) = BitTest(v, 63 - n);

predicate Bit8Array{A}(uint8_t* a, integer n) = Bit8(a[n / 8],n % 8);
\end{lstlisting}


\begin{lstlisting}[style=acsl-block]
predicate
  UpperBitsNotSet{A}(integer value, integer length) =
    \forall integer i; length <= i ==> !BitTest(value, i);
\end{lstlisting}




\section{Reading and writing individual bits}

\subsection{64 bits}

\begin{listing}[hbt]
\begin{minipage}{0.99\textwidth}
\begin{lstlisting}[style=acsl-block]
/*@
   requires pre: pos < 64;

   assigns \nothing;

   ensures set_bit: \result != 0 <==> Bit64(value, pos);
*/
static inline int TestBit64(uint64_t value, uint32_t pos)
{
    uint64_t mask = ((uint64_t) 1) << (63u - pos);
    uint64_t flag = value & mask;

    return flag != 0u;
}
\end{lstlisting}
\end{minipage}
\caption{\label{lst:TestBit64}Reading a bit of \inl{uint64_t}}
\end{listing}


\begin{listing}[hbt]
\begin{minipage}{0.99\textwidth}
\begin{lstlisting}[style=acsl-block]
/*@
    requires pre: pos < 64;

    assigns \nothing;

    ensures left:     EqualBits64(\result, value, 0,  pos);
    ensures set_bit:  flag != 0  <==>  Bit64(\result, pos);
    ensures right:    EqualBits64(\result, value, pos + 1,  64);
    ensures upper:    \forall integer i; i >= 64 - pos ==>
                         (UpperBitsNotSet(value, i) ==> UpperBitsNotSet(\result, i));
*/
static inline uint64_t SetBit64(uint64_t value, uint32_t pos, int flag)
{
    uint64_t mask = ((uint64_t) 1u) << (63 - pos);

    return (flag == 0) ? (value & ~mask) : (value | mask);
}
\end{lstlisting}
\end{minipage}
\caption{\label{lst:SetBit64}Writing a bit of \inl{uint64_t}}
\end{listing}

\begin{listing}[hbt]
\begin{minipage}{0.99\textwidth}
\begin{lstlisting}[style=acsl-block]
/*@
    requires pre: length <= 64;

    assigns \nothing;

    ensures  not_set: \result <==> UpperBitsNotSet(value, length);
*/
static inline int UpperBitsNotSet64(uint64_t value, uint32_t length)
{
    if (length == 64)
    {
        return 1;
    }
    else
    {
        const uint64_t MaxValue = ((uint64_t) 1) << length;
        // assert equiv: UpperBitsNotSet(value, length) <==> value < MaxValue;

        return value < MaxValue;
    }
}
\end{lstlisting}
\end{minipage}
\caption{\label{lst:UpperBitsNotSet64}Test that upper bits are not set}
\end{listing}


\FloatBarrier

\subsection{8 bits}
\begin{listing}[hbt]
\begin{minipage}{0.99\textwidth}
\begin{lstlisting}[style=acsl-block]
/*@
    requires pre:  pos < 8;

    assigns \nothing;

    ensures pos:  \result != 0 <==> Bit8(value, pos);
*/
static inline int TestBit8(uint8_t value, uint32_t pos)
{
    uint8_t mask = ((uint8_t) 1) << (7u - pos);
    uint8_t flag = value & mask;

    return flag != 0;
}
\end{lstlisting}
\end{minipage}
\caption{caption}
\end{listing}

\begin{listing}[hbt]
\begin{minipage}{0.99\textwidth}
\begin{lstlisting}[style=acsl-block]
/*@
    requires pre: pos < 8;

    assigns \nothing;

    ensures left:   EqualBits8(\result, value, 0,  pos);
    ensures pos:    Bit8(\result, pos) <==> (flag != 0);
    ensures right:  EqualBits8(\result, value, pos + 1,  8);
*/
static inline uint8_t SetBit8(uint8_t value, uint32_t pos, int flag)
{
    uint8_t mask = ((uint8_t) 1) << (7u - pos);

    return (flag == 0) ? (value & ~mask) : (value | mask);
}
\end{lstlisting}
\end{minipage}
\caption{caption}
\end{listing}

\FloatBarrier

\subsection{8 bit arrays}

\begin{listing}[hbt]
\begin{minipage}{0.99\textwidth}
\begin{lstlisting}[style=acsl-block]
/*@
    requires valid:  \valid_read(addr + (0..size-1));
    requires size:   8 * size <= UINT32_MAX;
    requires pos:    pos < 8 * size;

    assigns \nothing;

    ensures result:  \result != 0 <==> Bit8Array(addr, pos);
*/
static inline int TestBit8Array(uint8_t*  addr, uint32_t size, uint32_t pos)
{
    return TestBit8(addr[pos / 8], pos % 8);
}
\end{lstlisting}
\end{minipage}
\caption{caption}
\end{listing}


\begin{listing}[hbt]
\begin{minipage}{0.99\textwidth}
\begin{lstlisting}[style=acsl-block]
/*@
    requires valid:  \valid(addr + (0..size-1));
    requires size:   8 * size <= UINT32_MAX;
    requires pos:    pos < 8 * size;

    assigns addr[0..size-1];

    ensures left:   Unchanged{Here,Old}(addr, 0, pos);
    ensures middle: Bit8Array(addr, pos) <==> (flag != 0);
    ensures right:  Unchanged{Here,Old}(addr, pos + 1, 8 * size);
*/
static inline void SetBit8Array(uint8_t* addr, uint32_t size, uint32_t pos, int flag)
{
    uint32_t i = pos / 8u;
    uint32_t k = pos % 8u;

    addr[i] = SetBit8(addr[i], k, flag);

    // The following assertion claims that in byte with index "pos/8"
    // the bits with indices different from "k" do not change
    /*@
      assert bits_in_byte:
        \forall integer j; (0 <= j < 8  && j != k) ==>
        (Bit8(addr[pos/8], j) <==> \at(Bit8(addr[pos/8], j), Pre));
    */

    // The following assertion claims that in every byte
    // with an index that is different from "pos/8" no bit is changed.

    /*@
        assert other_bytes:
        \forall integer l, j; (0 <= l < size  &&  l != pos/8  &&  0 <= j < 8) ==>
          (Bit8(addr[l], j) <==> \at(Bit8(addr[l], j), Pre));
    */

}
\end{lstlisting}
\end{minipage}
\caption{caption}
\end{listing}

\FloatBarrier


\section{Reading and writing bit sequences}


\begin{listing}[hbt]
\begin{minipage}{0.99\textwidth}
\begin{lstlisting}[style=acsl-block]
/*@
   predicate Readable{L}(uint8_t* addr, integer size) = \valid_read(addr + (0..size-1));

   predicate Writeable{L}(uint8_t* addr, integer size) = \valid(addr + (0..size-1));

   predicate Invariant{L}(integer size, integer bitpos, integer length) =
       8 * size <= UINT32_MAX         &&
       length <= 64                   &&
       bitpos + length <= UINT32_MAX;

    predicate Normal{L}(integer size, integer bitpos, integer length) =
       bitpos + length <= 8 * size;
*/
\end{lstlisting}
\end{minipage}
\caption{caption}
\end{listing}



\begin{listing}[hbt]
\begin{minipage}{0.99\textwidth}
\begin{lstlisting}[style=acsl-block]
/*@
  requires  valid:      Readable(addr, size);
  requires  invariant:  Invariant(size, bitpos, length);
  requires  normal:     Normal(size, bitpos, length);

  assigns   \nothing;

  ensures   equal:      EqualBits(addr, bitpos, bitpos + length, \result);
  ensures   upper:      UpperBitsNotSet(\result, length);
*/
uint64_t Bitwalker_Read(uint8_t* addr, uint32_t size, uint32_t bitpos, uint32_t length);
\end{lstlisting}
\end{minipage}
\caption{caption}
\end{listing}

\begin{listing}[hbt]
\begin{minipage}{0.99\textwidth}
\begin{lstlisting}[style=acsl-block]
/*@
  requires valid:      Writeable(addr, size);
  requires invariant:  Invariant(size, bitpos, length);
  requires normal:     Normal(size, bitpos, length);
  requires upper:      UpperBitsNotSet(value, length);

  assigns addr[0..size-1];

  ensures  left:       Unchanged{Here,Old}(addr, 0, bitpos);
  ensures  middle:     EqualBits(addr, bitpos, bitpos + length, value);
  ensures  right:      Unchanged{Here,Old}(addr, bitpos + length, 8 * size);
*/
void Bitwalker_Write(uint8_t* addr, uint32_t size, uint32_t bitpos, uint32_t length, uint64_t value);
\end{lstlisting}
\end{minipage}
\caption{caption}
\end{listing}



\FloatBarrier

\section{The \inl{Bitstream} abstraction}

\section{Formal verification}


\begin{listing}[hbt]
\begin{minipage}{0.99\textwidth}
\begin{lstlisting}[style=acsl-block]

\end{lstlisting}
\end{minipage}
\caption{caption}
\end{listing}

