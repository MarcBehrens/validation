\documentclass{template/openetcs_report}
% Use the option "nocc" if the document is not licensed under Creative Commons
%\documentclass[nocc]{template/openetcs_article}
\usepackage{lipsum,url}
\usepackage{supertabular}
\usepackage{multirow}
\usepackage{color, colortbl}
\definecolor{gray}{rgb}{0.8,0.8,0.8}
\usepackage[modulo]{lineno}
\graphicspath{{./template/}{.}{./images/}}

\begin{document}
\frontmatter
\project{openETCS}

%Please do not change anything above this line
%============================
% The document metadata is defined below

%assign a report number here
\reportnum{OETCS/WP4/D4.3.3V0.0}

%define your workpackage here
\wp{Work-Package 4: ``Validation \& Verification Strategy''}

%set a title here
\title{openETCS Safety case for tool chain and processes}

%set a subtitle here
\subtitle{Process and Toolchain verification for the openETCS on-board unit software development}

%set the date of the report here
\date{November 2015} %\\ Revised April 2015}

%document approval
%define the name and affiliation of the people involved in the documents approbation here
\creatorname{Jan Welte]}
\creatoraffil{TU Braunschweig}

\techassessorname{Abdelnasir Mohamed}
\techassessoraffil{AEbt}

\qualityassessorname{Veronique Gontier}
\qualityassessoraffil{All4Tec}

\approvalname{Klaus-R\"udiger Hase}
\approvalaffil{DB Netz}

%define a list of authors and their affiliation here

\author{Jan Welte}

\affiliation{Technische Universität Braunschweig\\
  Institute for Traffic Safety and Automation Engineering\\
  Hermann-Blenk-Str. 42\\
  38108 Braunschweig, Germany\\
  eMail: openetcs@iva.ing.tu-bs.de \\
  WebSite: www.iva.ing.tu-bs.de}
  
  
%add yourself as author, if you contributed to the document



% define the coverart
\coverart[width=350pt]{openETCS_EUPL}

%define the type of report
\reporttype{Output Document}


\begin{abstract}
This document addresses the general quality and safety assurance concept implemented and used by the openETCS development process and its respective toolchain. 

\end{abstract}

%=============================
%Do not change the next three lines
\maketitle
\tableofcontents
\listoffiguresandtables
\newpage
%=============================

\chapter{Document Control}

\begin{tabular}{|p{4.4cm}|p{8.7cm}|}
\hline
\multicolumn{2}{|c|}{Document information} \\
\hline
Work Package &  WP4  \\
Deliverable ID & D 4.3.3\\
\hline
Document title & Process and Toolchain verification for the openETCS on-board unit software development \\
Document version & 0.1 \\
Document authors (org.)  & Jan Welte (TU-BS)\\
\hline
\end{tabular}

\begin{tabular}{|p{4.4cm}|p{8.7cm}|}
\hline
\multicolumn{2}{|c|}{Review information} \\
\hline
Last version reviewed & \\
\hline
Main reviewers (org.) & \\
\hline
\end{tabular}

\begin{tabular}{|p{2.2cm}|p{4cm}|p{4cm}|p{2cm}|}
\hline
\multicolumn{4}{|c|}{Approbation} \\
\hline
  &  Name & Role & Date   \\
\hline  
Written by    &  Jan Welte & WP4-T4.4 Task Leader  &  November 2015\\
\hline
Approved by & -- & -- & \\
\hline
\end{tabular}

\begin{tabular}{|p{2.2cm}|p{2cm}|p{3cm}|p{5cm}|}
\hline
\multicolumn{4}{|c|}{Document evolution} \\
\hline
Version &  Date & Author(s) & Justification  \\
\hline
0.1 & 18/10/2013 & Jan Welte &  Document creation \\
\hline  
%0.1 & 28/01/2014 & Jan Welte &  Extended Introduction  \\
\hline  
\end{tabular}
\newpage

% The actual document starts below this line
%=============================

\mainmatter

\chapter{Introduction}
\label{sec:introduction}

..

\section{Purpose}
\label{sec:purpose}

...

\section{Document Structure}
\label{sec:document-structure}

...

\section{Document Evolution}

...

\section{Reference Documents}
\label{sec:refdoc}

This document essentially refers to the following standards, ETCS specification documents and openETCS project documents.

\begin{itemize}
\item \textbf{ISO~9000} --- 12/2005 --- \emph{Quality management}
\item \textbf{ISO~9001} --- 12/2008 --- \emph{Quality management systems — Requirements}
\item \textbf{ISO~25010} --- 03/2011 --- \emph{Systems and software engineering -- Systems and software Quality Requirements and Evaluation (SQuaRE) -- System and software quality models}
\item \textbf{CENELEC EN~50126-1} --- 01/2000 --- \emph{Railways applications –- The specification and 
demonstration of Reliability, Availability, Maintenability and Safety (RAMS) –- Part 1: 
Basic requirements and generic process}
\item \textbf{CENELEC EN~50128} --- 10/2011 --- \emph{Railway applications -- Communication, signalling and 
processing systems -- Software for railway control and protection systems}
\item \textbf{CENELEC EN~50129} --- 05/2003 --- \emph{Railway applications –- Communication, signalling and 
processing systems –- Safety related electronic systems for signalling}
\item \textbf{CCS~TSI} --- \emph{ CCS TSI for HS and CR transeuropean rail has been adopted by a Commission Decision 2012/88/EU on the 25th January 2012}
\item \textbf{SUBSET-026} 3.3.0 --- \emph{System Requirement Specification}
\item \textbf{SUBSET-091} 3.2.0 --- \emph{Safety Requirements for the Technical Interoperability
of ETCS in Levels 1 \& 2}
\item \textbf{SUBSET-088} 2.3.0 --- \emph{ETCS Application Levels 1 \& 2 - Safety Analysis}
\item \textbf{OpenETCS FPP} --- \emph{Project Outline Full Project Proposal Annex OpenETCS} -- v2.2
\item \textbf{OpenETCS D2.2} -- Report on CENELEC standard
\item \textbf{OpenETCS D2.3} -- Definition of the overall process for the formal description of ETCS and the rail system it works in 
\item \textbf{OpenETCS D2.4} -- Definition of the methods used to perform the formal description
\end{itemize}


%%%%%%%%%%%%%%%%%%%%%%%%%%%%%%%%%%%%%%%%%%%%%%%%%%%%%%%%%%%%%%%

\section{Glossary}
\label{sec:glossary}



\begin{tabular}{rl}
\textbf{ACedit} & Assurance Case Editor \\ 
\textbf{ARM} & Argumentation  Metamodel \\ 
\textbf{ETCS} & European Train Control System \\ \textbf{ERA} & European Railway Agency \\ \textbf{FMEA} & Failure Mode Effect Analysis \\ 
\textbf{GSN} & Goal Structured Notation \\ 
\textbf{MoRC} & Management of Radio Communication \\ 
\textbf{RAMS} & Reliability, Availability, Maintainability and Safety \\
\textbf{SIL} & Safety Integrity Level \\ 
\textbf{SRS} & System Requirement Specification \\ 
\textbf{THR} & Tolerable Hazard Rate \\ 
\textbf{V\&V} & Verification \& Validation \\ 
\end{tabular} 




\section{Background Information}
\label{sec:Background}


If specific information are needed the can be place here. (D4.2.3 shall not be repeated)


\chapter{Tool Chain}

\section{overview}

by Jan Welte

\section{Tool Qualification}

by Michael Jastram (or other expert from WP7)


broad overview of the toolchain and the status of qualification (generall information can be placed in section Overview)
- which tools have to be qualified
- which tools are qualified? (in which way)
- how should qualification be address for tools with pending qualification

\section{SCADE}

by Jan Welte and Marc Behrens

- use of SCADE for quality assurance
- limitations of SCADE
- addressing safety issues and properties in SCADE 
(potential specific aspects in openETCS deviation from the usual use of SCADE)


\section{Safety Architect}

by FrederiqueVallee (or Francois Revest)

- use of Safety Architect in openETCS (maybe addressing relation to Eclipse Safety Framework)
- function in development process
- inputs and outputs
- results (in general, and specific for openETCS)

\chapter{OpenETCS Development}
\label{sec:development-process}

\section{overview}

by Jan Welte

Short overview of current work.

- Main principals to ensure consistency 

- Mainly collecting findings

- allocate the tools to the process steps used/ qualified

\section{Compatibility to CENELEC standards}

by Mohamed Abdelnasir

- overview results relation to EN 50126/50128 lifecycle 
- reasons for deviations
- additional findings

\section{Traceability}

by @janwelte @raphaelfaudou

- addressing specific position of traceabilty for safety argumentation
- introducing basic concept
- main findings (limitations)

\chapter{Generic OpenETCS Safety Case}
\label{sec:hazardandrisk}

\section{System/ Sub-System Definition}

by Jan Welte

- general information concerning openETCs system and sub-system structure
- potential applications for artifacts

\section{Quality Management}

by Mohamed Abdelnasir

- basic concept for quality management in openETCS
- missing aspects in quality management
- main finding to address additional measures to complete quality management

\section{Safety Management}

by Jan Welte

- basic concept for safety management in openETCS
- missing aspects in safety management
- main finding to address additional measures to complete safety management

\section{Functional/Technical Safety}
by Jan Welte

- addressing general system safety properties and allocation to functional structure
- listing needed integration properties for "safe" use of software model (specifically interface assumptions)

by Francois Revest

- addressing concrete findings from safety propagation analysis
- additional measures applicable to tackle open points


\chapter{Conclusion}
\label{sec:conclusion}

This document presents the final results ...

\bibliographystyle{unsrt}
\bibliography{./ref/ref-HaRA}


%===================================================
%Do NOT change anything below this line

\end{document}


