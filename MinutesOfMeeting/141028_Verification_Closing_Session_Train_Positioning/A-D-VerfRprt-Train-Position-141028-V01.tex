\documentclass{article}

\usepackage{verbatim}
\usepackage{hyperref}
\usepackage{lineno}
\newcommand{\FIXME}[1]{\marginpar{FIXME}\textsf{FIXME: #1}}
\linenumbers


 \title{Verification Report for Architecture and Design of Train Positioning \\Version 0.1}
 \author{Marc Behrens (DLR), Bernd Gonska (DLR),\\ Jens Gerlach (Fraunhofer), Bernd Hekele (DB),\\ Jan Welte (TU-BS)} 
 \date{Oct 29, 2014}

%based on VnVRprtTmplt-131101-02.tex by Hardi Hunger

\newcommand{\tbi}[1]{$<$\textit{#1}$>$}

% Starts a new line nearly everywhere
\newcommand{\nl}{\mbox{}\\}
\newcommand{\nlskip}[1]{\mbox{}\\[#1]}

%
%Comments
\newcommand{\cmmnt}[1]{\framebox{#1}}
\newcommand{\bgcmmnt}[1]{\nl\framebox{\parbox{.95\textwidth}{#1}}\nl[2mm]}
%\renewcommand{\bgcmmnt}[1]{}
%

\newcommand{\eod}{\nl\rule{.95\textwidth}{1pt}\nl\textit{End of Document}}

\begin{document}
\maketitle

\begin{abstract}

This verification report presents the verification results for the architecture, interfaces and design artifacts for the component "Train Positioning" in the overall openETCS Kernel architecture.

\begin{comment}
This template provides the required content to complete the verification of architecture and design artifacts.
To close the development phase for this artifact all required information shall be given, even if it can only be stated that specific aspects are missing in the artifact due to open points in related artifacts. 

The template should be used as a guideline to check whether all
information is given appropriately. The wording used in this proposal
is by no means mandatory. And if you feel that more information is
useful to describe your activity within the context of openETCS, you
should of course do so. Feel free to add additional categories of
description as adequate. 

Also the \LaTeX{} macros may be changed, though the use of
\texttt{paragraph} and \texttt{subparagraph} enables easy integration
into higher-level documents (they are not numbered automatically,
which may be a drawback in other respects). 
\end{comment}
\end{abstract}

\section{Roles}

\begin{itemize}
\item Fausto Cochetti - Design (PM)
\item Jens Gerlach - Verification of SW/Implementation
\item Uwe Steinke - Design/Implementation
\item Jan Welvaarts - Design
\item Vincent Nuhaan - Simulation/Design
\item Bernd Gonska - Verification
\item Marc Behrens - Verification
\item Jan Welte - Verification
\item Bernd Hekele - Verification
\end{itemize}


\section{Verification Object}

\input{verification_object}


\section{Software Architecture, Interface and Design Verification}

This section presents all verification results concerning the verification object \texttt{Train Positioning}. 

\begin{comment}
\bgcmmnt{For all verification aspects addressed in the following section the following 3 points shall be state clearly:
\begin{enumerate}
\item Responsible verifier
\item Use verification strategy and technique (with reference to the V\&V Plan)
\item Verification results (level of conformity, detected errors or deficiencies and made assumptions)
\end{enumerate}
\end{comment}

The following subsection present all different verification aspects in accordance with EN 50128 7.3.4.42 \cite{EN50128}.

\subsection{Internal Consistency}

\textit{by Jan Welte and Marc Behrens}



Contant:
\begin{itemize}
\item relations
\item historical development
\item claim of same approach
\item of naming between documents consistence 
\end{itemize}

Are the internal functional allocation and all related input and output consistent?


\subsection{Adequacy to fulfill Software Requirements}
by Bernd Gonska

\subsubsection{Description of the developed train positioning system}
The train positioning system clearly deviates from the SRS on purpose. This is justified by the intended operational performance.

It basically implements the following concepts:
\begin{itemize}
\item All distances are given as the triple of safe distances (minimum,nominal,maximal)

\item The estimated position, (also named nominal position) is calculated to be the middle of the maximum safe position and the minimum safe position.

\item Each Balise Group (BG) has a an own accuracy and position, relative to the Last Relevant Balise Group (LRBG) and the LRBG accuracy. Locations do not change their reference BG.
  
\item Linking distances and accuracies are used to improve the accuracy when ever possible.

\item Accuracy of a distance is calculated taking the worst possible case into account. For example: The accuracy of the distance between two ends of a linking chain includes the first and the last BG accuracy. The distance between two BG without linking is the odometry measured distance. The accuracy is the odometry acuracy during that distance. This inaccuracy is not reseted later.

\item The confidence interval of an announced location never increases when a new BG is accepted. Always use the most accurate information. The odometry error estimation is trustworthy enough to optimize linking accuracy and distances.
\end{itemize}

\subsubsection{Deviations}
Within this paragraph the deviations to the specification is described by first mentioning the number of the Subset-026 paragraph \cite{SRS}
and then stating how the design deviates:

\begin{description}
\item[3.6.4.1 REMARK:] There are several confidence intervals: They depend on the announced location.
\item[3.6.4.2:] In addition, the odometry inaccuracy of older track areas and older linking accuracy can be taken into account to widen a the confidence interval for safety reasons. The location accuracy of the LRBG is shortened on if the detected Balise group position is extreme. An old confidence interval can be taken instead of a larger new one.//
3.6.4.2.2: An odometer inaccuracy may not be reset at the new LRBG.
\item[3.6.4.3:] Even if the linking chain is not complete, linked parts replace odometry distances if they provide higher accuracy.

In the "sandwich problem" where two linked BGs enclose an unlinked BG 
(linked BG1 $\rightarrow$ unlinked BG2 $\rightarrow$ linked BG3) 
the distance and the accuracy between BG2 and BG3 can be calculated involving the linking accuracy of BG1and BG3 and the linking distance between BG1 and BG3. 

The estimated distance may differ from the linking distance and from the odometry measured distance. The estimated distance is set to the middle of the maximum and minimum safe distance.

\item[3.6.4.3.1:]
The train takes responsibility, it does not reset inaccuracies if this could lead to unsafe behavior. 
\item[Figure 13 a,b,c:]
There is more than one confidence interval.

The confidence interval is calculated differently.

The estimated distance can be different since it is the middle of the maximum and minimum safe distance.

Linking distance is not used if it leads to a less accurate distances.

\item[3.6.4.4:]
The estimated distance is the middle of the maximum and the minimum safe position with respect to the possible LRBG position. It may differ from the measured traveled distance.

\item[3.6.4.4.1:]
analogue for the rear end position.

\item[3.6.4.7.1:]
The unlinked BG confidence interval is not reset at the next LRBG.

\item[3.6.4.7.2]
The unlinked BG confidence interval is not reset at the next unlinked BG. In some cases the estimated traveled distance between two unlinked BG is calculated by using other rules.
\end{description}


\subsubsection{Readability and Traceability}

by Marc Behrens

Content
\begin{itemize}
\item traceability of requirements
\item unique references
\end{itemize}


Are all related system and software requirements uniquely referenced and is the relationship to other documents clearly defined?
Are all parts of the architecture and inputs and outputs referenced to the related requirements.
Are the elements referred to in the same way in all documents?




\subsubsection{Consideration of hardware and software constraints}



Hardware design is out of scope of the openETCS project. 
No hardware assumptions have been formulated so far.

\noindent 
Software constraints encompass:

\begin{enumerate}
   \item Constraints by the software design method. 
         The design should rely on
   \begin{itemize}
   \item modelling
   \item a modular approach
   \item defensive programming
   \end{itemize}

   \item Restrictions implied by the coding standards.
         The coding standard should include
   \begin{itemize}
   \item a coding style guide
   \item restrict the use of pointers, dynamic objects, recursion and global variables
   \end{itemize}

   \item Constraints on timing, performance, or memory of  individual modules.
   \item Any constraints implied by the interfacing system (e.g. decoder and encoder functions).
   \item Constraints of the operating system.
\end{enumerate}




\section{Conclusions}
Concluding the meeting the following points were agreed on as action item list:

\subsection{Implementation and Design}

Differences within the simulation and implementation has been detected
and the two approaches have to be merged (Jan: Specify, Uwe: Implement)
\begin{enumerate}
\item synchronize on using the inaccuracies of the BG passed or just of the last BG and the 1st BG (Uwe and Jan Welvaarts)
\item backwards calculation of inaccuracies need to be synchronized (Uwe, Jan)
\item agree on the output data (Uwe, Jan Welvaarts, Vincent)
\item synchronize data structure for balise storage and storage of other trackside related information
\item agree on the input data (Uwe, Jan Welvaarts, Vincent)
\item integer arithmetic only on the model (design decision of WP3)
\item resolution of train position in cm (design decision of WP3)
\item Inaccuracy of the Odometer is agreed to be taken into account (Uwe)
\item Center Detection accuracy is agreed to be taken into account (Uwe)
\item define national values to be used (Jan Welvaarts)
\item national values can also change at some point on track (Uwe)
\item a as common reference for saving the data the start-up position of the OBU is taken
\item rules have to be implemented correctly
\item  Q\_LOCACC will be made dependant on the national values and on the linking information (Uwe, Vincent)
\item  Implement distance calculation to track objects, other than balises (Uwe)
\end{enumerate}

\subsection{Scenarios}

Concerning the scenario simulation the following was decided:
\begin{enumerate}
\item Two scenarios to contain all possible situations are to be sketched (Jan)
\item Calculate distance related values with the LabView Simulation to be compared with the SCADE execution (Vincent)
\end{enumerate}

\subsection{Report}

Concerning the follow up of the verification report the following points were decided:
\begin{enumerate} 
\item Verification report is to be written (Marc, Jan Welte, Jens Gerlach, Bernd Gonska)
\item once the design is merged and implemented the verification will rework the results based on the merged version (Marc)
\item traceability to the SRS will be delivered (Jan, Marc)
\item justification on the deviations need to be documented (Jan, Uwe)
\end{enumerate}
               
\bibliographystyle{plain}
\bibliography{verification}

\end{document}

