\section{Two-step Approach}




% Two-step approach:
% first step: over-approximation, build a statemachine with input equivalence
% classes. 
% 
% 
% In the current flow of the ECPT\todo[inline]{ECPT erklaeren in Background}
% approach tests are generated automatically from the state machine. At first the state machine is transformed into an internal
% representation. The behavioral semantics are defined by a state transition
% system. On this an input equivalence class partitioning is applied, resulting in
% a finite state machine, on which the W-method can be applied. By selecting
% concrete test data from the input equivalence classes, concrete test cases can
% be obtained. Concrete test data can be calculated, because all concrete system
% inputs state changes are modeled explicitly by state machines. 
% 
% Drawbacks: no notion of time, modelling effort


\paragraph{Step~1. System abstraction}

In the first step, the system behavior is abstracted. The idea is to use an 
overapproximation of the specified system's behavior: all traces possible in the concrete system based on the specification, considering all
physical and system related constraints, are also observable in the abstract
model. Whether a trace in the abstract model has a concrete solution in the real
world, depends on the abstraction and constraints. This is a conservative
approach, which guarantees that every system behavior is covered by the model,
while unrealistic traces might remain. If in the second step, no concrete
solution for an abstract test exists, it can be deduced, that this behavior is
not possible (assuming that the constraints are correct), and therefore it is
safe to neglect this test case.

%In the model abstraction, guard conditions occurring in state machines are 
%abstracted by Boolean variables, following the technique for generating simulations
%as described in~\cite{clarke_em-etal:1999a}. Additional physical constraints modeled in 
%constraint blocks are disregarded at this stage. Every state sequence that may occur
%in the concrete system is represented by an   


\begin{example}
\label{ex:abstraction}
%One way to include abstraction is to introduce abstract input symbols.
Consider  the following complex boolean expression, taken from the
ETCS specification \cite{ETCSSRS-Principles}:

\begin{align}
V_\text{est} > V_\text{mrsp} + DV_\text{ebi}(V_\text{mrsp}) 
\lor D_\text{front} > D_\text{EBI} 
\end{align} 

This boolean expression describes a trigger condition for the onboard controller
of the train to automatically activate the 
%If the condition is fulfilled, the system shall automatically activate the
%train's 
emergency brakes. Variable $V_\text{est}$ describes the current train
speed.
If this speed exceeds the maximum allowed speed $V_\text{mrsp}$ by more than an
offset $DV_\text{ebi}$, the emergency brakes are activated. Alternatively,
%If the train speed is
%below this threshold, but 
if the front of the train $D_\text{front}$ is too close
to the target location (this is expressed by $D_\text{front} > D_\text{EBI}$), the emergency brakes are
activated as well.

Instead of explicitly describing this trigger condition by concrete model variables, an abstract input variable
$t_{13}$ is introduced. The state machine transition that uses this abstract input variable looks like
this:
\begin{tikzpicture}
\node[draw,rounded corners] (l1) {L1};
\node[draw,rounded corners, right=1cm of l1] (l2) {L2};
\draw[->] (l1) -- (l2) node[pos=0.5,above] {$\lbrack t_{13}\rbrack$};
\end{tikzpicture}.
In this case we introduced a new free input variable. It can be assumed, that
the value of this variable can change arbitrarily over time. This is a safe
overapproximation of the real system.
\end{example}

The replacement of a complex guard-condition by a new boolean input
variable is always possible, but $n$ replacements introduce $n$ new inputs.
In the worst case this results in $2^n$ new input equivalence classes of the SUT~\cite{huang_complete_2014}: each feasible conjunction of the positive or negated 
Boolean expressions associated with each of the abstraction variables $t_i$ specifies
an input equivalence class.
For realistic models, 
not all combinations of valuations of the complex conditions are possible.
Using an SMT solver, impossible input combinations can be dropped a priori. These combinations
are ignored in the test generation.  

With this abstract description of the system behavior at hand, we are able to
apply the test generation strategy 
presented in~\cite{huang_complete_2014} and create a test suite,
whose inputs are represented by abstract equivalence classes.
%Note that our approach described in this paper is independent of the concrete
%test strategy, and every other strategy could be applied as well.

\begin{example}
For our running example, the TSM function, we introduced nine abstract input variables. 
From the $2^9=512$ possible combinations, only eight combinations were valid. Two of the combinations
can be considered as equivalent~\cite{huang_complete_2014}. Thus, seven input equivalence classes
were calculated from our abstract model.
\end{example} 

The result of the first step in our test case generation is a set of
abstract test cases. An abstract test case is a sequence of system states
$\langle\bar{s}_0,\ldots,\bar{s}_n\rangle$, where $\bar{s}_i: V_\text{abstract} \rightarrow D$ is the
valuation function, mapping the (abstract) Boolean system variables $t_k$
 to their values $\bar{s}_i(t_k)$    in the
$i$-th state.

% ===============================================================
\paragraph{Step 2. Concretization}\label{sec:concretization}

While the first step defined abstract test cases that are related to
computations in the abstract test model, the second step
aims at constructing concrete test data.
%for system tests concrete test data is
%needed. For testing it is necessary to specify concrete test data composed
%of inputs that stimulate the SUT and outputs that can be observed.

A concrete test case is a sequence of concrete 
system states $\langle s_0,\ldots,s_n\rangle$. Each $s_i$ is a valuation function   
$s_i: V\cup  V_\text{abstract} \rightarrow D$,
where the concrete system variables $v$, including the inputs and outputs, as well as the 
abstraction variable introduced before, are mapped
to concrete values  $s_i(v)\in D$.

For the concretization of the test cases, parametric diagrams were
added as an additional description means. The 
abstract variables introduced in Step~1 have to be related to concrete system
variables by parametric diagrams.
Two types of constraints can be
defined: \emph{state invariants} and \emph{temporal evolution}.

State invariants constrain the concrete system variables in every state and, therefore, in every test step of the test case, these
invariants have to be fulfilled. In a parametric diagram
a state invariant can be described by a constraint property. A parametric
diagram with $m$ constraint properties that define state invariants
($\text{INV}_1,\ldots,\text{INV}_m$) yields the invariant:

$\text{INV} \equiv \bigwedge_{j=1}^{m} \text{INV}_j$

Invariants in the concrete system are also used 
to bind the values of abstract system variables to
the concrete guard conditions they are abstracting.
\begin{example}
Considering trigger condition $t_{13}$ from example~\ref{ex:abstraction} the corresponding parametric diagram is shown in Figure~\ref{fig:parametric}. In this diagram $t_{13}$ gets bound to its concrete guard condition. 
\end{example}
Given this invariant, and given an abstract test case
\mbox{$\langle\bar{s}_0,\ldots,\bar{s}_n\rangle$}, we want to calculate a concrete test case
\mbox{$\langle s_0,\ldots,s_n\rangle$} that uses the same valuations of the abstract 
system variables -- that is, the same equivalence classes -- as prescribed by the
abstract test case. This is encoded by the following formula, where
%
%
%
%. To describe this problem as an SMT instance, let
%$V_i$ denote the variable symbols from $V\cup V_\text{abstract}$ in the $i$-th state. $v_i$ denotes the variable $v$ in the
%$i$-th state. 
%In the following formulas, 
$\phi[t/x]$ denotes  the formula derived from $\phi$ by exchanging every free occurrence of $x$ by the term $t$.


\begin{align}
\bigwedge_{i=0}^{n}& \big(\bigwedge_{t\in V_\text{abstract}} s_i(t) =\bar{s}_i(t) \wedge {}\\
& 
%\text{INV}[\bar{s}_i(v')/v'\in V_\text{abstract},v_i/v\in (V\setminus V_\text{abstract})]
\ \ \text{INV}[s_i(v)/v, v\in V\cup V_\text{abstract}]\big)
\end{align}
The above formula can be solved by an SMT solver.
The result is a concrete mapping of variables from $V$ to concrete values.
Using these concrete values, a concrete test case $\langle s_0,\ldots,s_n\rangle$ is
calculated.
% 
%  \todo[inline]{ohne Gewaehr:
% was die folgende Formel ausdruecken soll:
% konkreter Testfall ergibt sich, indem man (a) die Werte aus dem abstrakten
% Testfall uebernimmt (b) fuer jedes i die Invarianten loesen laesst, um die
% konkreten Werte zu bekommen }
% $<s_0,\ldots,s_n>$ is a concrete test case of $<\bar{s}_0,\ldots,\bar{s}_n>$ if and only if 
% $\bigwedge_{i=0}^{n}(s_i|_{V\cap V_\text{abstract}}= \bar{s}_i|_{V\cap V_\text{abstract}} \land\text{INV}[x/s_i(x),y/\bar{s}_i(y)])$. 

Apart from that, the temporal evolution of system variables is described by means of parametric diagrams. 
In contrast to an invariant, such constraints describe the change of values between two test steps. For example it may necessary to constrain the change of velocity
and location due to the physical laws of acceleration.
Figure~\ref{fig:temporal-evolution} gives an example for this kind of
constraints. This kind of constraints can best be described by differential equations.

In our approach we support the definition of linear differential equations.
These equations can be discretized and translated to expressions relating pre
and post states. These expressions contain unprimed variable symbols $V$ describing
the variable in the pre-state $s_i$, and primed variable symbols~$V'$ describing the
post-state $s_{i+1}$.
%constraints over variables from one system state $s_i$ (unprimed variables) to
% the next system state $s_{i+1}$ (primed variables).
For example the differential equation $\frac{d v}{d t}=a$ in
Figure~\ref{fig:temporal-evolution} can be expressed by the constraint:
$t'=t+\Delta t \land 
{V_\text{est}}'={V_\text{est}}+{a}\cdot\Delta t$.

All constraint properties describing temporal evolution
(TEMP$_1$,\ldots,TEMP$_p$) can be summarized:

$\text{TEMP}=\bigwedge_{k=1}^{p}\text{TEMP}_k$

We extend the SMT instance to generate concrete test cases respecting  the
time-continuous evolution defined by parametric diagrams as follows:

%\begin{align}
%\bigwedge_{i=0}^{n}& \wedge_{v\in V_\text{abstract}} v=\bar{s}_i(v) \\
%&\land
%\text{INV}[\bar{s}_i(v')/v'\in V_\text{abstract},v_i/v\in (V\setminus
%V_\text{abstract})]\\
%&\land\bigwedge_{i=0}^{n}\text{TEMP}[\bar{s}_i(v)/v\in
%V_\text{abstract},\bar{s}_{i+1}(v)/v'\in V'_\text{abstract},\\
%&v_i/v\in (V\setminus
%V_\text{abstract}),v_{i+1}/v'\in (V'\setminus
%V'_\text{abstract})]
%\end{align}

\begin{align}
\bigwedge_{i=0}^{n}& \big(\bigwedge_{t\in V_\text{abstract}} s_i(t) =\bar{s}_i(t) \wedge {}\\
& \ \ 
\text{INV}[s_i(v)/v, v\in V\cup V_\text{abstract}]\big) \wedge  {}  \\
\bigwedge_{i=0}^{n-1} & \text{TEMP}[s_i(v)/v, v \in V\cup V_\text{abstract}, \\
&
\phantom{\text{TEMP}[}
s_{i+1}(v)/v', v'\in V'\cup V_\text{abstract}']
\end{align}



  
% $<s_0,\ldots,s_n>$ is a concrete test case of $<\bar{s}_0,\ldots,\bar{s}_n>$ if and only if
% \begin{align*}
% \bigwedge_{i=0}^{n}&(s_i|_{V\cap V_\text{abstract}}= \bar{s}_i|_{V\cap V_\text{abstract}}\land\text{INV}[x/s(x),y/\bar{s}(y)])\\
% \land\bigwedge_{i=0}^{n-1}&\text{TEMP}[x/s_i(x),x'/s_{i+1}(x)]
% \end{align*} 
