% \chapter{\VV Strategy for a Full Development }
% \label{sec:vv-strategy-full}



The overall strategy is to support the design process as specified in
D2.3 and its partial instantiations within openETCS. In accordance
with the project approach, V\&V shall be done in a FLOSS style, and it
has to suit a model-based development. A further main consideration
shall be to strive for conformance with the requirements of the
standards (EN~50128 and further). Of particular importance in that
respect is of course the interface to the safety considerations and
the contribution of V\&V to the safety case.

Here, the ideal shall be described: How will the V\&V part of a FLOSS
development of the open-source EVC software look like, what are its
constituents, and how do they act together in developing and
maintaining this software within the openETCS ecosystem.

\section{Verification Strategy for a Full Development}
\label{sec:verif-strategy-full}

This section defines the strategy for verifying a full development of
the EVC software from the requirements source
(ss~026+TSIs+\ldots). This ends with the verification of the
software/hardware integration. In current view, the API defines the
interface and relevant properties of the hardware. Thus, SW/HW
integration for openETCS will most probably done virtually with an
instantiation of the API playing the role of the hardware. 

\subsection{System Test Campaigns}
Testing is one of the main means for verification. The three main
features of test campaigns which need to be specified are:
\begin{itemize}
\item Definition of the different campaigns for each design phase and
  their characteristics.
\item Definition of the campaign objectives, and of the kind of test
  to be performed in order to achieve these objectives.
\item Traceability matrix between characteristics to check/ validate
  and test cases.
\end{itemize}

\subsection{V\&V Actions Criteria}
These criteria shall define the acceptance of system or component
modification, and the interruption criteria shall define the condition
under which a test campaign should be stopped for instance. These
criteria have to bet set during the V\&V plan redaction.  During the
verification phase, these items are verified thanks to the actions
criteria list described in the subpart related to the development
phases and items verification.

\subsection{Organization Charts and Responsibilities} 
The plan shall define the roles and responsibilities of people
involved in the V\&V activities (according to standards and project
needs). Must also be defined the teams compositions, and the
deliverables product owners, as well as the group and committee for
modification.  These items are verified during the verification phase.

\subsection{Internal Means and Tools} 
The plan shall list and define the means and tools needed for V\&V
actions (also when they are needed).  These items are verified during
the verification phase.

\subsection{External Means and Tools} 
The plan shall list and define the means and the tools used by the
interfaces with the project, and which company or structure is using
them.  These items are verified during the verification phase.

\subsection{Development Phases and Item Verification}
\bgcmmnt{Needs to be revised. }

The plan for the development phases verification is based on
requiement coverage at system and sub-system level.  The requirements
are usually decomposed and gathered in a table, and this table
constitutes a road-map for verification activities, and is closely
followed in order to figure out if the required items are covered, and
if so, in which document.  The following items give an overview of
items checked during verification phases, regarding the kind of
prescriptions checked, and the appropriate related development phase.

\begin{description}
\item[Introduction:] This part gives an overview of Verification
  general purposes, of main verification phases (Safety Specification
  of Software, architecture, and overall test campaign), and
  verification support formalism (harmonized against different
  Verification activities, can be a simple table).
\item[ Software Security/Safety Prescriptions Specification Phase:]
  Gives a list of prescriptions (specifications top level from a top
  level specification providing the customer needs), and trace their
  coverage (justification and in which document this justification is
  provided).  Here is an example: ``Prescription specification related
  to Software safety should come out from upper level system safety
  prescriptions and from prescriptions from Security schedule.  This
  information should be communicated to the Software developer.''
\item[Design Phases and Software Development:] This phase is based on
  the same approach of the previous part, and prescriptions are
  gathered in a table.  The different categories considered are
  defined in the following parts: (), , detailed design and
  development prescriptions, coding prescriptions, Software modular
  tests and Software integration tests prescriptions.
\item[General Requiements:] For instance: design representations shall
  be based on a clearly defined notations, or limited to clearly
  stated characteristics
\item[Software Architecture Related Requirements:]. For instance: Any
  System Safety prescription modification related to security should
  be documented and acknowledged by the developer.
\item[Support Tools and Coding Languages Requirements:]. For instance:
  According to the type of development, the compliance with the given
  prescriptions is under the software supplier responsibility.
\item[Detailed Design and Development Prescriptions:] For instance:
  ``The Software shall be produced in order to insure the required
  modularity, testability and modification ease.''
\item[Coding Prescriptions:] For instance: ``Each Software code
  module shall be reviewed''
\item[Software Modular Tests Prescriptions:] For instance: ``Modular
  tests shall be documented''
\item[Software Integration Tests Requirements:] For instance:
  ``Integration test specification shall encompass: integration groups
  easily manageable, test cases and test data, test kinds,
  environment, tools configuration on test program, test acceptance
  criteria, corrective actions procedures''
\item[Software Security Validation Planning:] The same formalism as
  previous parts can be used, and this part sums up the different
  requirements/specification for the planning regarding the safety
  activity. For instance: ``Planning shall be managed in order to
  specify technical and process steps necessary to prove that the
  software is compliant with safety prescriptions''.
\item[Programmed Electronic Components (Hardware and Software):] This
  part is specifically related to the integration conditions. This
  part is not supposed to be considered in openETCS, as no
  integration is planned in the project so far.
\item[Software Safety Validation:] For instance: ``Validation
  activities shall be performed as specified in the validation plan''.
\item[Conclusion:] (related to the requirements coverage). In our
  case, the prescriptions are the requirements explained in the WP4
  (these prescriptions have to be refined from the different project
  inputs, such as CENELEC, SRS, FPP\ldots).
\end{description}


\section{Validation Strategy for a Full Development}
\label{sec:valid-strategy-full}

Validation, according to the standard, starts after SW/HW
integration. In this section, it shall be detailed how this should
look like for the openETCS architecture approach (with SSRS and
API). Ideal would be a description of how a full openETCS EVC software
could be taken up by some manufacturer and brought to life in a
product (validation aspect only, of course). Validation will use tests
covering operational scenarios.  

Not-so-classical validation can start earlier when executable models
become available. If a model can be animated to run an operational
scenario (perhaps with some additional environment/rest-of-system
modeling), design defects may get unveiled before the real
validation. This is, however, not an activity which is mentioned as a
development activity in the standard EN~50128. Thus, to use results of
``early validation'' in a validation report requires a definition of
its role and an argument for its usefulness.

\subsection{Validation Case}
The aim of the Validation Report is to demonstrate that the objectives
of validation which have been set in the validation plan have been
achieved. This concerns the adequacy of the software design
documentation and that components and system behavior is compliant
with the software requirements.

The inputs for validation at system level are:
\begin{itemize}
\item Software Requirement Specification (design, architecture),
\item Sub-System Requirement Specification (encompassing the architecture, 
interface description and requirement allocation),
\item Application Programming Interface,
\item Internal and External constraints,
\item Validation constraints list,
\item Validation Plan,
\end{itemize}

The inputs for validation at software level are:
\begin{itemize}
\item Components Requirements Specification,
\item Integration Specification,
\item Integration Report,
\item Test Specifications,
\item Test reports.
\end{itemize}


The outputs are:
\begin{itemize}
\item Software Validation Report (Verdict on software ability to fulfill 
the objectives and functionalities defined in the requirement specification.),
\item Software Validation Verification Report,
\end{itemize}

\section{Safety Interface}
\label{sec:safety-interface}

\bgcmmnt{There are two aspects: (1) Safety requirements come from a hazard
and risk analysis. (2) The main contribution of \vv of the openETCS SW
of the EVC for the safety case of the
EVC (HW and SW) should be the validation report of the software. This
has to prove that the software as implement achieves all its safety
goals which have been assigned to it. These two aspects are adressed
(partly) in D4.2.3. The following text should be adapted.}

The safety activities in the software development process are closely
connected to the \vv activities as these provide the overall system
safety requirements, derive corresponding safety related design
specifications and collect the \vv documentation to build the safety
case. The safety activities are identifying unwanted accidents
resulting in harm and analyses the potential hazards, which could lead
to the harm. Resulting from this in depth analysis certain
requirements are derived, which provide the overall safety goals for
the system. If the initial risk of a hazard resulting in harm has to
be reduced or the possibility of harm shall be eliminated at all
specific safety designs are developed to reach the required risk. The
\vv process has to verify these specifications and validated that the
requirements hold for the developed software. Respectively, a complete
documentation is needed to show in the safety case that this steps
have been done according to the overall \vv plan.

Therefore three main interfaces to the safety activities have to be
maintained in the \vv process:
\begin{itemize}
\item \textbf{Verification of safety design specification}: Based on
  risk control measures documented in the Hazard Log specific safety
  design specifications are derived and written in the backlogs for
  model and code development. As it is done for the overall model and
  code specification the verification activities have to demonstrate
  that all of the safety specifications have been implemented
  properly.
\item \textbf{Validation of safety requirements}: The overall software
  validation has to demonstrated that the safety goals are met by the
  developed product. Therefore the safety requirements which have been
  stated based on potential accidents and accepted risk levels have to
  be validated. As these in many cases requires proof for the absence
  of certain conditions, it is important that \vv activities work in a
  close iterations with the safety activities to ensure that the
  safety requirements are stated in a way that can be validated.
\item \textbf{Creation of \vv documentation}: As the safe case has to
  provide the complete argumentation that all needed steps to ensure a
  qualified and safe development process have been performed properly,
  it is important that the \vv plan and all resulting \vv reports are
  coherent and allow to show a close chain of arguments.
\end{itemize}

