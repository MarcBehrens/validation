\documentclass{article}

\title{Proposal for a Subsection Describing a VnV User Story in the Verification Plan\\Version 2.0}
\author{Hardi Hungar}
\date{July 16, 2014}

\newcommand{\tbi}[1]{$<$\textit{#1}$>$}

% Starts a new line nearly everywhere
\newcommand{\nl}{\mbox{}\\}
\newcommand{\nlskip}[1]{\mbox{}\\[#1]}

%
%Comments
\newcommand{\cmmnt}[1]{\framebox{#1}}
\newcommand{\bgcmmnt}[1]{\nl\framebox{\parbox{.95\textwidth}{#1}}\nl[2mm]}
%\renewcommand{\bgcmmnt}[1]{}
%

\newcommand{\eod}{\nl\rule{.95\textwidth}{1pt}\nl\textit{End of Document}}

\begin{document}
\maketitle

\begin{abstract}
This document contains a proposal for structure and content of an
subsection to the verification or validation plan describing a ``User Story'' .

The proposal should be used as a guideline to check whether all
information is given appropriately. The wording used in this proposal
is by no means mandatory. And if you feel that more information is
useful to describe your activity within the context of openETCS, you
should of course do so. Feel free to add additional categories of
description as adequate. 

Also the \LaTeX{} macros may be changed, though the use of
\texttt{paragraph} and \texttt{subparagraph} enables easy integration
into higher-level documents (they are not numbered automatically,
which may be a draback in other respects). 
\end{abstract}

\subsection{\tbi{headline} }
\bgcmmnt{This subsection declaration has to be inserted into the file 
\texttt{WP41-V02-Project-Verification-Plan.tex}. A label to permit 
referencing the subsection number may be added. Following the 
the subsection declaration an \texttt{input} macro shall be inserted
which refers to the file containing the desciption (the rest
of this template, adequately instantiated. These are the only modifications
to source files of the VnV plan which are necessary.}
This section describes the verification plan of \tbi{partner}. It concerns
\tbi{verification object}. The goal of the activity is to establish \tbi{goal}. 

\bgcmmnt{
Different activities, even if performed by one partner, should be
described in separate ``user stories''. What counts as different
should be judged individually.

There are two categories of goals: One is to do something for
  the EVC software to be developed in openETCS. That is, a design
  artifact is verified here. The other is to do something for the VnV
  methods, eg.\ evaluating or demonstrating the suitability of tools. 
Both goals could be relevant to a user story} 


\paragraph{Object of verification}
\nl
The object of verification is \tbi{name, github ref}. It is from
\tbi{design phase} and represents/describes/implements \tbi{the
  what}. 

\bgcmmnt{Design steps according to the openETCS process described in
  D2.3. Add more detailed characterisation if suitable.} 

\paragraph{Available specification}

\bgcmmnt{Specification to be implemented by the verification object.}

\tbi{References, short description}

\paragraph{Methods and Means}

\bgcmmnt{Which methods are applied, tools, etc. Refer to 
descriptions in Part III of the VnV plan, if available}

\paragraph{Results to be achieved}

\bgcmmnt{A detailed description of the goals.}

\paragraph{Timeline}

\bgcmmnt{Describe the steps to be performed}


\subparagraph{\tbi{Step description}}

\bgcmmnt{What, How, When (dates or Verification Level), State and
  results (if partly of fully done).  Several steps, usually at least
  one for each Verification Level}


\paragraph{Maturity Classification}

The tools applied have the following TRLs (Technology Readiness
Levels):
\begin{description}
\item[\tbi{Tool}:] TRL~\tbi{level}. \cmmnt{reference or short explanation.}
\end{description}

\bgcmmnt{Technology readiness level of the tools in analogy to the definitions
from
\texttt{http://ec.europa.eu/research/participants/data/ref/h2020/
wp/2014\_2015/annexes/h2020-wp1415-annex-g-trl\_en.pdf}:

\begin{description}
\item[TRL 1] basic principles observed
\item[TRL 2] technology concept formulated
\item[TRL 3] experimental proof of concept
\item[TRL 4] technology validated in lab
\item[TRL 5] technology validated in relevant environment (industrially relevant
environment in the case of key enabling technologies)
\item[TRL 6] technology demonstrated in relevant environment (industrially relevant
environment in the case of key enabling technologies)
\item[TRL 7] system prototype demonstration in operational environment
\item[TRL 8] system complete and qualified
\item[TRL 9] actual system proven in operational environment (competitive
manufacturing in the case of key enabling technologies; or in space)
\end{description}
These categories are formulated for ``real'' systems, not verification
tools, so some interpretation of the definitions is needed. For us,
the levels 3 to 6 seem the most probable. SCADE with its simulation
capabilities would be an example of a system of higher TRL which could
be used in verification. RT Tester (the plugin together with the
server installed components) might perhaps be classified as 5 (DLR
guess): It has been used like it would be applied in a real
development, but not extensively (not demonstrated, just
validated. This could be different at the end of the project.).}

The activity shall comply in the following way to the requirements of
a SIL~4 development.  \tbi{Compliance description} 

\bgcmmnt{According to the role the activity would have in a
  development process. Tools must be qualified, depending on their
  usage (e.g., error detection by supplementing activities). If an
  activity is not intended to perform some verification completely,
  state what would be needed for being able to use its result. Qualify
  your statement if you are not sure about your judgment: e.g., guess,
  tentative, informed estimation, or similar.}

\eod

\end{document}